\phantomsection
\addcontentsline{toc}{section}{Distribution}
\section*{Distribution}\label{distribution}
GNU Emacs est un logiciel libre; cela signifie que chacun est libre de
l'utiliser et libre de le redistribuer sous certaines conditions. GNU
Emacs n'est pas dans le domaine public; il est protégé et il y a des
restrictions sur sa distribution, mais ces restrictions sont conçues
pour permettre tout ce qu'un bon citoyen coopérant voudrait faire. Ce
qui n'est pas autorisé est d'essayer d'empêcher les autres de partager
plus d'une version de GNU Emacs qu'ils pourraient obtenir de vous. Les
conditions précises se trouvent dans la GNU General Public License qui
vient avec Emacs et apparaît également dans ce manuel. Voir
l'Annexe\cfap{A}[Copie], page\cfapg{A}.\par 

Une façonon d'obtenir une copie de GNU Emacs est de quelqu'un d'autre
qui l'a. Vous n'avez pas besoin de demander notre permission de le
faire, ou le dire à personne d'autre ; il suffit de copier. Si vous
avez accès à internet, vous pouvez obtenir la dernière version de la
distribution de GNU Emas par FTP anonyme ; voir
\url{://www.gnu.org/software/emacs} sur notre site Web pour plus
d'informations. \par

Vous pouvez également recevoir GNU Emacs lorsque vous achetez un
ordinateur. Les fabriquants d'ordinateurs sont libres de distribuer
des copies dans les mêmes conditions qui s'appliquent à tout le
monde. Ces conditions les obligent à vous donner les sources, y
compris les modifications qu'ils auraient peut-être fait, et vous
permettre de redistribuer GNU Emacs reçu d'eux dans les conditions de
la General Public License. En d'autres termes, le programme doit être
libre pour vous lorsque vous l'obtenez, pas seulement libre pour le
fabriquant. \par

Si vous trouvez GNU Emacs utile, s'il vous plaît envoyez un don à la
Free Software Foundation pour soutenir notre travail. Les dons à la Free
Software Foundation sont déductibles d'impôts aux \'Etats-Unis. Si
vous utilisez GNU Emacs sur votre lieu de travail, s'il vous plaît
suggérez que l'entreprise fasse un don. Pour plus d'informations sur
comment vous pouvez aider, voir
\url{http://www.gnu.org/help/help.html}.\par

Nous vendons aussi des versions papier de ce manuel et Introduction à
la programmation en Emacs Lisp, par Robert J. Chassell. Vous pouvez
visiter notre boutique en ligne \url{http://shop.fsf.org/}. Le revenu
de la vente va soutenir l'objectif de la fondation: le développement
de nouveaux logiciels libres, et l'amélioration de nos programmes
existants, y compris GNU Emacs.\par

Si vous avez besoin de contacter la Free Software Foundation, voir
\url{http://www.fsf.org/about/}, ou écrire à 

Free Software Foundation

51 Franklin Street, Fifth Floor Boston, MA 02110-1301

USA \par


