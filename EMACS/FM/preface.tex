\phantomsection
\addcontentsline{toc}{section}{Préface}
\section*{Préface}\label{preface}
Ce manuel décrit l'utilisation et la personnalisation simple de
l'éditeur Emacs. Les personnalisations simples d'Emacs ne nécessitent
pas que vous soyez un programmeur, mais si vous n'êtes pas intéressé
par la personnalisation, vous pouvez ignorer les conseils de
personnalisation. \par

Il s'agit simplement d'un manuel de référence, mais qui peut également
être utilisé comme une amorce. Si vous êtes nouveau avec Emacs, nous
vous recommandons de commencer avec le tutoriel intégré, apprendre par
la pratique, avant de lire le manuel. Pour lancer le tutoriel, démarrer
Emacs et tapez \texttt{C-h-t}. Le tutoriel décrit les commandes, vous
indique quand les essayer, et explique les résultats. Le tutoriel est
disponible en plusieurs langues.\par

En première lecture, parcourez juste les chapitres\cfch{1}
et\cfch{2}, qui décrivent les différentes conventions de notation
du manuel et l'aspect général de l'écran d'affichage Emacs. Notez que
les questions ont leurs réponses dans ces chapitres, de sorte que vous
pouvez vous y référer plus tard. Après avoir lu le
chapitre\cfch{4}, vous devriez pratiquer les commandes qui y sont
indiquées. Les prochains chapitres décrivent les techniques et les
concepts fondamentaux qui sont utilisés en permanence. Vous avez
besoin de les comprendre en profondeur, afin de les expérimenter
jusqu'à ce que vous soyez à l'aise. \par 

Les chapitres\cfch{14} à\cfch{19} décrivent les
fonctionnalités de niveau intermédiaire qui sont utiles pour de
nombreux types d'édition. Le chapitre\cfch{20} et les suivants
décrivent des fonctions facultatives mais utiles ; lisez ces chapitres
lorsque vous en aurez besoin. \par 

Lisez le chapitre Problèmes Courants (chapitre\cfch{34}) si Emacs
ne semble pas fonctionner correctement. Il explique comment faire face
à plusieurs problèmes communs (voir Section\cfchs{34}{2}
[S'arranger avec les problèmes Emacs], page\cfchsg{34}{2}),
ainsi que quand et comment signaler les bogues Emacs (voir
Section\cfchs{34}{3} [Bugs], page\cfchsg{34}{3}). \par 

Pour trouver la documentation d'une commande particulière, regardez
dans l'index. Touches (commandes de caractères) et noms de commandes
ont des index distincts. Il y a aussi un glossaire, avec un renvoi
pour chaque terme.\par

Ce manuel est disponible sous forme de livre imprimé et aussi en
fichier Info. Le fichier Info est pour la lecture dans Emacs lui-même,
ou avec le programme Info. Info est le format principal de la
documentation dans le système GNU. Le fichier Info et le livre imprimé
contiennent sensiblement le même texte et sont générés à partir des
mêmes fichiers sources, qui sont également distribués avec GNU
Emacs.\par

GNU Emacs est un membre de la famille de l'éditeur Emacs. Il existe de
nombreux éditeurs Emacs, ils partagent tous des principes communs
d'organisation. Pour plus d'informations sur la philosophie
sous-jacente à Emacs et les leçons tirées de son développement, voir
Emacs, l'éditeur extensible, personnalisable et auto-documenté,
disponible sur
\url{ftp://publications.ai.mit.edu/ai-publications/pdf/AIM-519A.pdf}.\par

Cette version du manuel est principalement destinée à être utilisée
avec GNU Emacs installé sur les systèmes GNU et Unix. GNU Emacs peut
également être utilisé sur MS-DOS, Microsoft Windows et les systèmes
Macintosh. La version du fichier Info de ce manuel contient un peu
plus d'informations sur l'utilisation d'Emacs sur ces systèmes. Ces
systèmes utilisent différentes syntaxe de nom de fichier; en plus
MS-DOS ne prend pas en charge toutes les fonctionnalités de GNU
Emacs. Voir l'Annexe\cfap{G} [Microsoft Windows], page 496, pour plus
d'informations sur l'utilisation d'Emacs sous Windows. Voir l'Annexe\cfap{F}
[Mac OS / GNUstep], page\cfapg{F}, pour plus d'informations sur
l'utilisation d'Emacs sur Macintosh (et GNUstep).\par
