\phantomsection
\addcontentsline{toc}{section}{Introduction}
\section*{Introduction}\label{introduction}
Vous lisez sur GNU Emacs, l'incarnation de l'éditeur GNU,
auto-documenté, avancé, personnalisable et extensible Emacs. (Le <<G>>
de <<GNU>> n'est pas muet.)\par 

Nous appelons Emacs avancé, car il peut faire beaucoup plus que de la
simple insertion et suppression de texte. Il peut contrôler des
sous-processus, indenter automatiquement des programmes, afficher
plusieurs fichiers à la fois, et plus encore. Les commandes d'édition
Emacs fonctionnent en termes de caractères, mots, lignes, phrases,
paragraphes, et pages, ainsi que les expressions et commentaires dans
différents langages de programmation.\par 

Auto-documenté signifie qu'à tout moment vous pouvez utiliser des
commandes spéciales, appelées commandes d'aide, pour savoir quelles
sont vos options, ou pour savoir ce que n'importe quelle commande
fait, ou de trouver toutes les commandes qui se rapportent à un sujet
donné. Voir le chapitre\cfch{7} [Aide], page\cfchg{7}.\par 

Personnalisable signifie que vous pouvez aisément modifier le
comportement des commandes Emacs de façon simple. Par exemple, si vous
utilisez un langage de programmation dans lequel les commentaires
commencent par \texttt{'<**'} et finissent par \texttt{'**>'}, vous
pouvez dire à Emacs de commenter les commandes de manipulation afin
d'utiliser ces chaînes (voir Section\cfchs{23}{5} [Comments],
page\cfchsg{23}{5}). Pour prendre un autre exemple, vous pouvez
relier les commandes de base de déplacement du curseur (haut, bas,
gauche et droite) pour toutes les touches du clavier que vous trouvez
confortable. Voir le chapitre\cfch{33} [Personnalisation],
page\cfchg{33}.\par  

Extensible signifie que vous pouvez aller au-delà de la simple
personnalisation et créer de nouvelles commandes. De nouvelles
commandes sont tout simplement des programmes écrits dans le langage
Lisp, qui sont gérées par un interprète Lisp d'Emacs. Les commandes
existantes peuvent même être redéfinies au milieu d'une session
d'édtition, sans avoir à redémarrer Emacs. La plupart des commandes
d'édition dans Emacs sont écrites en Lisp; les quelques exceptions
auraient pu être écrites en Lisp mais utiliser C à la place est plus
efficace. La rédaction d'une extension est de la programmation, mais
les non-programmeurs peuvent les utiliser par la suite. Voir la
section <<Préface>> dans Introduction à la programmation en Emacs
Lisp, si vous voulez apprendre la programmation Emacs Lisp. \par


