\chapter{Emacs et Mac OS/GNUstep}\label{appF}

Cette section décrit les particularités d'utiliser l'Emacs construit
avec les bibliothèques GNUSTEP sur GNU/LINUX ou d'autres systèmes
d'exploitation, ou sur Mac OS X avec système de fenêtre natif. Sur le
Mac OS X, l'Emacs peut être construit sans support système de fenêtre,
avec X11, ou avec l'interface de Cocoa; cette section s'applique
seulement à la construction Cocoa. Ceci ne concerne  pas les
versions de Mac OS X antérieures à la version 10.4.

Pour des raisons historiques et techniques diverses, l'Emacs utilise
le terme <<\texttt{Nextstep}>> intérieurement, au lieu <<de Cocoa>> ou <<Mac OS
X>>; par exemple, la plupart des commandes et des variables décrites
dans cette section commencent par <<\texttt{ns-}>>, qui est un raccourci pour 
<<\texttt{Nextstep}>>. NeXTstep était une interface d'application
sortie par NeXT Inc pendant les années 1980, dont le Cocoa est un
descendant direct. En dehors du Cocoa, il y a un autre système de
NeXTstep-style : GNUstep, qui est le logiciel libre. À partir de cette
écriture, le support Emacs GNUstep est le statut alpha
(voir\cfaps{F}{4}), mais nous espérons l'améliorer dans l'avenir.

\section{Utilisation de base Emacs sous Mac OS et
  GNUstep}\label{appFsec1} 

Par défaut, les touches \texttt{alt} et \texttt{option} sont les même
que \texttt{Meta}. La touche Mac \texttt{Cmd} est la même que
  \texttt{Super}, et Emacs fournit un jeu de raccourcis clavier
  utilisant cette touche modificatrice comme imitation d'autres
  applications Mac / GNUstep (voir\cfaps{F}{3}). Vous pouvez changer ces
  raccourcis clavier en utilisant la manière habituelle (voir page
  420).

La variable \texttt{ns-right-alternate-modifier} contrôle le
comportement de la touche droite \texttt{alt} et
\texttt{options}. Cette touche se comporte comme la touche gauche si
la valeur est \texttt{left} (par défaut). Une valeur de
\texttt{control}, \texttt{meta}, \texttt{alt}, \texttt{super}, ou
\texttt{hyper} les fait se comporter comme des touches modificatrices;
une valeur \texttt{none} dit à Emacs de les ignorer.

\texttt{S-Mouse-1} ajuste la région à la position du clic, comme
\texttt{Mouse-3} (\texttt{mouse-save-then-kill}); cela ne produit pas
un menu pop up pour changer la face par défaut, comme
\texttt{S-Mouse-1} fait normalement (voir page 76). Ce changement fait
Emacs se comporter plus comme les autres applications Mac / GNUstep.

Quand vous ouvrez ou sauvez des fichiers en utilisant les menus, ou en
utilisant \texttt{Cmd-O} et \texttt{Cmd-S}, Emacs utilise les boîtes
de dialogues pour lire les noms de fichiers.

Sur GNUstep, dans un environnement X-windows vous avez besoin
d'utiliser \texttt{Cmd-C} à la place de \texttt{C-w} ou \texttt{M-w}
pour transférer du texte vers la première sélection X; d'autre part,
Emacs va utiliser la sélection <<clipboard>>. Comme,  \texttt{Cmd-y}
(au lieu de  \texttt{C-y}) pour coller depuis une première sélection X
au lieu de kill-ring ou clipboard.

 
\subsection{Variables d'environnement grabbing}\label{appFsec1subsec1}

Beaucoup de programmes qui peuvent tourner sous Emacs, comme latex ou
man, dépendent de la configuration des variables d'environnement. Si
Emacs est lancé depuis le shell, il héritera automatiquement de ces
variables d'environnement et ses sous-processus en hériteront
aussi. Mais si Emacs est lancé depuis le Finder il n'est pas un
descendant d'aucun shell, donc son environnement de variables n'a pas
été configuré, ce qui provoque souvent que les sous-processus qu'il
déclenche se comportent différemment que s'ils étaient lancés depuis
le shell.

Pour le PATH et MANPATH des variables, un large système de méthodes de
configuration PATH est recommandé sur Mac OS X 10.5 et antérieur, en
utilisant les fichiers \texttt{/etc/paths} et le répertoire
\texttt{/etc/paths.d}. 


\section{Personnalisation Mac / GNUstep}\label{appFsec2}

Emacs peut être personnalisé de plusieurs manières en plus des buffers
de personnalisation standard et le menu d'Options. 

\subsection{Panels de fontes et couleurs}\label{appFsec2subsec1}

Les panels de fontes et couleurs standard Mac / GNUstep sont
accessible via des commandes Lisp. Le panel de fontes peut être obtenu
avec \texttt{M-x ns-popup-font-panel}. Ca configurera par défaut la
fonte de la diapositive la plus récente ou cliquer dessus.

Vous pouvez soulever un panneau coloré avec \texttt{M-x ns-popup-color-panel}
et glisser la couleur que vous voulez sur la face d'Emacs que vous
voulez changer. Le glissement normal altèrera la couleur de premier
plan. Le glissement shift altèrera la couleur de fond. Pour renoncer
(annuler) les changements (de configuration), créer une nouvelle
diapositive et fermer celle qui a subi les altérations.

Très utile dans ce contexte, la liste complète des face est obtenue
avec \texttt{M-x list-faces-display}.

\subsection{Personnalisation spécifique pour Mac OS /
  GNUstep}\label{appFsec2subsec2} 

Les options de personnalisation suivantes sont spécifiques à Nextstep
\texttt{ns-auto-hide-menu-bar}

Non-nil signifie que la barre de menu est cachée par défaut, mais
apparaît si vous déplacez le pointeur de la souris dessus. (Nécessite
Mac OS X 10.6 ou ultérieur.)

\section{Système de fenêtrage sous Mac OS / GNUstep}\label{appFsec3}

Les applications Nextstep reçoivent un nombre spécial d'événements qui
n'ont pas d'équivalent X. Ils sont envoyés comme clés spécialement
définies, ce qui ne correspond à aucune combinaisons de touche. Sous
Emacs, ces événements <<clés>> peuvent être ajoutés aux fonctions
comme n'importe quel macros.

Voici une liste des ces événements.
\begin{description}
\item[\texttt{ns-open-file}] cet événement se produit lorsqu'une autre
  application Nextstep demande à Emacs d'ouvrir un fichier. Une raison
  typique pour cela serait qu'un utilisateur double-clique sur un
  fichier dans le Finder. Par défaut, Emacs répond à cet événement en
  ouvrant un nouveau cadre (frame) et en visitant le fichier dans ce
  cadre \texttt{(ns-find-file)}. Par exception, si le tampon (buffer)
  choisi est le tampon \texttt{*scratch*}, Emacs visite le fichier
  dans le cadre sélectionné.

Vous pouvez changer la façon dont Emacs répond à un événement
\texttt{ns-open-file} en changeant la variable
\texttt{ns-pop-up-frames}. Sa valeur par défaut, \texttt{'fresh'}, est
ce que nous venons de décrire. Une valeur de \texttt{t} signifie que l'on
visite toujours le fichier dans un nouveau cadre. Une valeur de
\texttt{nil} signifie que l'on visite toujours le fichier dans le
cadre existant.
\item[\texttt{ns-open-temp-file}] cet événement se produit lorsqu'une
  autre application demande à Emacs d'ouvrir un fichier
  temporaire. Par défaut, c'est géré par la création d'un événement
  \texttt{ns-open-file}, dont les résultats sont décrits ci-dessus.
\item[\texttt{ns-open-file-line}] certaines applications, telles que
  ProjectBuilder et gdb, demandent non seulement un fichier
  particulier, mais aussi une ligne ou une suite de lignes
  particulières dans le fichier. Emacs gère en visitant ce fichier et
  en soulignant la ligne demandée \texttt{(ns-open-file-select-line)}.
\item[\texttt{ns-drag-file}] cet événement se produit lorsqu'un
  utilisateur fait glisser des fichiers à partir d'une autre
  application dans un cadre Emacs. Le comportement par défaut est
  d'insérer le contenu de tous les fichiers déplacés dans le tampon
  courant \texttt{(ns-insert-files)}. La liste des fichiers déplacés
  est stockée dans la variable \texttt{ns-input-file}.
\item[ns-drag-color] cet événement se produit lorsqu'un utilisateur
  fait glisser une couleur depuis le color well (ou d'une autre
  source) dans un cadre Emacs. Le comportement par défaut est de
  modifier la couleur de premier plan de la zone dont la couleur a été
  glissée sur \texttt{(ns-set-foreground-at-mouse)}. Si cet événement
  est délivré avec le modifieur Shift, Emacs change la couleur de fond
  à la place \texttt{(ns-set-background-at-mouse)}. Le nom de la
  couleur glissée est stockée dans la variable
  \texttt{ns-input-color}.
\item[\texttt{ns-change-font}] cet événement se produit lorsque
  l'utilisateur sélectionne une fonte (police) dans un panel Nextstep
  (qui peut être ouvert avec \texttt{Cmd-t}. Le comportement par
  défaut est d'ajuster la fonte (police) du cadre sélectionné
  \texttt{(ns-respond-to-changefont)}. Le nom et la taille de la fonte
  (police) sélectionnnée sont stockés dans les variables
  \texttt{ns-input-fontsize}, respectivement.
\item[\texttt{ns-power-off}] cet événement se produit lorsque
  l'utilisateur se déconnecte et Emacs est encore en cours
  d'exécution, ou lorsque \texttt{'Quit Emacs'} est choisi dans le
  menu de l'application. Le comportement par défaut est de sauver tous
  les tampons de visite de fichier.  
\end{description}
Emacs permet également aux utilisateurs de faire usage de services
Nextstep, via un ensemble de commandes dont le nom commence par
\texttt{'ns-service'} et se termine par le nom du service. Tapez
\texttt{M-x ns-service-TAB} pour voir la liste de ces commandes. Ces
fonctions opèrent soit sur le texte marqué (le remplaçant par la
suite) ou prennent un argument de chaîne et renvoient le résultat sous
forme de chaîne. Vous pouvez également utiliser la fonction Lisp
\texttt{ns-perform-service} pour passer des chaînes arbitraires à
l'appel services arbitraires et recevoir les résultats en
retour. Notez que vous devrez peut-être redémarrer Emacs pour accéder
aux services nouvellement disponibles.

\section{Support GNUstep}\label{appFsec4}
Emacs peut être construit et tourner sous GNUstep, mais il y a encore
des problèmes à régler. Les développeurs intéressés devraient
contacter \url{emacs-devel@gnu.org}.
