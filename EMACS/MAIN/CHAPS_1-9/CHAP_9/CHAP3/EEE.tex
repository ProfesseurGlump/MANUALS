\chapter{Entrer et sortir de Emacs}\etch{3}
Ce chapitre explique comment entrer dans Emacs, et comment en
sortir.\par

\section{Entrer dans Emacs}\etchs{3}{1}
La façon habituelle d'invoquer Emacs est avec la commande emacs
shell. Dans une fenêtre de terminal fonctionnant sous le système X
Window, vous pouvez exécuter Emacs en arrière-plan avec emacs \&; de
cette façon, Emacs ne sera pas attaché à la fenêtre du terminal, de
sorte que vous pouvez l'utiliser pour exécuter d'autres commandes
shell.\par 

Quand Emacs démarre, le cadre initial affiche un tampon spécial appelé
<<*GNU Emacs*>>. Cet écran de démarrage contient des informations sur
Emacs et des liens vers les tâches courantes qui sont utiles pour les
utilisateurs débutants. Par exemple, en activant le lien <<Tutoriel
Emacs>> Emacs ouvre le tutoriel; ce qui fait la même chose que la
commande \texttt{C-h t (help-with-tutorial)}. Pour activer un lien,
soit déplacer le point sur celui-ci et taper \texttt{RET}, ou cliquer
dessus avec \texttt{mouse-1} (le bouton gauche de la souris).\par  

En utilisant un argument de ligne de commande, vous pouvez demander
à Emacs de visiter un ou plusieurs fichiers dès qu'il démarre. Par
exemple, \texttt{emacs foo.txt} démarre Emacs avec un tampon affichant
le contenu du fichier <<\texttt{toto.txt}>>. Cette fonctionnalité
existe principalement pour la compatibilité avec d'autres éditeurs,
qui sont conçus pour être lancés à partir de l'enveloppe pour les
sessions d'éditions courtes. Si vous appelez Emacs de cette façon, le
cadre initial est divisé en deux fenêtres ---une montrant le fichier
spécifié, et l'autre montrant l'écran de démarrage. Voir le
chapitre\cfch{17} [fenêtres], page\cfchg{17}.\par

En général, il est inutile et coûteux de commencer un nouveau Emacs
chaque fois que vous souhaitez éditer un fichier. La méthode
recommandée pour utiliser Emacs est de commencer juste une fois, juste
après que vous vous soyez connecté faites toute votre édition dans la
même session Emacs. Voir le chapitre\cfch{15} [fichiers],
page\cfchg{15}, pour plus d'informations sur la visite de plus
d'un fichier. Si vous utilisez Emacs de cette façon, la session Emacs
accumule des données précieuses, comme l'anneau de kill, les
registres, l'historique des actions (à défaire), l'anneau des données
de marques, qui forment un ensemble d'édition plus pratique. Ces
caractéristiques sont décrites plus loin dans le manuel.\par 

Pour éditer un fichier d'un autre programme tout en gardant Emacs en
cours, vous pouvez utiliser le programme \texttt{emacsclient} d'aide
pour ouvrir un fichier dans la session Emacs existante. Voir la
section\cfchs{31}{4} [serveur Emacs],
page\cfchsg{31}{4}.\par 

Emacs accepte d'autres arguments en ligne de commande qui lui
indiquent de charger certains fichiers Lisp, où mettre la trame
initiale, et ainsi de suite. Voir l'annexe\cfap{C} [invocation
Emacs], page\cfapg{C}.\par 

Si la variable \texttt{inhibit-startup-screen} est non-\texttt{nil},
Emacs n'affiche pas l'écran de démarrage. Dans ce cas, si un ou
plusieurs fichiers ont été spécifiés sur la ligne de commande, Emacs
affiche simplement ces fichiers, sinon, il affiche un tampon appelé
\texttt{*scratch*}, qui peut être utilisé pour évaluer de façon
interactive les expressions Emacs Lisp. Voir la
section\cfchs{24}{10} [intéraction Lisp],
page\cfchsg{24}{10}. Vous pouvez régler la variable
\texttt{inhibit-startup-screen} en utilisant le menu de
personnalisation facile (voir section\cfchs{33}{1}
[personnalisation facile], page\cfchsg{33}{1}), ou en éditant
votre fichier d'initialisation (voir section\cfchs{33}{4} [fichier
d'initialisation], page\cfchsg{33}{4})\footnote{Le réglage
  de la variable \texttt{inhibit-startup-screen} dans le programme
  \texttt{site-start.el} ne fonctionne pas, parce que l'écran de
  démarrage est mis en place avant la lecture du programme
  \texttt{site-start.el}. Voir section\cfchs{33}{4} [fichier
  d'initialisation], page\cfchsg{33}{4}, pour plus
  d'information à propos de \texttt{site-start.el}}.\par 

Vous pouvez également forcer Emacs pour afficher un fichier ou un
répertoire au démarrage en définissant la variable
\texttt{initial-buffer-choice} à une valeur non-\texttt{nil}. (Dans ce
cas, même si vous spécifiez un ou plusieurs fichiers en ligne de
commande, Emacs s'ouvre mais ne les affiche pas). La valeur de
\texttt{initial-buffer-choice} devrait être le nom du fichier ou du
répertoire souhaité.\par 

  
\section{Sortir de Emacs}\etchs{3}{2}
\begin{description}
\item[\texttt{C-x C-c}] kill (quitter) Emacs
  (\texttt{save-buffers-kill-terminal}).
\item[\texttt{C-z}] dans un terminal texte, suspend Emacs; en mode
  graphique, iconise le cadre sélectionné (\texttt{suspend-emacs}).
\end{description}

Kill Emacs signifie que le programme Emacs va se fermer. Pour faire cela,
tapez la séquence \texttt{C-x C-c} (\texttt{save-buffers-kill-terminal}). Une
séquence de touches à deux caractères est utilisée pour rendre
difficile de la taper par accident. S'il y a des tampons de fichiers
visités modifiés quand vous tapez \texttt{C-x C-c}, Emacs vous propose
d'abord de sauvegarder les tampons. Si vous ne les sauvegardez pas
tous, il vous demande encore la confirmation, alors les fichiers non
sauvés seront perdus. Emacs demande également confirmation si des
sous-processus sont encore en cours, car quitter Emacs va aussi clôre
les sous-processus (voir section\cfchs{31}{3} [Shell],
page\cfchsg{31}{3}).\par 

\texttt{C-x C-c} se comporte spécialement si Emacs est utilisé comme
serveur. Si vous la tapez à partir d'un <<cadre client>>, il ferme la
connexion du client. Voir section\cfchs{31}{4} [serveur Emacs],
page\cfchsg{31}{4}.\par 

Emacs peut, le cas échéant, enregistrer certaines informations de
session quand vous le quittez, tels que les fichiez que vous visitiez
à ce moment là. Cette information est alors disponible lors de la
prochaine session Emacs. Voir section\cfchs{31}{8} [sauvegarde de
sessions Emacs], page\cfchsg{31}{8}.\par 

Si la valeur de la variable \texttt{confirm-kill-emacs} est
non-\texttt{nil}, \texttt{C-x C-c} suppose que cette valeur est une
fonction de prédicat, et appelle cette fonction. Si le résultat de la
fonction appelée est non-\texttt{nil}, la session est terminée, sinon
Emacs continue à tourner. Une fonction pratique à utiliser comme
valeur de \texttt{confirm-kill-emacs} est la fonction
\texttt{yes-or-no-p}. La valeur par défaut de
\texttt{confirm-kill-emacs} est \texttt{nil}.\par 

Pour quitter Emacs sans être invité à sauvegarder, tapez \texttt{M-x
  kill-emacs}. 

\texttt{C-z} lance la commande \texttt{suspend-frame}. En mode
graphique, cette commande iconise le cadre Emacs sélectionné, le
cachant de façon à vous laisser le ramener plus tard (la façon exacte
avec laquelle il se cache dépend du système de fenêtrage). Sur un
terminal de texte, la commande \texttt{C-z} suspend Emacs, arrêtant
temporairement le programme et retournant vers le processus parent
(habituellement un shell); dans beaucoup de shells, vous pouvez
reprendre Emacs après sa suspension avec la commande shell
\texttt{\%emacs}.\par 

Les terminaux textuels ont l'habitude de recevoir des caractères
spéciaux pour quitter ou suspendre un programme en cours. Cette
fonction du terminal est éteinte pendant que vous êtes dans Emacs. Les
significations de \texttt{C-z} et de \texttt{C-x C-c} comme touches
dans Emacs ont été inspirées par l'utilisation de \texttt{C-z} et
\texttt{C-c} sur plusieurs systèmes d'exploitation comme caratères
pour interrompre ou clôre un programme, mais c'est leur seul lien avec
le système d'exploitation. Vous pouvez personnaliser ces touches pour
exécuter les commandes de votre choix (voir
section\cfchss{33}{3}{1} [claviers],
page\cfchssg{33}{3}{1}).\par 

