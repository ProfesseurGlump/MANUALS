%!utf-8
\chapter{\'Edition de commandes de base}\etch{4}
Nous expliquons ici les bases de la saisie de texte, faire des
corrections, et enregistrer le texte dans un fichier. Si ce matériel
est nouveau pour vous, nous vous suggérons d'abord d'exécuter le tutoriel Emacs
apprendre par la pratique, en tapant \texttt{C-h t}
(\texttt{help-with-tutorial}). \par

\section{Insertion de texte}\etchs{4}{1}
Vous pouvez insérer un caractère ordinaire graphique (par exemple,
<<a>>, <<B>>, <<3>>, et <<=>>) en tapant sur la touche associée. Cela
ajoute le caractère au point dans le tampon. Le point d'insertion se
déplace vers l'avant, de sorte que le point reste juste après le
texte inséré. Voir la section\cfchs{1}{1} [point],
page\cfchsg{1}{1}. \par

Pour terminer une ligne et en commencer une nouvelle, tapez \texttt{RET}
(\texttt{newline}). (La touche \texttt{RET} peut être étiquetée
\texttt{Return} ou \texttt{Enter} sur votre clavier, mais nous nous
référerons à elle comme \texttt{RET} dans ce manuel). Cette commande
insère un caractère de nouvelle ligne dans le tampon. Si le point est
en fin de ligne, l'effet est de créer une ligne blanche après lui; si
le point est au milieu de la ligne, la ligne est partagée à cette
position.\par

Comme nous l'expliquons plus loin dans ce manuel, vous pouvez changer
la façon dont Emacs gère l'insertion de texte en basculant sur les
modes mineurs. Par exemple, le mode \emph{mineur} appelé Auto Fill
partage automatiquement les lignes lorsqu'elles sont trop longues
(voir section\cfchs{22}{5} [remplissage],
page\cfchsg{22}{5}). Le mode mineur appelé Overwrite mode
provoque l'effacement des caractères existant par insertion (<<écriture par
dessus>>) de nouveaux, au lieu de pousser vers la droite. Voir
section\cfchs{20}{2} [modes mineurs],
page\cfchsg{20}{2}.\par 

Seuls les caractères graphiques peuvent être insérés en tapant sur la
touche associée; les autres touches fonctionnent comme les commandes
d'édition et ne s'insèrent pas. Par exemple, \texttt{DEL} exécute la
commande \texttt{delete-backward-char} par défaut (certains modes se
lient à une commande différente); il n'insère pas de caractère
littéral <<\texttt{DEL}>> (de code caractère ASCII 127). \par

Pour insérer un caractère non-graphique, ou un caractère que votre
clavier ne prend pas en charge, d'abord citez-le en tapant
\texttt{C-q} (\texttt{quoted-insert}). Il y a deux façons d'utiliser
\texttt{C-q} :
\begin{itemize}
\item \texttt{C-q} suivi d'un caractère non graphique (même
  \texttt{C-q}) insère ce caractère. Par exemple, \texttt{C-q DEL}
  insère un caractère littéral <<\texttt{DEL}>>.
\item \texttt{C-q} suivi d'une séquence de chiffres octaux insère le
  caractère correspondant à ce code octal. Vous pouvez utiliser
  n'importe quel nombre de chiffres en octal; tout non-chiffre termine
  la séquence. Si le caractère de terminaison est \texttt{RET},
  \texttt{RET} ne sert qu'à mettre fin à la séquence. Tout autre
  non-chiffre termine la séquence et agit alors comme une saisie
  normale donc, \texttt{C-q 1 0 1} insère <<\texttt{AB}>>.
\end{itemize}
L'utilisation de séquences octale est désactivée en mode Overwrite
non-binaire ordinaire, pour vous donner un moyen pratique d'insérer un
chiffre au lieu d'effacer avec elle (la séquence).\par

Pour utiliser le décimal ou l'hexadécimal au lieu de l'octal,
définissez la variable \texttt{read-quoted-char-radix} à 10 ou 16. Si
la base est 16, les lettres \texttt{a} et \texttt{f} servent dans
le cadre d'un code de caractère, tout comme les chiffres. La casse est
ignorée.\par 

Alternativement, vous pouvez utiliser la commande \texttt{C-x 8 RET}
(\texttt{insert-char}). C'est l'invite pour le nom Unicode code-point
d'un caractère, en utilisant le mini-tampon. Si vous entrez un nom, la
commande fournit la fin (voir section\cfchs{5}{4} [complétion],
page\cfchsg{5}{4}). Si vous entrez un code-point, il doit être
aussi un nombre hexadécimal (la convention pour Unicode), ou un nombre
avec une base spécifique, par exemple, \texttt{\#{}o23072} (octal); voir
section <<bases entières>> dans le manuel de référence Emacs Lisp. La
commande insère ensuite le caractère correspondant dans le tampon. Par
exemple, les deux éléments suivants insèrent le signe de l'infini
(code-point Unicode \texttt{U +221 E}):
\begin{description}
\item[\texttt{C-x 8 RET infinity RET}] 
\item[\texttt{C-x 8 RET 221e RET}] 
\end{description}
Un argument numérique à \texttt{C-q} ou \texttt{C-x 8 RET} spécifie le
nombre de copies du caractère à insérer (voir section\cfchs{4}{10}
[Arguments], page\cfchsg{4}{10}). \par

\section{Changement d'emplacement du curseur}\etchs{4}{2}
Pour faire plus de caractères d'insertion, vous devez savoir comment
déplacer le point (voir la section\cfchs{1}{1} [point],
page\cfchsg{1}{1}). Les commandes \texttt{C-f, C-b, C-n}, et
\texttt{C-p} déplacent le point vers la droite, la gauche, le bas et le
haut, respectivement. Vous pouvez aussi utiliser les flèches présentes
sur la plupart des claviers: droite, gauche, bas et haut; cependant,
beaucoup d'utilisateurs Emacs trouvent qu'il est plus lent d'utiliser
les touches flèches que les touches de commandes, car vous avez besoin
de déplacer votre main vers la zone du clavier où se trouvent les
flèches.\par 

Vous pouvez aussi faire un clic gauche pour déplacer le point à
l'endroit souhaité. Emacs fournit aussi une variété de commandes
clavier additionnelles pour déplacer le point de façons plus
sophistiquées.\par
\begin{description}
\item[\texttt{C-f}] déplacement d'un caractère vers l'avant
  (\texttt{forward-char}). 
\item[\texttt{right}] cette commande (\texttt{right-char}) qui a le
  même effet que \texttt{C-f}, à l'exception près : quand on édite de
  droite à gauche comme par exemple pour l'Arabe, cela fait alors un
  déplacement \emph{arrière} si le paragraphe courant est de droite à
  gauche. Voir section\cfchs{19}{20} [édition bidirectionnelle],
  page\cfchsg{19}{20}. 
\item[\texttt{C-b}] déplacement arrière (\texttt{backward-char}).
\item[\texttt{left}] cette commande (\texttt{left-char}) se comporte
  comme \texttt{C-b}, excepté que le déplacement est vers
  l'\emph{avant} si le paragraphe en cours est de droite à
  gauche. Voir section\cfchs{19}{20} [édition bidirectionnelle],
  page\cfchsg{19}{20}.   
\item[\texttt{C-n}]
\item[\texttt{down}] descend d'une ligne vers le bas
  (\texttt{next-line}). Cette commande laisse la position horizontale
  inchangé, donc si vous commencez au milieu d'une ligne, vous vous
  déplacerez au milieu de la ligne suivante.
\item[\texttt{C-p}]
\item[\texttt{up}] monte d'une ligne vers le haut
  (\texttt{previous-line}). Cette commande préserve la position dans
  la ligne, comme \texttt{C-n}.
\item[\texttt{C-a}]
\item[\texttt{Home}] va au début de la ligne
  (\texttt{move-beginning-of-line}). 
\item[\texttt{C-e}]
\item[\texttt{End}] va à la fin de la ligne
  (\texttt{move-beginning-of-line}).
\item[\texttt{M-f}] avance d'un mot (\texttt{forward-word}).
\item[\texttt{C-right}]
\item[\texttt{M-right}] cette commande (\texttt{right-word}) se
  comporte comme \texttt{M-f}, excepté qu'il se déplace en
  \emph{arrière} mot par mot si le paragraphe en cours est de droite à
  gauche. Voir section\cfchs{19}{20} [édition bidirectionnelle],
  page\cfchsg{19}{20}.  
\item[\texttt{M-b}] recule d'un mot (\texttt{backward-word}). 
\item[\texttt{C-left}]
\item[\texttt{M-left}] cette commande (\texttt{left-word}) se
  comporte comme \texttt{M-f}, excepté qu'il se déplace en
  \emph{avant} mot par mot si le paragraphe en cours est de droite à
  gauche. Voir section\cfchs{19}{20} [édition bidirectionnelle],
  page\cfchsg{19}{20}. 
\item[\texttt{M-r}] sans déplacer le texte de l'écran, reposition du
  point sur la marge de gauche de la ligne de texte la plus centrée de
  la fenêtre; en conséquence des invocations successives, déplacent le
  point vers la marge de gauche de la ligne la plus haute, la plus
  basse, et ainsi de suite, de façon cyclique
  (\texttt{move-to-window-line-top-bottom}). 

  Un argument numérique dit à quelle ligne de l'écran placer le point,
  compter en descendant depuis la ligne du haut de la fenêtre (zéro
  signifie la ligne la plus haute). Un argument négatif compte les
  lignes depuis le bas (-1 signifie la ligne la plus basse). Voir
  section\cfchs{4}{10} [arguments], page\cfchs{4}{10}, pour plus
  d'informations sur les arguments numériques.
\item[\texttt{M-<}] accède au haut du tampon
  (\texttt{beginning-of-buffer}). Avec un argument numérique $n$,
  déplace de $n/10$ depuis le haut.
\item[\texttt{M->}] accède au bas du tampon (\texttt{end-of-buffer}). 
\item[\texttt{C-v}]
\item[\texttt{PageDown}]
\item[\texttt{next}] défile l'affichage de l'écran vers l'avant, et
  déplace le point sur l'écran si nécessaire
  (\texttt{scroll-up-command}). Voir section\cfchs{11}{1}
  [défilement], page\cfchs{11}{1}.
\item[\texttt{M-v}]
\item[\texttt{PageUp}]
\item[\texttt{prior}] défile l'affichage de l'écran vers l'arrière, et
  déplace le point sur l'écran si nécessaire
  (\texttt{scroll-down-command}). Voir section\cfchs{11}{1}
  [défilement], page\cfchs{11}{1}.
\item[\texttt{M-g c}] lit un nombre $n$ et déplace le point à la
  position $n$ du tampon. Position 1 est le début du tampon.
\item[\texttt{M-g M-g}]
\item[\texttt{M-g g}] lit un nombre $n$ et déplace le point au début
  de la ligne $n$ (\texttt{goto-line}). La ligne 1 est le début du
  tampon. Si le point est dessus ou juste après un nombre dans le
  tampon, c'est par défaut $n$. Juste tapez \texttt{RET} dans le mini
  tampon pour l'utiliser. Vous pouvez aussi spécifier $n$ en donnant à
  \texttt{M-g M-g} un argument numérique préfixé. Voir
  section\cfchs{16}{1} [sélection du tampon], page\cfchs{16}{1}, pour
  le comportement de \texttt{M-g M-g} quand vous lui donnez un
  argument préfixé.
\item[\texttt{M-g TAB}] lit un nombre $n$ et va à la colonne $n$ de la
  ligne courante. La colone 0 est celle la plus à gauche. Si elle est
  appelée avec un argument préfixé, elle va à la colonne du numéro
  spécifié par la valeur de l'argument numérique.
\item[\texttt{C-x C-n}] utilise la colonne courante du point comme
  \emph{colonne but semi-permanente} pour \texttt{C-n} et \texttt{C-p}
  (\texttt{set-goal-column}). Quand une colonne but semi-permanente
  est activée, ces commandes essaient toujours d'aller vers cette
  colonne, ou le plus près possible, après déplacements verticaux. La
  colonne but reste activée jusqu'à ce qu'elle soit désactivée.
\item[\texttt{C-u C-x C-n}] annule la colonne but. Désormais,
  \texttt{C-n} et \texttt{C-p} essaient de préserver la position
  horizontale, comme d'habitude.
\end{description}

Quand une ligne de texte du tampon est plus longue que la largeur de
la fenêtre, Emacs l'affiche sur plusieurs lignes. Par commodité,
\texttt{C-n} et \texttt{C-p} déplace le point par ligne, comme font
les touches équivalentes \texttt{down} et \texttt{up}. Vous pouvez
forcer ces commandes à se déplacer en accord avec des \emph{lignes
  logiques} (i.e, en accord avec les lignes du texte dans le tampon)
en configurant la variable \texttt{line-move-visual} à \texttt{nil};
si une ligne logique occupe plusieurs lignes d'écran, le curseur saute
alors les lignes d'écran supplémentaires. Pour les détails, voir
section\cfchs{4}{8} [lignes continues], page\cfchsg{4}{8}. Voir
section\cfchs{33}{2} [variables], page\cfchsg{33}{2}, pour comment
configurer les variables comme \texttt{line-move-visual}.\par

Contrairement à \texttt{C-n} et \texttt{C-p}, la plupart des commandes
Emacs qui fonctionnent sur les lignes fonctionnent sur les lignes
\emph{logiques}. Par exemple, \texttt{C-a}
(\texttt{move-beginning-of-line}) et \texttt{C-e}
(\texttt{move-end-of-line}) respectivement déplacent au début et à la
fin d'une ligne logique. Chaque fois que nous rencontrons des
commandes qui travaillent sur les lignes d'écrans, comme \texttt{C-n}
et \texttt{C-p}, nous les signalerons.\par

Quand \texttt{line-move-visual} vaut \texttt{nil}, vous pouvez aussi
configurer la variable \texttt{track-eol} avec une valeur
non-\texttt{nil}. Alors \texttt{C-n} et \texttt{C-p}, quand le
démarrage est à la fin d'une ligne logique, va à la fin de la
prochaine ligne logique. Normalement, \texttt{track-eol} vaut
\texttt{nil}.\par

\texttt{C-n} s'arrête normalement à la fin du tampon quand vous
l'utilisez sur la dernière ligne du tampon. Toutefois, si vous
configurez la variable \texttt{next-line-add-newlines} avec une
valeur non-\texttt{nil}, \texttt{C-n} sur la dernière ligne du tampon
crée une ligne supplémentaire à la fin et se déplace à la fin de cette dernière.



\section{Effacement du texte}\etchs{4}{3}

\begin{description}
\item[\texttt{DEL}]
\item[\texttt{Backspace}] efface le caractère avant le point, ou la
  région si elle a été sélectionnée (\texttt{delete-backward-char}).
\item[\texttt{Delete}] efface le caractère après le point, ou la
  région si elle a été sélectionnée (\texttt{delete-forward-char}).
\item[\texttt{C-d}] efface le caractère après le point
  (\texttt{delete-char}).
\item[\texttt{C-k}] coupe la ligne (\texttt{kill-line}).
\item[\texttt{M-d}] coupe en avant jusqu'au prochain mot
  (\texttt{kill-word}).
\item[\texttt{M-DEL}] coupe en arrière jusqu'au mot précédent
  (\texttt{backward-kill-word}).  
\end{description}
La commane \texttt{DEL} (\texttt{delete-backward-char}) efface le
caractère avant le point, déplace le curseur et les caractères qui le
suivent avant.\par

Si, toutefois, la région est sélectionnée, \texttt{DEL} efface le
texte de la région. Voir chapitre\cfch{8} [marquage], page\cfchg{8},
pour une description de la région.\par

Sur la plupart des claviers, \texttt{DEL} est appelée
\texttt{Backspace}, mais nous ferons référence à \texttt{DEL} dans ce
manuel. (Ne pas confondre \texttt{DEL} avec la touche \texttt{Delete}
(suppr.); nous discuterons de \texttt{Delete} momentannément.) Sur
certains terminaux, Emacs ne reconnaît pas correctement la touche
\texttt{DEL}. Voir section\cfchss{34}{2}{1} [DEL ne supprime pas],
page\cfchssg{34}{2}{1}, si vous rencontrez ce problème.\par 

La commande \texttt{delete} (\texttt{delete-forward-char}) supprime
dans le <<sens opposé>> : elle supprime après le point, i.e le
caractère sous le curseur. Si le point était en fin de ligne, cela
joint la ligne suivante à celle en cours. Comme \texttt{DEL}, cela
supprime le texte si la région est sélectionnée (voir chapitre\cfch{8}
[marquage], page\cfchg{8}).\par

\texttt{C-d} (\texttt{delete-char}) supprime le caractère après le
point, comme \texttt{delete}, mais sans tenir compte de la région
sélectionnée.\par 

Voir section\cfchss{9}{1}{1} [suppression], page\cfchssg{9}{1}{1},
pour plus d'information détaillée sur les commandes de
suppression.\par

\texttt{C-k} (\texttt{kill-line}) coupe ligne par ligne. Si vous tapez
\texttt{C-k} au début ou au milieu d'une ligne, cela coupe le texte
jusqu'à la fin de la ligne. Si vous tapez \texttt{C-k} à la fin de la
ligne, cela joint la ligne à la suivante. \par

Voir chapitre\cfch{9} [coupures], page\cfchg{9}, pour plus
d'information à propos de \texttt{C-k} et les commandes reliées.\par

\section{Retour en arrière des exécutions (undoing
  changes)}\etchs{4}{4} 

\begin{description}
\item[\texttt{C-/}] annule la dernière commande effectuée
\item[\texttt{C-x u}]
\item[\texttt{C-\_{}}] la même
\end{description}
Emacs enregistre une liste des changements réalisés dans le tampon,
donc vous pouvez annuler les changements récents. Cet usage de la
commande \texttt{undo}, est lié avec \texttt{C-/} (comme \texttt{C-x
  u} et \texttt{C-\_{}}). Normalement, cette commande annule la dernière
modification, déplace le point là où il était avant la
modification. La commande d'annulation ne s'applique uniquement aux
changements du tampon; vous ne pouvez pas l'utiliser pour annuler les
déplacement du curseur.\par

Bien que chaque commande d'édition fait habituellement une entrée
séparée dans l'enregistrement des annulations, les commandes très
simples peuvent être regroupées ensemble. Quelques fois, une entrée
couvre juste une part d'une commande complexe.\par 

Si vous répétez \texttt{C-/} (ou ses alias), chaque répétition annule
un autre, changement précédent, jusqu'à la limite de la mémoire des
modifications disponible. Si toutes les modifications enregistrées ont
déjà été annulées, la commande d'annulation affiche un message d'erreur
et ne fait plus rien.\par

Pour en apprendre plus sur la commande d'annulation, voir
section\cfchs{13}{1} [annulation de modification],
page\cfchs{13}{1}.\par 

\section{Fichiers}\etchs{4}{5}

Le texte que vous insérez dans le tampon Emacs reste seulement durant
la session Emacs. Pour garder un texte en permanence vous devez le
mettre dans un \emph{fichier}.\par

Supposez qu'il y a un fichier nommé \texttt{test.emacs} dans votre
répertoire principal (home). Pour commencer à l'éditer dans Emacs,
tapez \texttt{C-x C-f test.emacs RET}.\par

Ici le nom du fichier est donné comme un argument de la commande
\texttt{C-x C-f} (\texttt{find-file}). Cette commande utilise le
\emph{mini tampon} pour lire l'argument, et vous tapez \texttt{RET}
pour terminer l'argument (voir chapitre\cfch{5} [mini tampon],
page\cfchg{5}). \par

Emacs obéit à cette commande en \emph{visitant} le fichier: il crée un
tampon, copie le contenu du fichier dedans, et affiche le tampon pour
l'édition. Si vous modifier le texte, vous pouvez \emph{sauvegarder}
le nouveau texte en tapant \texttt{C-x C-s}
(\texttt{save-buffer}). Cela copie les contenus du tampon modifié dans
le fichier \texttt{test.emacs}, ça les rend permanent. Jusqu'à ce que
vous sauvegardiez, les changements n'existent que dans Emacs, et le
fichier \texttt{test.emacs} est inchangé.\par 

Pour créer un fichier, visitez-le avec \texttt{C-x C-f} comme si il
existait déjà. Cela crée un tampon vide, dans lequel vous pouvez
insérer le texte que vous souhaitez mettre dans le fichier. Emacs crée
en fait le fichier la première fois que vous sauvegardez le tampon
avec \texttt{C-x C-s}.\par 

Pour en apprendre plus sur l'utilisation des fichiers dans Emacs, voir
chapitre\cfch{15} [Fichiers], page\cfchg{15}.\par

\section{Aide}\etchs{4}{6}

Si vous oubliez ce que fait une touche, vous pouvez le trouver en tapant
\texttt{C-h k} (\texttt{describe-key}), suivi par la touche qui vous
intéresse; par exemple, \texttt{C-h k C-n} vous dit ce que
\texttt{C-n} fait.\par

La touche préfixée \texttt{C-h} sert pour <<aide>> (help). La touche
\texttt{F1} sert d'alias pour \texttt{C-h}. Outre \texttt{C-h k}, il y
a beaucoup d'autres commandes fournissant différentes sortes
d'aide.\par

Voir chapitre\cfch{7} [Help], page\cfchg{7}, pour les détails.\par

\section{Lignes blanches}\etchs{4}{7}

Voici les commandes et techniques spéciales pour insérer et supprimer
des lignes blanches.\par

\begin{description}
\item[\texttt{C-o}] insère une ligne blanche après le curseur
  (\texttt{open-line}). 
\item[\texttt{C-x C-o}] ne supprime pas tout mais une parmi plusieurs lignes
  blanches consécutives (\texttt{delete-blank-lines}).
\end{description}

Nous avons vu comment \texttt{RET} (\texttt{newline}) commence une
nouvelle ligne du texte. Toutefois, il peut être plus facile de voir
ce que vous faites si vous créez votre première ligne blanche et
ensuite insérez le texte désiré dedans. C'est facile de le faire en
utilisant \texttt{C-o} (\texttt{open-line}), qui insère une nouvelle
ligne après le point mais laisse le point en face de la ligne. Après
\texttt{C-o}, tapez le texte de la nouvelle ligne.\par

Vous pouvez faire de nombreuses lignes blanches en tapant \texttt{C-o}
plusieurs fois, ou en donnant un argument numérique spécifiant combien
de lignes blanches vous créez. Voir section\cfchs{4}{10} [Arguments],
page\cfchs{4}{10} pour savoir comment. Si vous avez un préfixe de remplissage,
la commande \texttt{C-o} insère le préfixe de remplissage sur la nouvelle
ligne, si tapé au début d'une ligne. Voir section\cfchss{22}{5}{3}
[Préfixe de remplissage], page\cfchss{22}{5}{3}.\par

La façon simple de se débarasser des lignes vides supplémentaires est
avec la commande \texttt{C-x C-o} (\texttt{delete-blank-lines}). Si le
point se trouve dans une série de plusieurs lignes vides, \texttt{C-x
  C-o} ne supprime pas tout mais l'une d'elles. Si le point est sur une seule
ligne vierge, \texttt{C-x C-o} la supprime. Si le point est sur une
ligne non vide, \texttt{C-x C-o} supprime toutes les lignes vides
suivantes, s'il en existe.\par

\section{Continuité des lignes}\etchs{4}{8}

Parfois, une ligne de texte dans le tampon  --- une \emph{ligne
  logique}  --- est trop longue pour tenir dans la fenêtre, et Emacs
l'affiche sur deux ou plus \emph{lignes d'écran}. On appelle ça
\emph{l'enroulement de ligne} ou \emph{continuation}, et la longue
ligne logique est appelée une \emph{ligne continue}. En mode
graphique, Emacs indique l'enroulement de ligne avec des petites
flèches incurvées sur les bords gauches et droits de la fenêtre. Sur
un terminal de texte, Emacs indique l'enroulement de ligne par un caractère
<<\textbackslash>> sur la marge de droite. \par

La plupart des commandes qui agissent sur les lignes agissent sur les
lignes logiques, pas sur les lignes d'écran. Par exemple, \texttt{C-k}
coupe une ligne logique. Comme décrit précédemment, \texttt{C-n}
(\texttt{next-line}) et \texttt{C-p} (\texttt{previous-line}) sont des
exceptions spéciales: elles déplacent le point vers le bas et le haut,
respectivement, par une ligne d'écran (voir section\cfchs{4}{2}
[déplacement du point], page\cfchsg{4}{2}).\par 

Emacs peut optionnellement \emph{tronquer} de longues lignes logiques
au lieu de les continuer. Cela signifie que chaque ligne logique
occupe une seule ligne d'écran; si c'est plus grand que la largeur de
la fenêtre, le reste de la ligne n'est pas affiché. En mode graphique,
une ligne tronquée est indiquée par une petite flèche droite dans le
coin; sur un terminal de texte, c'est indiqué par le caractère <<\$>>
dans la marge de droite. Voir section\cfchs{11}{21} [troncature de
ligne], page\cfchsg{11}{21}.\par

Par défaut, les lignes continues sont enroulées sur le bord droit de
la fenêtre. Lorsque l'enroulement peut se réaliser au milieu d'un mot,
les lignes continues peuvent être difficile à lire. La solution
usuelle est de couper la ligne avant qu'elle ne soit trop longue, en
insérant une nouvelle ligne. Si vous préférez, vous pouvez demander à
Emacs d'insérer automatiquement une nouvelle ligne quand une ligne
devient trop grande, en utilisant Auto Fill mode. Voir
section\cfchs{22}{5} [remplissage], page\cfchs{22}{5}.\par

Parfois, vous pouvez avoir besoin d'éditer des fichiers contenant
beaucoup de lignes logiques, et ça peut ne pas être pratique de les
couper toutes en ajoutant des nouvelles lignes. Dans ce cas, vous
pouvez utiliser Visual Line mode, qui rend possible
\emph{l'enroulement de mot}: au lieu de l'enroulement de longues
lignes exactement sur le bord droit de la fenêtre, Emacs les enroule
au bord des mots (i.e, espace ou tabulation) les plus proches du coin
droit de la fenêtre. Visual Line mode redéfinit aussi les commandes
d'édition comme \texttt{C-a, C-n}, et \texttt{C-k} pour opérer sur les
lignes d'écran au lieu des lignes logiques. Voir section\cfchs{11}{22}
[Visual Line Mode], page\cfchs{11}{22}.\par

\section{Information sur la position du curseur}\etchs{4}{9}
Voilà des commandes pour obtenir des informations concernant la taille
et la position de parties du tampon, et pour compter les mots et les
lignes.
\begin{description}
\item[\texttt{M-x what-line}] affiche le numéro de la ligne du point.
\item[\texttt{M-x line-number-mode}] bascule l'affichage automatique
  de la ligne courante. Voir section\cfchs{11}{18} [Option du mode
  ligne], page\cfchsg{11}{18}.
\item[\texttt{M-x column-number-mode}] bascule l'affichage automatique
  de la colonne courante. Voir section\cfchs{11}{18} [Option du mode
  ligne], page\cfchsg{11}{18}.
\item[\texttt{M-=}] affiche le nombre de lignes, mots, et caractères
  présents dans la région (\texttt{count-words-region}). Voir
  chapitre\cfch{8} [Marques], page\cfchg{8}, pour les informations au
  sujet de la région.
\item[\texttt{M-x count-words}] affiche le nombre de lignes, mots, et
  caractères présents dans le tampon. Si la région est active (voir
  chapitre\cfch{8} [Marques], page\cfchg{8}), affiche le nombre pour
  la région à la place.
\item[\texttt{C-=}] affiche le code du caractère après le point, la
  position du caractère du point, et la colonne du point
  (\texttt{what-cursor-position}).
\item[\texttt{M-x hl-line-mode}] active ou désactive le surlignage de
  la ligne courante. Voir section\cfchs{11}{20} [Affichage du
  curseur], page\cfchsg{11}{20}.
\item[\texttt{M-x size-indication-mode}] bascule l'affichage
  automatique de la taille du tampon. Voir section\cfchs{11}{18} [Affichage du
  curseur], page\cfchsg{11}{18}.
\end{description}
\texttt{M-x what-line} affiche le numéro de la ligne courante dans la
zone d'écho. Cette commande est habituellement redondante, parce que
le numéro de ligne courante est montré dans la ligne de mode (voir
section\cfchs{1}{3} [ligne de mode], page\cfchsg{1}{3}). Toutefois, si
vous réduisez le tampon, la ligne de mode montre le numéro de ligne
relatif à la portion accessible (voir section\cfchs{11}{5}
[Réduction], page\cfchsg{11}{5}). Par contraste, \texttt{what-line}
affiche à la fois le numéro de ligne relatif à la région réduite et le
numéro de ligne relatif au tampon entier. \par

\texttt{M-=}(\texttt{count-words-region}) affiche un message
rapportant le nombre de lignes, mots, et caractères dans la région (Voir
  chapitre\cfch{8} [Marques], page\cfchg{8}, pour une explication de
  la région). Avec un argument préfixé, \texttt{C-u M-=}, la commande
  affiche le compte pour le tampon entier.\par

La commande \texttt{M-x count-words} fait le même travail, mais avec
des conventions d'appels différentes. \c{C}a affiche le compte pour la
région si la région est active, et sinon pour le tampon entier. \par

La commande \texttt{C-x =} (\texttt{what-cursor-position}) montre des
informations à propos de la position courante du curseur et le contenu
du tampon à cette position. \c{C}a affiche une ligne dans la zone
d'écho qui ressemble à \c{c}a :
\begin{center}
  \texttt{Char: c (99, \#o143, \#x63) point=28062 of 36168 (78\%) column=53}
\end{center}
Après <<\texttt{Char :}>>, \c{c}a montre le caractère dans le tampon
au niveau du point. Le texte entre les parenthèses montre la
correspondance décimale, octale et hexadécimale des codes de
caractères; pour plus d'informations au sujet de comment \texttt{C-x
  =} affiche l'information de caractère, voir section\cfchs{19}{1}
[Caractères internationaux], page\cfchsg{19}{1}. Après
<<\texttt{point=}>> est la position du point comme caractère compté
(le premier caractère dans le tampon est en position 1, le second en
position 2, et ainsi de suite). Le nombre après \c{c}a est le nombre
total de caractères dans le tampon, et le nombre entre parenthèses
exprime la position comme un pourcentage du total. Après
<<\texttt{column=}>> est la position horizontale du point, on compte
les colonnes depuis le bord gauche de la fenêtre.\par

Si le tampon a été réduit, cela fait qu'une partie du texte au début et
à la fin sont temporairement inaccessible, \texttt{C-x =} affiche du
texte supplémentaire décrivant les rangs accessible. Par exemple,
\c{c}a pourrait afficher \c{c}a :
\begin{center}
  \texttt{Char: C (67, \#o103, \#x43) point=252 of 889 (28\%)
    <231-599> column=0}
\end{center}
où les deux nombres supplémentaires donnent la position la plus petite
et la plus grande des caractères autorisée. Les caractères entre ces
deux positions sont accessibles. Voir section\cfchs{11}{5}
[Réduction], page\cfchsg{11}{5}.\par

\section{Arguments numériques}\etchs{4}{10}
Dans la terminologie mathématique et informatique, \emph{argument}
signifie <<donnée fournie à une fonction ou opération>>. Vous pouvez
donner à n'importe quelle commande Emacs un \emph{argument numérique}
(aussi appelé un \emph{argument préfixé}). Certaines commandes
interprètent l'argument comme une répétition de compte. Par exemple,
donner à \texttt{C-f} un argument de dix provoque le déplacement du
point de dix caractères vers l'avant au lieu d'un seul. Avec ces
commandes, l'absence d'argument est équivalente à un argument de un,
et les arguments négatifs provoquent le déplacement ou l'action dans
le sens opposé.\par

La façon la plus simple de spécifier un argument numérique est de
taper un chiffre et/ou un signe moins tout en maintenant la touche
\texttt{META} enfoncée. Par exemple,
\begin{center}
  \texttt{M-5 C-n}
\end{center}
se déplace vers le bas de 5 lignes. Les touches \texttt{M-1, M-2}, et
ainsi de suite, tout comme \texttt{M--}, sont bornées aux commandes
(\texttt{digit-argument} et \texttt{negative-argument}) qui
configurent un argument numérique de la prochaine
commande. \texttt{Meta--} sans chiffre signifie $-1$.\par

Si vous saisissez plus qu'un chiffre vous devez relâcher la touche
\texttt{META} pour le second chiffre et les suivants. Donc, pour se
déplacer de 50 lignes vers le bas, tapez
\begin{center}
  \texttt{M-5 0 C-n}
\end{center}
Remarquez que cela \emph{ne fait pas} l'insertion de cinq copies de
<<$0$>> et le déplacement d'une ligne vers le bas, comme vous l'aurez
deviné --- le <<$0$>> est traité comme une partie de l'argument
préfixé.\par

(Que faire si vous voulez insérer cinq copies de <<$0$>>? Tapez
\texttt{M-5 C-u 0}. Ici, \texttt{C-u} <<finit>> l'exécution. Remarquez
que cela signifie que \texttt{C-u} s'applique uniquement dans ce
cas. Pour le rôle habituel de \texttt{C-u}, voir ci-dessous.)\par

Au lieu de taper \texttt{M-1, M-2}, et ainsi de suite, une autre façon
de spécifier un argument numérique est de taper \texttt{C-u}
(\texttt{universal-argument}) suivi par quelques chiffres, ou (pour un
argument négatif) un signe moins suivi par des chiffres. Un signe
moins sans chiffre signifie normalement $-1$.\par

\texttt{C-u} seul a le sens spécial de <<quatre fois>> : \c{c}a
multiplie l'argument de la prochaine commande par quatre.\texttt{C-u
  C-u} multiplie par seize. Donc, \texttt{C-u C-u C-f} se déplace de
seize caractères vers l'avant. D'autres combinaisons utiles sont
\texttt{C-u C-n, C-u C-u C-n} (se déplace vers le bas d'une bonne
fraction de l'écran), \texttt{C-u C-u C-o} (crée <<beaucoup>> de
lignes blanches), et \texttt{C-u C-k} (supprime quatre lignes). \par

Vous pouvez utiliser un argument numérique avant une auto-insertion de
caractère afin d'insérer des copies multiples de celui-ci. C'est très
direct quand le caractère n'est pas un chiffre; par exemple,
\texttt{C-u 64 a} insère 64 copies du caractère <<\texttt{a}>>. Mais
\c{c}a ne marche pas pour l'insertion de nombre; \texttt{C-u 6 4 1}
spécifie un argument de 641. Vous pouvez séparer les arguments des
chiffres pour insérer d'autres \texttt{C-u}; par exemple, \texttt{C-u
  64 C-u 1} fait insérer 64 copies du caractères <<1>>.\par

Certaines commandes font attention si il y a un argument, mais ignore
sa valeur. Par exemple la commande \texttt{M-q}
(\texttt{fill-paragraph}) remplit le texte; avec un argument, \c{c}a
justifie le texte aussi. (Voir section\cfchs{22}{5} [Remplissage],
page\cfchsg{22}{5}, pour plus d'information sur \texttt{M-q}.) Pour
ces commandes, c'est suffisant de spécifier l'argument avec un simple
\texttt{C-u}.\par

Certaines commandes utilisent la valeur de l'argument comme une
répétition de compte, mais font quelque chose de spécial quand il n'y
a pas d'argument. Par exemple, la commande \texttt{C-k}
(\texttt{kill-line}) avec un argument $n$ supprime $n$ lignes,
incluant les nouvelles lignes de fin. Mais \texttt{C-k} sans argument
spécial: \c{c}a supprime le texte au dessus de la nouvelle ligne, ou,
si le point est à la fin de la ligne, \c{c}a supprime la ligne
elle-même. Donc, deux \texttt{C-k} sans argument peuvent supprimer une
ligne non blanche, comme \texttt{C-k} avec un argument de un. (Voir
chapitre\cfch{9} [Suppression], page\cfchg{9}, pour plus d'information
sur \texttt{C-k}.)\par

Quelques commandes traitent \texttt{C-u} différemment qu'un argument
ordinaire. Quelques autres peuvent traiter un argument signe moins
différemment d'un argument $-1$. Ces cas non usuels sont décrits quand
ils arrivent; ils existent pour créer une commande individuelle plus
commode, et ils sont documentés dans la chaîne de documentation de
commande.\par 

Nous utilisons le terme \emph{argument préfixé} pour insister sur le
fait que vous tapez ces arguments avant la commande, et pour
les distinguer des arguments du mini-tampon (voir chapter\cfch{5}
[Mini-tampon], page\cfchg{5}), qui sont saisis après l'invocation de la
commande.\par 

\section{Répétition de commande}\etchs{4}{11}

De nombreuses commandes simples, comme celles invoquées avec une touche
simple ou avec \texttt{M-x \emph{command-name} RET}, peuvent être
répétées par leur invocation avec un argument numérique qui sert de
compte de répétition (voir section\cfchs{4}{10} [Arguments]
page\cfchsg{4}{10}). Toutefois, si la commande que vous souhaitez
répéter vous invite à la saisie, ou utilise un argument numérique
d'une autre façon, la méthode ne marchera pas.\par

La commande \texttt{C-x z} (\texttt{repeat}) fournit ne autre façon de
répéter des commandes Emacs plusieurs fois. Cette commande répète la
commande Emacs précédente, qu'elle quelle soit. La répétition d'une
commande utilise le même argument qui était utilisé avant; \c{c}a ne
lit pas de nouveaux arguments à chaque fois.\par

Pour répéter la commande plus d'une fois, tapez des \texttt{z}
supplémentaires: chaque \texttt{z} répète la commande une fois de
plus. La répétition finit quand vous tapez un autre caractère que
\texttt{z}, ou cliquez sur un bouton de la souris.\par

Par exemple, supposez que vous tapez \texttt{C-u 20 C-d} pour
supprimer 20 caractères. Vous pouvez répéter cette commande (incluant
ses arguments) trois fois supplémentaires, pour effacer un total de 80
caractères, en tapant \texttt{C-x z z z}. Le premier \texttt{C-x z}
répète la commande une fois, et chaque \texttt{z} suivant la répète
une fois encore.\par
\newpage 