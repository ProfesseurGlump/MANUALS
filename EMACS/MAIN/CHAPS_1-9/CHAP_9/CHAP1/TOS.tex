\chapter{Organisation de l'écran}\etch{1}
En mode graphique, comme sur GNU/Linux avec le système de fenêtrage X
Window, Emacs occupe une <<fenêtre graphique>>. Sur un terminal en
mode texte, Emacs occupe l'écran du terminal en entier. Nous allons
utiliser le terme de cadre pour désigner une fenêtre graphique ou
l'écran du terminal occupée par Emacs. Emacs se comporte de manière
similaire sur les deux types de cadres. Il commence normalement avec
un seul cadre, mais vous pouvez créer des images supplémentaires si
vous le souhaitez (voir le chapitre\cfch{18} [Cadres],
page\cfchg{18}). \par 

Chaque image est composée de plusieurs régions distinctes. Au sommet du
cadre il y a une barre des menus, ce qui vous permet d'accéder à des
commandes par l'intermédiaire d'une série de menus. En mode graphique,
directement en dessous de la barre des menus il y a une barre
d'outils, une rangée d'icônes qui exécutent des commandes d'édition si
vous cliquez sur elles. Au bas du cadre vous trouverez une zone
d'écho, où des messages d'informations sont affichés et où vous pouvez
saisir des informations quand Emacs le demande.\par 

La zone principale du cadre, en dessous de la barre d'outils (le cas
échéant) et au-dessus de la zone d'écho, est appelée la
fenêtre. Désormais dans ce manuel, nous allons utiliser le mot
<<fenêtre>> dans ce sens (différent du sens usuel). Les systèmes
d'affichage graphique utilisent couramment le mot <<fenêtre>> dans un
sens différent ; mais, comme indiqué ci-dessus, nous nous référons à
ces <<fenêtres graphiques>> comme des <<cadres>>.\par

Une fenêtre Emacs est là où le tampon  ---le texte que vous éditez --- est
affiché. En mode graphique, la fenêtre possède une barre de défilement
sur un côté qui peut être utilisée pour faire défiler le tampon. La
dernière ligne de la fenêtre est une ligne de mode. Cette dernière
affiche diverses informations sur ce qui se passe dans le tampon, par
exemple, s'il y a des modifications non enregistrées, les modes
d'édition qui sont en cours d'utilisation, le numéro de la ligne
actuelle, et ainsi de suite.\par

Lorsque vous démarrez Emacs, il y a normalement une seule fenêtre dans
le cadre. Cependant, vous pouvez subdiviser cette fenêtre
horizontalement ou verticalement pour créer plusieurs fenêtres, dont
chacune peut indépendamment afficher un tampon (voir le
chapitre\cfch{17} [Fenêtres], page\cfchg{17}).\par

\`A tout moment, une fenêtre est la fenêtre sélectionnée. En mode
graphique, la fenêtre sélectionnée affiche un curseur plus important
(généralement plein et glignotant); les autres fenêtres affichent un
curseur moins important (généralement une boîte creuse). Sur un
terminal de texte, il n'existe qu'un seul curseur qui est affiché dans
la fenêtre sélectionnée. Le tampon affiché dans la fenêtre
sélectionnée est appelé le tampon courant, et c'est là que se produit
l'édition. La plupart des commandes Emacs peuvent s'appliquer
implicitement au tampon courant; le texte affiché dans les fenêtres
non sélectionnées est surtout visible pour référence. Si vous utilisez
plusieurs cadres en mode graphique, la sélection d'un cadre
particulier sélectionne une fenêtre dans ce cadre.\par


\section{Point}\etchs{1}{1}
Le curseur de la fenêtre sélectionnée indique l'emplacement où la
plupart des commandes d'édition prennent effet, qui est appelé
point\footnote{Le terme <<point>> vient du caractère <<.>>, qui était la
  commande TECO (le langage dans lequel l'Emacs original a été
  écrit) pour accéder à la position d'édition.}. Beaucoup
de commandes Emacs déplacent le point à différents endroits dans le
tampon; par exemple, vous pouvez placer en cliquant sur le bouton 1 de
la souris (usuellement le gauche) à l'endroit désiré. Par défaut, le
curseur de la fenêtre sélectionnée est dessiné comme un bloc plein et
paraît être un caractère, mais vous devrez penser le point comme entre
deux caractères; il est situé avant le caractère sous le curseur. Par
exemple; si votre texte ressemble à <<frob>> avec le curseur sur <<b>>,
alors le point est entre le <<o>> et le <<b>>. Si vous insérez le
caractère <<!>> à cette position, le résultat sera <<fro!b>>, avec le
point entre le <<!>> et le <<b>>. Ainsi, le curseur reste au-dessus du
<<b>>, comme précédemment.\par 

Si vous éditez plusieurs fichiers dans Emacs, chacun dans son propre
tampon, chaque tampon a sa propre valeur du point. Un tampon qui n'est
pas actuellement affiché se souvient encore de sa valeur du point si vous
l'afficher plus tard. En outre, si un tampon est affiché dans
plusieurs fenêtres, chacune de ces fenêtres a sa propre valeur du
point.\par

Voir la section\cfchs{11}{20} [Affichage Curseur],
page\cfchsg{11}{20}, pour les options qui contrôlent la façon
dont Emacs affiche le curseur.\par 


\section{La zone d'écho}\etchs{1}{2}
La ligne au bas du cadre est la zone d'écho. Elle est utilisée pour
afficher de petites quantités de texte à des fins diverses.\par

La zone d'écho est ainsi nommée parce que l'une de ses
caractéristiques est faire l'écho, ce qui signifie l'affichage des
caractères d'une commande multi-caractères. Les commandes à caractère
unique ne sont pas affichées. Les commandes multi-caractères (voir la
section\cfchs{2}{2} [Clavier], page\cfchsg{2}{2}) sont
reprises si vous vous arrêtez pendant une seconde au milieu d'une
commande. Emacs fait alors l'écho de tous les caractères de la commande
à ce moment-là, pour vous inviter à saisir le reste. Une fois l'écho
commencé, le reste de la commande se fait l'écho à mesure que vous
tapez. Ce comportement est conçu pour donner aux utilisateurs
confiants une réponse rapide, tout en donnant aux utilisateurs
hésitants un maximum de retours. \par

La zone d'écho est également utilisée pour afficher un message
d'erreur quand une commande ne peut pas faire son travail. Les
messages d'erreurs peuvent être accompagnés par un bip ou par un
clignotement de l'écran. \par 

Certaines commandes affichent des messages d'information dans la zone
écho pour vous dire ce que la commande a fait, ou pour vous
fournir des informations spécifiques. Ces messages d'information,
contrairement aux messages d'erreurs, ne sont pas accompagnés d'un
signal sonore ni d'un flash. Par exemple, \texttt{C-x =} (Maintenez la
touche ctrl enfoncée et tapez x, puis relâchez la touche ctrl puis
tapez =) affiche un message décrivant le caractère au point, sa
position dans le tampon, et sa colonne en cours dans la fenêtre. Les
commandes qui prennent beaucoup de temps affichent souvent des
messages se terminant par <<\dots{}>> pendant qu'elles travaillent
(parfois aussi indiquent combien de progrès ont été réalisés, en
pourcentage), et ajouter <<fait>> quand elles ont fini.\par 

Les messages d'informations de la zone d'écho sont sauvegardés dans un
tampon spécial nommé \texttt{*Messages*}. (Nous n'avons pas encore
expliqué les tampons; voir le chapitre\cfch{16} [tampons],
page\cfchg{16}, pour plus d'informations à leur sujet.). Si
vous ratez un message qui est apparu brièvement à l'écran, vous pouvez
changer de tampon et visiter le tampon \texttt{*Messages*} afin de revoir les
messages. Le tampon \texttt{*Messages*} est limité à un certain nombre
de lignes, scpécifié par la variable \texttt{message-log-max}. (Nous
n'avons pas encore expliqué les variables soit, voir la
section\cfchs{33}{2} [Variables], page\cfchsg{33}{2}, pour
plus d'informations sur ce sujet.). Au-delà de cette limite, une ligne
est supprimée depuis le début à chaque fois qu'une nouvelle ligne de
message est ajoutée à la fin.\par 

Voir la section\cfchs{11}{23} [affichage perso],
page\cfchsg{11}{23}, pour les options qui contrôlent la façon
dont Emacs utilise la zone d'écho.\par

La zone d'écho est également utilisée pour afficher le mini-tampon,
une fenêtre spéciale où vous pouvez saisir des arguments de commandes,
telles que le nom d'un fichier à éditer. Lorsque le mini-tampon est en
cours d'utilisation, le texte affiché dans la zone d'écho commence par
une chaîne d'invite, et le curseur actif apparaît dans le mini-tampon,
qui est considérée comme temporairement la fenêtre sélectionnée. Vous
pouvez toujours sortir du mini-tampon en tapant \texttt{C-g}. Voir le
chapitre\cfch{5} [mini-tampon], page\cfchg{5}. \par 


\section{Le mode ligne}\etchs{1}{3}
Au bas de chaque fenêtre il y a une ligne de mode, qui décrit ce qui
se passe dans le tampon en cours. Quand il y a une seule fenêtre, la
ligne de mode apparaît juste au-dessus de la zone d'écho; il s'agit de
l'avant-dernière ligne dans le cadre. En mode graphique, la ligne de
mode est dessiné avec une apparence de boîte 3D. Emacs aussi dessine
habituellement la ligne de mode de la fenêtre sélectionnée avec une
couleur différente de celle des fenêtres non sélectionnées, afin de la
faire ressortir.\par

Le texte affiché dans la ligne de mode est de la forme :
\begin{center}
  \texttt{cs:ch-fr}\quad \texttt{buf}\qquad \texttt{pos line}\quad
  \texttt{(major minor)}
\end{center}
Sur un terminal de texte, le texte est suivi par une série de traits
qui s'étendent vers le bord droit de la fenêtre. Ces traits sont omis
en mode graphique.\par

La chaîne \emph{cs} et le caractère deux-points qui la suit décrivent
le jeu de caractères de la nouvelle ligne et les conventions utilisées
pour le tampon courant. Normalement, Emacs gère automatiquement ces
paramètres pour vous, mais il est parfois utile d'avoir cette
information.\par

\emph{cs} décrit le jeu de caractères du texte dans le tampon (voir
section\cfchs{19}{6} [systèmes d'encodage],
page\cfchsg{19}{6}). S'il s'agit d'un tiret (<<->>), qui
inidique l'absence de caractère spécial (avec de possibles exceptions
de conventions de fin de ligne (césure), décrites dans le paragraphe
suivant). <<=>> signifie aucune conversion que ce soit, et est
habituellement utilisé pour les fichiers contenant des données
non-textuelles. D'autres caractères représentent différents systèmes
d'encodages  ---par exemple, <<1>> représente ISO Latin-1.\par

Sur un terminal de texte, \emph{cs} est précédée par deux caractères
supplémentaires qui décrivent les systèmes d'encodage pour la saisie
au clavier et la sortie affichée. En outre, si vous utilisez une
méthode de saisie, \emph{cs} est précédée par une chaîne qui identifie la
méthode de saisie (voir section\cfchs{19}{4} [méthodes de
saisies], page\cfchsg{19}{4}).\par

Le caractère après \emph{cs} est généralement <<:>>. Si une chaîne de
caractère est affichée, cela indique une convention de fin de ligne
non négligeable pour encoder un fichier. Habituellement, les lignes de
texte sont séparées par des caractères de nouvelle ligne dans un
fichier, mais deux autres conventions sont parfois utilisées. La
convention MS-DOS utilise un caractère <<retour chariot>> suivie d'un
caractère <<saut de ligne>>; lors de l'édition de ces fichiers, le
caractère <<:>> est remplacé par (<<\textbackslash>>) ou <<(DOS)>>,
selon l'OS. Une autre convention, employée par les anciens Mac OS,
utilise le caractère <<retour chariot>> au lieu d'un retour à la
ligne; lors de l'édition de ces fichiers, le caractère <<:>> est
remplacé par (<</>>) ou <<(Mac)>> . Sur certains systèmes, Emacs
affiche <<(Unix)>> au lieu de <<:>> pour les fichiers qui utilisent le
saut de ligne comme séparateur de ligne.\par 

L'élément suivant la ligne de mode est indiqué par la chaîne ch. Cela
montre deux tirets (<< - - >>) si le tampon est affiché dans la fenêtre a
le même contenu que le fichier correspondant sur le disque; à savoir
si le tampon est <<non modifié>>. Si le tampon est modifié, il montre
deux étoiles (<<**>>). Pour un tampon en lecture seule, il montre
<<\%*>> si le tampon est modifié, et <<\%\%>> autrement.\par

Le caractère après le \emph{ch} est normalement un tiret (<<->>). Cependant,
si le répertoire par défaut pour le tampon courant est sur une machine
distante, <<@>> est affiché à la place (voir la
section\cfchs{15}{1} [noms de fichiers],
page\cfchsg{15}{1}). \par

\emph{fr} donne le nom du cadre sélectionné (voir le
chapitre\cfch{18}[cadres], page\cfchg{18}). Il n'apparaît
que sur les terminaux de texte. Le nom initial du cadre est
<<F1>>.\par 

\emph{buf} est le nom du tampon (buffer) affiché dans la
fenêtre. Habituellement, c'est le même nom que le nom d'un fichier que
vous éditez. Voir le chapitre\cfch{16} [tampons],
page\cfchg{16}.\par  

\emph{pos} vous indique s'il y a un texte complémentaire au-dessus du haut de
la fenêtre, ou en dessous. Si votre tampon est petit et tout est
visible dans la fenêtre, pos est <<Tout>> (all). Sinon, il est <<Top>>
si vous êtes à la recherche au début du tampon, <<Bot>> si vous
cherchez à la fin du tampon, ou <<nn\%>>, où nn est le pourcentage du
tampon au-dessus du haut de la fenêtre. Avec le mode d'indication de
taille, vous pouvez afficher la taille du tampon. Voir
section\cfchs{11}{18} [options du mode ligne],
page\cfchsg{11}{18}.\par 

\emph{line} est le caractère <<L>> suivi du numéro de la ligne au
point. (Vous pouvez afficher le numéro de colonne courante aussi, en
activant le mode numéro de colonne. Voir la section\cfchs{11}{18}
[options du mode ligne], page\cfchsg{11}{18}.) \par

\emph{major} est le nom du mode majeur utilisé dans le tampon. Un mode
majeur est un mode d'édition principale pour le tampon, telles que le
mode texte, le mode Lisp, le mode C, et ainsi de suite. Voir la
section\cfchs{20}{1} [Les principaux modes],
page\cfchsg{20}{1}. Certains modes majeurs affichent des
informations supplémentaires après le nom du mode majeur. Par exemple,
des tampons de compilation et des tampons de Shell affichent l'état du
sous-processus. \par 

\emph{minor} est une liste de quelques-uns des modes mineurs permis,
qui sont les modes d'édition en option qui offrent des fonctionnalités
supplémentaires au-dessus du mode majeur. Voir la
section\cfchs{20}{2} [modes mineurs],
page\cfchsg{2}{2}. \par

Certaines fonctions sont répertoriées avec les modes mineurs quand ils
sont allumés, même si elles ne sont pas vraiment des modes
mineurs. <<Narrow>> signifie que le tampon est affiché à l'édition
limitée à seulement une partie de son texte (voir la
section\cfchs{11}{5} [rétrécissement],
page\cfchsg{11}{5}). <<Def>> signifie qu'une macro clavier est
en cours de définition (voir le chapitre\cfch{14} [macros
clavier], page\cfchg{14}).\par 

En outre, si Emacs est à l'intérier d'un niveau d'édition récursive,
les crochets (<<[\dots{}]>>) apparaissent autour des paranthèses qui
entourent les modes. Si Emacs est à un niveau d'édition récursive à
l'intérieur d'un autre une paire de crochets doubles apparaît, et
ainsi de suite. Puisque les niveaux d'édition récursifs affectent
globalement Emacs, ces crochets apparaissent dans la ligne de mode de
chaque fenêtre. Voir la section\cfchs{31}{9} [édition récursive],
page\cfchsg{31}{9}.\par 

Vous pouvez modifier l'apparence de la ligne de mode ainsi que le
format de son contenu. Voir section\cfchs{11}{18} [options de mode
ligne], page\cfchsg{11}{18}. En outre, la ligne de mode est
sensible à la souris; en cliquant sur les différentes parties de la
ligne de mode on effectue diverses commandes. Voir la
section\cfchs{18}{5} [mode ligne avec souris],
page\cfchsg{18}{5}.\par
 

\section{La barre de menu}\etchs{1}{4}
Chaque cadre d'Emacs a normalement une barre de menu en haut que vous
pouvez utiliser pour effectuer des opérations communes. Il n'est pas
nécessaire de les énumérer ici, comme vous pouvez le constater par
vous-même. \par

En mode graphique, vous pouvez utiliser la souris pour choisir une
commande dans la barre de menu. Une flèche sur le bord droit d'un
élément de menu signifie qu'il conduit à un menu de filiale ou
sous-menu. La présence de <<\dots{}>> à la fin d'un élément de menu
indique que la commande nécessite une (ou plusieurs) saisie(s)
supplémentaires avant d'opérer.\par 

Certaines de ces commandes dans la barre de menu ont des raccourcis
clavier ordinaires ainsi; si c'est le cas, une clé de liaison est
indiquée entre parenthèses après l'item lui-même. Pour afficher le nom
de la commande et la documentation de l'item, tapez \texttt{C-h k},
puis sélectionnez la barre de menu avec la souris de la façon
habituelle (voir la section\cfchs{7}{1} [Aide pour les touches],
page\cfchsg{7}{1}). \par 

Au lieu d'utiliser la souris, vous pouvez également appeler le premier
élément de la barre de menu en appuyant sur F10 (pour exécuter la
commande \texttt{menu-bar-open}). Vous pouvez ensuite naviguer dans
les menus avec les touches fléchées. Pour activer un élément de menu
sélectionné, appuyez sur RET; navigation pour annuler le menu, appuyez
sur ESC. \par

Sur un terminal de texte, vous pouvez utiliser la barre de menu en
tapant \texttt{M-'} ou F10 (ces touches lancent la commande
\texttt{tmm-menubar}). Cela vous permet de sélectionner un élément de
menu avec le clavier. Un choix provisoire apparaît dans la zone
d'écho. Vous pouvez utiliser les touches flèches vers le haut ou vers
le bas pour vous déplacer dans le menu, et alors vous pouvez taper RET
pour sélectionner l'élément. Chaque élément de menu est également
désigné par une lettre ou un chiffre (généralement la première d'un
mot dans le nom de l'objet). Cette lettre ou ce chiffre est séparé du
nom de l'élément par <<\texttt{==>}>>. Vous pouvez taper la lettre ou
le chiffre de l'élément afin de le sélectionner.\par 