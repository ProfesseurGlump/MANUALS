\section{Point}\etchs{1}{1}
Le curseur de la fenêtre sélectionnée indique l'emplacement où la
plupart des commandes d'édition prennent effet, qui est appelé
point\footnote{Le terme <<point>> vient du caractère <<.>>, qui était la
  commande TECO (le langage dans lequel l'Emacs original a été
  écrit) pour accéder à la position d'édition.}. Beaucoup
de commandes Emacs déplacent le point à différents endroits dans le
tampon; par exemple, vous pouvez placer en cliquant sur le bouton 1 de
la souris (usuellement le gauche) à l'endroit désiré. Par défaut, le
curseur de la fenêtre sélectionnée est dessiné comme un bloc plein et
paraît être un caractère, mais vous devrez penser le point comme entre
deux caractères; il est situé avant le caractère sous le curseur. Par
exemple; si votre texte ressemble à <<frob>> avec le curseur sur <<b>>,
alors le point est entre le <<o>> et le <<b>>. Si vous insérez le
caractère <<!>> à cette position, le résultat sera <<fro!b>>, avec le
point entre le <<!>> et le <<b>>. Ainsi, le curseur reste au-dessus du
<<b>>, comme précédemment.\par 

Si vous éditez plusieurs fichiers dans Emacs, chacun dans son propre
tampon, chaque tampon a sa propre valeur du point. Un tampon qui n'est
pas actuellement affiché se souvient encore de sa valeur du point si vous
l'afficher plus tard. En outre, si un tampon est affiché dans
plusieurs fenêtres, chacune de ces fenêtres a sa propre valeur du
point.\par

Voir la section\cfchs{11}{20} [Affichage Curseur],
page\cfchsg{11}{20}, pour les options qui contrôlent la façon
dont Emacs affiche le curseur.\par 
