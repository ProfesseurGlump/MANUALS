\chapter{Caractères, touches et commandes}\etch{2}
Ce chapitre explique les jeux de caractères utilisés par Emacs pour
les commandes de saisie, et les concepts fondamentaux de touches et
commandes, avec lesquels Emacs interprète vos saisies claviers et
clics de souris.\par

\section{Type de saisies}\etchs{2}{1}
GNU Emacs est principalement conçu pour être utilisé avec le
clavier. Bien qu'il soit possible d'utiliser la souris pour lancer des
commandes d'édition grâce à la barre de menu et celle d'outils, ce
n'est pas aussi efficace qu'à l'aide du clavier. Par conséquent, ce
manuel documente principalement comment éditer avec le clavier.\par

Les saisies au clavier dans Emacs sont basées sur une version
fortement augmentée du code ASCII. Des caractères simples, comme
<<a>>, <<B>>, <<3>>, <<=>>, et le caractère espace (noté SPC), sont
saisies en tapant sur les touches correspondante. Les caractères de
contrôle, comme RET (entrée), TAB (tabulation), DEL, ESC (échap), F1,
Accueil, et à gauche, sont également saisis de cette façon, comme le
sont certains caractères figurant sur les claviers non-anglais (voir
le chapitre\cfch{19} [international], page\cfchg{19}).\par 

Emacs reconnaît également les caractères de contrôle qui sont saisis à
l'aide des touches modificatrices. Deux touches modificatrices
couramment utilisées sont la touche contrôle (généralement étiquetée
Ctrl), et la touche Meta (généralement étiquetée Alt)\footnote{Pour
  des raisons historiques nous nous référons à Alt comme à Meta}. Par
exemple, Ctrl + A est saisi en maintenant la touche Ctrl enfoncée tout
en appuyant sur la touche A; nous allons nous référer à ça comme
\texttt{C-a} pour faire court. De même \texttt{Meta-a}, ou
\texttt{M-a} pour faire court, est entré en maintenant enfoncée la
touche Alt puis en pressant la touche A. Les touches modificatrices
peuvent églament être appliquées à des caractères non alphanumériques,
par exemple, \texttt{C-F1} ou \texttt{M-gauche}.\par 

Vous pouvez aussi taper des caractères Meta en utilisant des séquences
de deux caractères commençant par ESC. Ainsi vous pouvez saisir
\texttt{M-a} en tapant ESC a. Contrairement à Meta, \textsc{esc} est
saisie comme un caractère distinct. Vous ne maintenez pas la touche
\textsc{esc} tout en tapant le caractère suivant; à la place, vous
pressez \textsc{esc} et la relâchez, puis saisissez le caractère
suivant. Cette fonction est utile sur certains terminaux de texte où
la touche Meta ne fonction pas de manière fiable.\par 

En mode graphique, le gestionnaire de fenêtre peut bloquer certaines
saisies au clavier, y compris \texttt{M-TAB, M-SPC, C-M-d} et
\texttt{C-M-l}. Si vous avez ce problème, vous pouvez personnaliser
votre gestionnaire de fenêtre pour ne pas bloquer les touches, ou
<<relier>> les commandes Emacs touchées (voir chapitre\cfch{33}
[Personnalisation], page\cfchg{33}).\par

Des caractères simples et des caractères de contrôle, ainsi que
certaines saisies non-clavier tels que les clics de souris, sont
collectivement appelés événements de saisies. Pour plus de détails sur
la façon dont Emacs gère en internes les événements de saisies, voir
la section <<\'Evénements de saisies>> dans le manuel de référence
Emacs Lisp.\par

\section{Touches (du clavier)}\etchs{2}{2}
Certaines commandes Emacs sont invoquées par un seul événement de
saisie; par exemple \texttt{C-f} avance d'un caractère dans le
tampon. D'autres commandes ont deux ou plusieurs événements de saisies
à invoquer, comme \texttt{C-x C-f} et \texttt{C-x 4 C-f}.\par 

Une séquence de touches, ou touche pour faire court, est une séquence
d'un ou plusieurs événements de saisies qui est significatif comme une
unité. Si une séquence de touches appelle une commande, nous appelons
cela une touche complète; par exemple, \texttt{C-f, C-x C-f} et
\texttt{C-x 4 C-f} sont des touches complètes. Si une séquence de
touches n'est pas assez longue pour invoquer une commande, nous
appelons cela une touche préfixée; de l'exemple précédent nous voyons
que \texttt{C-x} est \texttt{C-x 4} sont des touches préfixées. Chaque
séquence de touches est soit une touche complète soit une touche
préfixée.\par 

Une touche préfixée se combine avec l'événement de saisie suivant pour
faire une séquence de touches plus longue. Par exemple, \texttt{C-x}
est une touche préfixée, donc \texttt{C-x} seule n'invoque pas une
commande; à la place, Emacs attend une autre entrée (si vous vous
arrêtez prendant plus d'une seconde, il fait écho à la touche
\texttt{C-x} pour vous inviter à poursuivre la saisie; voir la
section\cfch{1sec2} [zone d'écho],
page\cfchg{1sec2}). \texttt{C-x} se combine avec le prochain événement de
saisie pour faire une touche à deux événements ce qui pourrait être
une touche préfixée (comme \texttt{C-x 4}), ou une touche complète
(comme \texttt{C-x C-f}). Il n'y a pas de limite à la longueur des
séquences de touches, mais en pratique, ils sont rarement plus de
trois ou quatre événéments de saisies.\par 

Vous ne pouvez pas ajouter des événements de saisie sur une touche
complète. Par exemple, parce que \texttt{C-f} est une touche complète, la
séquence à deux événements \texttt{C-f C-k} est une séquence à deux
touches, pas à une seule.\par

Par défaut, les touches préfixées dans Emacs sont \texttt{C-c, C-h,
  C-x, C-x RET, C-x @, C-x a, C-x, n, C-x r, C-x v, C-x 4, C-x 5, C-x
  6, ESC, M-g} et \texttt{M-o}. (F1 et F2 sont des alias pour
\texttt{C-h} et \texttt{C-x 6}.) Cette liste n'est pas coulée dans le
béton; si vous personnalisez Emacs, vous pouvez faire de nouvelles
touches préfixées. Vous pouvez même en éliminer, mais ce n'est pas
recommandé pour la plupart des utilisateurs; par exemple, si vous
supprimez la définition du préfixe \texttt{C-x 4}, alors \texttt{C-x 4
  C-f} deviendrait une séquence de touche non valide. Voir la
section\cfchs{33}{3} [combinaison de touches],
page\cfchsg{33}{3}.\par 

Taper le caractère d'aide (\texttt{C-h} ou F1) après une touche
préfixée affiche une liste de commandes commençant par ce préfixe. La
seule exception à cette règle est \texttt{\textsc{esc}:\textsc{esc} C-h} est
équivalent à \texttt{C-M-h}, qui fait tout autre chose. Vous pouvez,
cependant, utiliser F1 pour afficher une liste de commandes commençant
par \texttt{\textsc{esc}}.\par 

\section{Touches et commandes}\etchs{2}{3}
Ce manuel est plein de passages qui vous indiquent ce que font les touches
particulières. Mais Emacs n'assigne pas de sens directement aux
touches. Au lieu de cela, Emacs attribue des significations aux
commandes nommées, puis donne aux touches leurs significations en les
liant aux commandes.\par

Chaque commande a un nom choisi par un programmeur. Le nom est
généralement fait de quelques mots d'anglais séparés par des tirets;
par exemple, la prochaine ligne ou avant-mot. En interne, chaque
commande est un type spécial de fonction Lisp, et les actions
associées à la commande sont réalisées en exécutant la fonction. Voir
la section <<Qu'est-ce qu'une fonction>> dans le manuel de référence
Emacs Lisp.\par 

Les liens entre les touches et les commandes sont enregistrés dans des
tableaux appelés claviers. Voir la section\cfch{33sec3subsec1}
[claviers], page\cfchssg{33}{3}{1}.\par 

Quand nous disons que <<\texttt{C-n} déplace verticalement une ligne
vers le bas>> nous faisons l'impasse sur une distinction subtile qui
n'est pas pertinente dans l'usage ordinaire, mais vitale pour la
personnalisation d'Emacs. La commande ligne suivante fait un
déplacement vertical vers le bas. \texttt{C-n} a cet effet, car cette
touche est liée à la commande \texttt{prochaine-ligne}. Si vous changez le lien
en liant \texttt{C-n} à la commande \texttt{prochain-mot}, alors \texttt{C-n}
va se déplacer d'un mot au lieu d'une ligne.\par 

Dans ce manuel, nous allons souvent parler de touches comme
\texttt{C-n} comme d'une commande, même si à proprement parler cette
touche est liée à une commande. Habituellement nous indiquons le nom
de la commande qui fait vraiment le travail entre parenthèses après la
mention de la touche qui l'exécute. Par exemple, nous dirons que <<la
commande \texttt{C-n} (\texttt{next-line}) déplace le point
verticalement vers le bas>>, ce qui signifie que la commande
\texttt{next-line} se déplace verticalement vers le bas, et la touche
\texttt{C-n} est normalement liée à elle.\par 

Puisque nous parlons de personnalisation, nous devons vous en dire
plus sur les variables. Souvent la description d'une commande dira :
<<Pour changer cela, réglez la varibale \texttt{mumble-foo}.>> Une
variable est un nom utilisé pour stocker une valeur. La plupart des
variables documentées dans ce manuel sont destinées à la
personnalisation : certaines commandes ou d'autres parties d'Emacs
examinent la variable et se comportent différemment selon la valeur que
vous définissez. Vous pouvez ignorer les informations sur les
variables jusqu'à ce que vous soyez intéressé par la
personnalisation. Ensuite, lisez les informations de base sur les
variables (voir Section\cfchs{33}{2} [variables],
page\cfchsg{33}{2}) et les informations sur les variables
spécifiques prendront tout leur sens.\par
