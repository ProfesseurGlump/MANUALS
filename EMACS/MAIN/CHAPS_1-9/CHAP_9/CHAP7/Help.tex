\chapter{Aide}\etch{7}

Emacs fournit une grande variété de commandes d'aide, toutes
accessibles par la touche de préfixe \texttt{C-h} (ou, de façon
équivalente, la touche \texttt{F1}). Ces commandes d'aide sont
décrites dans les sections suivantes. Vous pouvez également taper
\texttt{C-h C-h} pour afficher une liste des commandes d'aide
(\texttt{help-for-help}). Vous pouvez faire défiler la liste avec la
\SPC et de \DEL, puis tapez la commande d'aide que vous voulez. Pour
annuler taper \rep{C}{g}.

Beaucoup de commandes d'aide affichent leurs informations dans un
tampon d'aide spécial. Dans ce tampon, vous pouvez taper \SPC et \DEL
pour faire défiler et tapez \RET pour suivre des liens
hypertexte. Voir la section\cfchs{7}{4} [Mode aide],
page\cfchsg{7}{4}.

Si vous cherchez une certaine fonction, mais ne savez pas comment elle
s'appelle ou où chercher, nous vous recommandons trois
méthodes. Premièrement, essayez une commande à propos, puis essayez de
chercher dans l'index du Manuel, puis regardez dans les \texttt{FAQ}
et les mots clés. 

\begin{tabular}[m]{>{\ttfamily}lp{9cm}}
  C-h a topics RET & \\
  & Cette fonction recherche des commandes dont les noms correspondent
  avec les arguments donnés. L'argument peut être un mot-clé, une
  liste de mots-clés, ou une expression régulière (voir
  Section\cfchs{12}{6} [Regexps], page\cfchsg{12}{6}. Voir la
  Section\cfchs{7}{3} [A propos], page\cfchsg{7}{3}. \\
  C-h i d m emacs RET i topic RET & \\
  & Cette fonction recherche les sujets dans les indices du manuel
  Emacs Info, l'affichage de la première correspondance
  trouvée. Appuyez, pour voir les correspondances suivantes. Vous
  pouvez utiliser une expression régulière comme sujet.\\
  C-h i d m emacs RET s topic RET & \\
  & Similaire, mais cherche le texte du manuel plutôt que les indices.\\
  C-h C-f & \\
  & Affiche les FAQ Emacs, utilisant Info.\\
  C-h p & \\
  & Affiche les paquetages Emacs disponibles basés sur le
  clavier. Voir Section\cfchs{7}{5} [Paquetages du clavier], page\cfchsg{7}{5}.
\end{tabular}

\rep{C}{h} ou \texttt{F1} signifie <<aide>> dans divers autres
contextes. Par exemple, vous pouvez les saisir après une touche
préfixée pour afficher une liste des touches qui peuvent suivre la
touche préfixée. (Quelques touches préfixées ne supportent pas
\rep{C}{h} de cette manière, car ils définissent d'autres
significations pour ça, mais ils soutiennent tous \texttt{F1} pour
l'aide.)

Voici un résumé des commandes d'aide pour accéder à la documentation
intégrée. La plupart d'entre elles sont décrites en détail dans les
sections suivantes. 

\begin{longtable}[m]{>{\ttfamily}lp{9cm}}
  C-h a topics RET & \\
  & Affiche une liste de commandes dont les noms correspondent aux
  sujets (\texttt{apropos-command})
  avec les arguments donnés. L'argument peut être un mot-clé, une
  liste de mots-clés, ou une expression régulière (voir
  Section\cfchs{12}{6} [Regexps], page\cfchsg{12}{6}. Voir la
  Section\cfchs{7}{3} [A propos], page\cfchsg{7}{3}. \\
  C-h b & \\
  & Affiche tous les raccourcis clavier actifs ; ceux des modes
  mineurs en premier, puis ceux des modes majeurs
  (\texttt{describe-bindings}). \\  
  C-h c key & \\
  & Affiche le nom de la commande associée au raccourci clavier
  (\texttt{describe-key-briefly}). Ici c est synonyme de
  <<caractère>>. Pour des informations plus complètes sur la touche,
  utiliser \rep{C}{h} \texttt{k}. \\
  C-h d topics RET & \\
  & Affiche les commandes et les variables dont la documentation
  correspond aux sujets (\texttt{apropos-documentation}).\\
  C-h e & \\
  & Affiche le tampon \texttt{*Messages*}
  (\texttt{view-echo-area-messages}). \\
  C-h f function RET & \\
  & La documentation de l'affichage de la fonction Lisp nommé
  (\texttt{describe-function}). Les commandes étant des fonctions
  Lisp, cela fonctionne pour les commandes aussi. \\
  C-h h & \\
  & Affiche le fichier \texttt{HELLO}, qui montre des exemples de
  différents jeux de caractères. \\
  C-h i & \\
  & Lance Info, le navigateur de documentation GNU (info). Le manuel
  Emacs est disponible dans Info. \\
  C-h k key & \\
  & Affiche le nom et la documentation de la commande que ce raccourci
  lance (\texttt{describe-key}). \\
  \endfirsthead
  C-h l & \\
  & Affiche une description des 300 dernières touches
  (\texttt{view-lossage}). \\
  C-h m & \\
  & Affiche la documentation du mode majeur courant
  (\texttt{describe-mode}). \\
  C-h n & \\
  & Affiche les changements récents dans Emacs
  (\texttt{view-emacs-news}). \\
  C-h p & \\
  & Trouve les paquetages par sujet de raccourci
  (\texttt{finder-by-keyword}). Cela liste les paquetages utilisant un
  tampon de menu de paquetage. Voir Chapitre\cfch{32}[Paquetages],
  page\cfchg{32}. \\
  C-h P package RET & \\
  & Affiche la documentation à propos du paquetage nommé paquetage
  (\texttt{describe-package}). \\
  C-h r & \\
  & Affiche le manuel Emacs dans Info (\texttt{info-emacs-manual}). \\
  C-h s & \\
  & Affiche le contenu de la table de syntaxe courante
  (\texttt{describe-syntax}). La table de syntaxe dit quels caractères
  sont les délimiteurs ouvrant, quels sont ceux qui sont des parties
  de mots, et ainsi de suite. Voir Section ``Tables de syntax'' dans
  le \textit{Emacs Lisp Reference Manual} pour plus détails. \\
  C-h t & \\
  & Entre dans le tutoriel interactif de Emacs
  (\texttt{help-with-tutorial}). \\
  C-h v var RET & \\
  & Affiche la documentation de la variable Lisp var
  (\texttt{describe-variable}). \\
  \endhead
  C-h w command RET & \\
  & Montre quelles touches exécutent les commandes nommées
  \textit{commande} ( \texttt{where-is}). \\
  C-h C coding RET & \\
  & Décrit le système d'encodage \textit{coding}
  (\texttt{describe-coding-system}). \\
  C-h C RET & \\
  & Décrit les systèmes d'encodage couramment utilisés. \\
  \endfoot
  C-h F command RET & \\
  & Lance Info et va au noeud correspondant à la commande Emacs
  \textit{commande} (\texttt{Info-goto-emacs-command-node}). \\
  C-h I method RET & \\
  & Décrit la méthode de saisie \textit{method}
  (\texttt{describe-input-method}). \\
  C-h K key & \\
  & Lance Info et va au noeud qui documente la séquence de touches
  \textit{key} (\texttt{Info-goto-emacs-key-command-node}). \\
  C-h L language-env RET & \\
  & Affiche des informations sur les ensembles de caractères, systèmes
  d'encodage, et méthode de saisie utilisées dans l'environnement de
  langage \textit{language-env}
  (\texttt{describe-language-environment}). \\
  C-h S symbol RET & \\
  & Affiche la documentation Info sur le symbole \textit{symbol} en
  accord avec le langage de programmation que vous éditez
  (\texttt{info-lookup-symbol}). \\
  \endlastfoot
  C-h & \\
  & Affiche le message d'aide pour une zone de texte spéciale, si le
  point est dans l'une d'elle (\texttt{display-local-help}). (Cela
  inclut, par exemple, les liens dans les tampons \texttt{*Help*}.)
\end{longtable}

\section{Documentation pour une clé (touche)}\etchs{7}{1}
Les commandes d'aide pour obtenir des informations sur une séquence de
touches sont \texttt{C-h c} (\texttt{describe-key-briefly}) et
\texttt{C-h k} (\texttt{describe-key}).

\texttt{C-h c \textit{key}} affiche dans la zone de répercussion du
nom de la commande qui est liée à la clé. Par exemple, \texttt{C-h c
  C-f} affiche '\texttt{forward-char}'. 
\texttt{C-h k \textit{key}} est similaire mais donne plus
d'information : il affiche un tampon d'aide contenant la chaîne de
documentation de la commande, qui décrit exactement ce que fait la
commande.
\texttt{C-h K \textit{key}} affiche la section du manuel Emacs qui
décrit la commande correspondant à la clé.
\texttt{C-h c}, \texttt{C-h k} et \texttt{C-h K} travaillent pour
toute sorte de séquences clés, y compris les touches de fonction, les
menus et le événements de souris. Par exemple, après \texttt{C-h k}
vous pouvez sélectionner un élément de menu dans la barre de menu,
pour afficher la chaîne de documentation de la commande qui s'exécute.

\texttt{C-h w \textit{command} RET} répertorie les touches qui sont
liées à \textit{command}. Il affiche la liste dans la zone d'écho. Si
il dit que la commande n'est sur aucune touche, ça signifie que vous
devez utiliser \rep{M}{x} pour la lancer. \texttt{C-h w}  lance la
commande \texttt{where-is}.
\section{Aide par commande ou par nom de variable}\etchs{7}{2}
\section{Apropos}\etchs{7}{3}
\section{Commandes du mode aide}\etchs{7}{4}
\section{Mot-clé pour la recherche d'extension}\etchs{7}{5}
\section{Aide pour les langues internationales}\etchs{7}{6}
\section{Autres commandes d'aide}\etchs{7}{7}
\section{Fichiers d'aide}\etchs{7}{7}
\section{Aide dans du texte actif et infobulles}\etchs{7}{8}
