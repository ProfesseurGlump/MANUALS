\chapter{Caractères internationaux et paramétrages}\etch{19}
Emacs supporte une grande variété de jeux de caractères
internationaux, y compris les variantes européennes et vietnamiennes
de l'alphabet latin, ainsi que le cyrillique, devanagari (pour l'hindi
et le marathi), éthiopien, grec, Han (pour le chinois et le japonais),
Hangul (pour le coréen, l'hébreu, l'IPA, le kannada, le Laos, le
malayalam, tamoul, thaï, tibétain, et vietnamiens. Emacs prend
également en charge divers encodages de ces caractères qui sont
utilisés par un autre logiciel internationalisé, tels que traitement
de texte et assistant de courriels.

Emacs permet l'édition de texte avec des caractères internationaux en
soutenant toutes les activités liées :
\begin{itemize}
\item Vous pouvez visiter les fichiers avec des caractères non-ASCII,
  enregistrer du texte non-ASCII, et passer du texte non-ASCII entre
  Emacs et des programmes qui l'invoquent (tels que des compilateurs,
  des correcteurs orthographiques, et des assistants de courriels). La
  configuration de votre environnement de langue (vois Section
  \cfchs{19}{3} [Environnements de langue], page\cfchsg{19}{3}) prend
  soin de mettre en place les systèmes d'encodage et d'autres options
  pour une langue ou une culture. Alternativement, vous pouvez
  spécifier comment Emacs devrait encoder ou décoder un texte pour
  chaque commande; voir Section \cfchs{19}{10} [Encodage de texte],
  page \cfchsg{19}{10}.
\item Vous pouvez afficher les caractères non-ASCII encodés par les
  différents scripts. Cela fonctionne en utilisant des polices
  appropriées sur les écrans graphiques (voir Section \cfchs{19}{15}
  [Définition des jeux de polices], page \cfchsg{19}{10}), et en
  envoyant des codes spéciaux pour les écrans de texte (voir Section
  \cfchs{19}{13} [Encodage des terminaux], page \cfchsg{19}{13}). Si
  certains caractères ne s'affichent pas correctement, reportez-vous à
  la Section \cfchs{19}{17} [Caractères non-affichables], page
  \cfchsg{19}{17}, qui décrit les problèmes possibles et explique
  comment les résoudre.
\item Les caractères des scripts qui sont naturellement de droite à
  gauche sont réorganisés pour l'affichage (voir Section
  \cfchs{19}{20} [Edition bidirectionnelle], page
  \cfchsg{19}{20}). Ces scripts comprennent l'arabe, l'hébreu, le
  syriaque, le Thaana, et quelques autres.
\item Vous pouvez insérer des caractères non-ASCII ou en
  chercher. Pour ce faire, vous permettant de spécifier une méthode
  d'entrée (voir Section \cfchs{19}{5} [Sélectionnez la méthode
  d'entrée], page \cfchsg{19}{5}) adapté à votre langue, ou utilisez
  la méthode d'entrée par défaut mise en place lorsque vous avez
  choisi votre environnement de langue. Si votre clavier peut produire
  des caractères non-ASCII, vous pouvez sélectionner un système
  d'encodage du clavier approprié (voir Section \cfchs{19}{13}
  [Encodage des terminaux], page \cfchsg{19}{13}), et Emacs acceptera
  ces caractères. Les caractères Latin-1 peuvent également être entré
  en utilisant le préfixe \rep{C}{x}, voir la Section \cfchs{19}{18}
  [Mode uni-octet], page \cfchsg{19}{18}.

Avec le système X Window, vos paramètres régionaux doivent être réglé
sur une valeur appropriée pour s'assurer qu'Emacs interprète
correctement les entrées au clavier ; voir Section \cfchs{19}{3}
[Environnements de langue], page \cfchsg{19}{3}.
\end{itemize}

Le reste de ce chapitre décrit ces questions en détail.

\section{Introduction des ensembles de caractères
  internationaux}\etchs{19}{1} 

Les utilisateurs des jeux de caractères internationaux ont établi
beaucoup de standard de systèmes d'encodage pour stocker les
fichiers. Ces systèmes d'encodage sont généralement multi-octets, ce
qui signifie que les suites de deux ou plusieurs octets sont utilisées
pour représenter les caractères individuels non-ASCII.

En interne, Emacs utilise son propre encodage de caractères
multi-octets, qui est un sur-ensemble de la norme
\textit{Unicode}. Cet encodage interne permet des caractères de
presque chaque script connu d'être mélangés dans un seul tampon ou
dans une chaîne de caractères. Emacs traduit entre l'encodage
multi-octets et divers autres de systèmes d'encodage lors de la
lecture et de l'écriture de fichiers, et quand il échange des données
avec les sous-processus.

La commande \texttt{C-h h (view-hello-file)} affiche le fichier
\texttt{etc/HELLO}, qui illustre divers scripts montrant comment dire
<<bonjour>> dans de nombreuses langues. Si certains caractères ne
peuvent être affihcés sur votre terminal, ils apparaissent comme '?'
ou comme caissons creux (voir Section \cfchs{19}{17} [Caractères non
affichables], page \cfchsg{19}{17}). 

Les claviers, même dans les pays où ces jeux de caractères sont
utilisés, n'ont généralement pas les touches pour tous les
caractères. Vous pouvez insérer des caractères que votre clavier ne
prend pas en charge, en utilisant \texttt{C-q (quoted-insert)} ou
\texttt{C-x 8 RET (insert-char)}. Voir Section \cfchs{4}{1} [Insertion
de texte], page \cfchsg{4}{1}. Emacs prend également en charge diverse
méthodes de saisie, généralement une pour chaque script ou langue, ce
qui rend plus facile de taper des caractères dans le script. Voir la
Section \cfchs{19}{4} [Les méthodes de saisie], page \cfchsg{19}{4}. 

La touche préfixée \rep{C}{x} \RET est utilisée pour les commandes qui
se rapportent aux caractères multi-octets, systèmes d'encodage, et les
méthodes d'entrée. 

La commande \texttt{C-x = (what-cursor-position)} renseigne sur le
caractère au point. En plus de la position du caractère, qui a été
décrit dans la Section \cfchs{4}{9} [Infos sur la position], page
\cfchsg{4}{9}, cette commande affiche comment le caractère est
codé. Par exemple, il affiche la ligne suivante dans la zone d'écho
pour le caractère <<c>> : 
\begin{center}
  \texttt{Char: c (99, \#o143, \#x63) point=28062 of 36168 (78\%) column=53}
\end{center}
Les quatre valeurs après <<\texttt{Char:}>> décrivent le caractère qui
suit le point, d'abord en montrant, puis en donnant son code de
caractère en décimal, octal et hexadécimal. Pour un caractère
multi-octets non ASCII, ceux-ci sont suivis par <<\texttt{file}>> et
la représentation du caractère, en hexadécimal, dans le système
d'encodage du tampon, si ce système d'encodage code le caractère en
toute sécurité avec un seul octet (voir Section \cfchs{19}{6} [Système
d'encodage], à la page \cfchsg{19}{6}. Si l'encodage du caractère est
plus d'un octet, Emacs montre <<\texttt{file \dots}>>

Comme cas particulier, si le caractère se trouve dans la plage de 128
(0200 en octal) à 159 (0237 en octal), cela signifie un octet <<brut>>
qui ne correspond pas à n'importe quel caractère spécial
affichable. Un tel <<caractère>> se trouve dans le jeu de caractère
\texttt{eight-bit-control} et est affiché comme un code de caractère
octal échappé. Dans ce cas, \texttt{C-x =} montre <<\texttt{part of
  display \dots}>> au lieu de <<\texttt{file}>>.

Avec un argument préfixé (\rec{C}{u}{C}{x =}), cette commande affiche
une description détaillée du caractère dans une fenêtre : 
\begin{itemize}
\item Le nom du jeu de caractères et les codes qui identifient le
  caractère dans ce jeu de caractères ; les caractères ASCII sont
  identifiés comme appartenant à l'ensemble de caractères \texttt{ascii}.
\item La syntaxe et les catégories du caractère.
\item Les encodages du caractère, tant à l'interne dans le tampon,
  qu'à l'extérieur si vous deviez enregistrer le fichier.
\item Quelles touches taper pour entrer le caractère de la méthode
  d'entrée naturelle (si elle prend en charge le caractère).
\item Si vous utilisez Emacs avec un affichage graphique, le nom de
  police et le code de glyphe pour le caractère. Si vous lancez Emacs
  dans un terminal de texte, le(s) code(s) est(sont) envoyé(s) au
  terminal.
\item Les propriétés du caractère de texte (voir la Section <<Text
  Properties>> in the Emacs Lisp Reference Manual), y compris les
  faces non-défaut utiisées pour afficher le caractère, et les
  incrustations en contenant (voir la section <<overlays>> dans le
  même manuel). 
\end{itemize}

Voici un exemple montrant le caractère Latin-1 A avec accent grave,
dans un tampon dont le système d'encodage est \texttt{utf-8-unix:}
\begin{center}
  \begin{tabular}[m]{>{\ttfamily}r >{\ttfamily}p{10cm}}
    position: & 1 of 1 (0\%), column: 0 \\
    character:& \`A (displayed as \`A) (codepoint 192, \#o300, \#xc0)
    \\
    prefered charset: & unicode (Unicode ISO10646)) \\
    code point in charset: & 0xC0 \\
    syntax: & w \quad wihich means: word \\
    category: & .:Base, L: Left-to-right (strong),\\
              & j:Japanese, l:Latin, v:Viet \\
    buffer code: & \#xC3 \#x80 \\
    file code: & not encodable by coding system undecided-unix \\
    display: & by this font (glyph code)
  \end{tabular}
  \begin{tabular}[m]{>{\ttfamily}c}
    xft:-unknown-DejaVu Sans Mono-normal-normal-\\
    normal-*-13-*-*-*-m-0-iso10646-1 (\#x82)\\
    \\
    Character code properties: customize what to show \\
    name: LATIN CAPITAL LETTER A WITH GRAVE \\
    old-name: LATIN CAPITAL LETTER A GRAVE \\
    general-category: Lu (Letter, Uppercase) \\
    decomposition: (65 768) ('A' '`')
  \end{tabular}
\end{center}
\section{Mise hors service des caractères multi-octets}\etchs{19}{2}
\section{Environnements de langues}\etchs{19}{3}
\section{Méthodes de saisie}\etchs{19}{4}
\section{Sélection d'une méthode de saisie}\etchs{19}{5}
\section{Systèmes de codage}\etchs{19}{6}
\section{Reconnaissance des systèmes de codage}\etchs{19}{7}
\section{Spécification du fichier du système de codage}\etchs{19}{8}
\section{Choix des systèmes de codage des sorties}\etchs{19}{9}
\section{Spécification du système de codage pour un fichier
  texte}\etchs{19}{10} 
\section{Systèmes de codage pour la communication
  inter-processus}\etchs{19}{11} 
\section{Systèmes de codage des noms de fichiers}\etchs{19}{12}
\section{Systèmes de codage pour les terminaux E/S}\etchs{19}{13}
\section{Ensembles de fontes}\etchs{19}{14}
\section{Définition d'ensemble de fontes}\etchs{19}{15}
\section{Modification d'ensemble de fontes}\etchs{19}{16}
\section{Caractères inaffichable}\etchs{19}{17}
\section{Ensemble de caractères}\etchs{19}{18}
\section{\'Edition bidirectionnelle}\etchs{19}{19}
