\chapter{Le minibuffer ou minitampon}\etch{5}

Le \emph{mini-tampon} est l'endroit où les commandes Emacs lisent les
arguments compliqués, comme des noms de fichiers, noms de tampons,
noms de commandes Emacs, ou expressions Lisp. On l'appelle le
<<mini-tampon>> parce que c'est un tampon à usage spécial avec une
petite taille d'espace d'écran. Vous pouvez utiliser les commandes
d'édition Emacs usuelles dans le mini-tampon pour éditer le texte de
l'argument.\par 

\section{Utilisation du minibuffer}\etchs{5}{1}
Quand le mini-tampon est utilisé, il apparaît dans la zone d'écho,
avec un curseur. Le mini-tampon commence avec une \emph{invite}
(prompt), terminant habituellement avec deux points. L'invite indique
le type de saisie attendue, et comment elle sera utilisée. L'invite
est surlignée et utilise la face \texttt{minibuffer-prompt} (voir
section\cfchs{11}{8} [Faces], page\cfchsg{11}{8}).\par

La façon la plus simple de saisir un argument dans le mini-tampon est
de taper le texte, puis \RET pour soumettre l'argument et
sortir du mini-tampon. Alternativement, vous pouvez taper \rep{C}{g}
pour sortir du mini-tampon en annulant la commande demandant
l'argument (voir section\cfchs{34}{1} [Quitter],
page\cfchsg{34}{1}).\par

Parfois, l'invite montre un \emph{argument par défaut}, à l'intérieur
de parenthèses avant les deux points. Cet argument par défaut sera
utilisé comme l'argument si vous tapez juste \RET. Par
exemple, des commandes qui lisent les noms de tampons montrent
usuellement un nom de tampon par défaut; vous pouvez taper
\RET pour travailler dans ce tampon par défaut.\par

Si vous activez Minibuffer Electric Default mode, un mode mineur
complet, Emacs cache l'argument par défaut dès que vous modifiez les
contenus du mini-tampon (alors taper \RET ne soumettra plus
l'argument par défaut). Si jamais vous ramenez le texte original du
mini-tampon, l'invite montrera à nouveau l'argument par défaut. En
outre, si vous changez la variable
\texttt{minibuffer-eldef-shorten-default} avec une valeur
non-\texttt{nil}, l'argument par défaut est affiché comme
<<\texttt{[default]}>> au lieu de <<\texttt{(default
  \emph{default})}>>, sauvegardant un peu d'espace d'écran. Pour
  activer ce mode mineur, tapez \texttt{M-x
    minibuffer-electric-default-mode}.\par 

Dès que le mini-tampon apparaît dans la zone d'écho, il peut y avoir
conflit avec d'autres usages de la zone d'écho. Si un message d'erreur
ou un message d'information est émis pendant que le buffer est actif,
le message cache le mini-tampon pour quelques secondes, ou jusqu'à ce
que vous tapiez quelque chose; alors le mini-tampon revient. Pendant
que le mini-tampon est en usage, les raccourcis ne font pas d'écho.\par  

\section{Minibuffer pour les noms de fichiers}\etchs{5}{2}

Des commandes comme \rec{C}{x}{C}{f} (\texttt{find-file}) utilisent le
mini-tampon pour lire un argument de nom de fichier (voir
section\cfchs{4}{5} [Fichiers de base], page\cfchsg{4}{5}). Quand le
mini-tampon est utilisé pour lire un nom de fichier, \c{c}a commence
typiquement avec du texte initial se terminant par une barre
oblique. C'est le \emph{répertoire par défaut}. Par exemple, \c{c}a
peut commençait par quelque chose comme \c{c}a :
\begin{center}
  \texttt{Find file: /u2/emacs/src/}
\end{center}
Ici, <<\texttt{Find file:}>> est l'invite et
<<\texttt{/u2/emacs/src/}>> est le répertoire par défaut. Si vous
tapez maintenant \texttt{buffer.c} comme saisie, cela spécifie le
fichier \texttt{/u2/emacs/src/buffer.c}. Voir section\cfchs{15}{1}
[Noms de fichiers], page\cfchsg{15}{1}, pour information à propos du
répertoire par défaut.\par

Vous pouvez spécifier le répertoire parent avec \texttt{..:
  /a/b/../foo.el} qui est équivalent à
\texttt{/a/foo/el}. Alternativement, vous pouvez utiliser
\rep{M}{DEL} pour supprimer le nom du répertoire (voir
section\cfchs{22}{1} [Mots], page\cfchsg{22}{1}).\par

Pour spécifier un fichier dans un répertoire complètement différent,
vous pouvez supprimer entièrement le répertoir par défaut avec
\rec{C}{a}{C}{k} (voir section\cfchs{5}{3} [\'Edition du mini-tampon],
page\cfchsg{5}{3}). Alternativement, vous pouvez ignorer le repertoire
par défaut, et saisir un chemin absolu en commençant par une barre
oblique ou un tilde après le répertoire par défaut. Par exemple, si
vous spécifiez \texttt{/etc/termcap} comme suit :
\begin{center}
  \texttt{Find file: /u2/emacs/src//etc/termcap}
\end{center}
Emacs interprète une double barre oblique comme <<ignore tout ce qu'il
y a avant la seconde barre oblique dans la paire>>. Dans l'exemple
ci-dessus, \texttt{/u2/emacs/src/} est ignoré, donc l'argument que
vous soumettez est \texttt{/etc/termcap}. La partie ignorée du chemin
ignoré est grisée si le terminal le permet. (Pour désactiver ce
grisement, désactiver le mode File Name Shadow avec la commande
\texttt{M-x file-name-shadow-mode}.)\par

Emacs interprète \texttt{\textasciitilde/} comme votre répertoir
personnel. Donc, \texttt{\textasciitilde/foo/bar.txt} spécifie un nom
de fichier \texttt{bar.txt}, dans un répertoire nommé \texttt{foo},
qui est localisé dans votre répertoire personnel. De plus,
\texttt{\textasciitilde user-id/} signifie : le répertoire personnel
d'un utilisateur dont l'identifiant est \emph{user-id}. N'importe quel
nom de répertoire avant le \texttt{\textasciitilde} est ignoré: donc,
\texttt{/u2/emacs/\textasciitilde/foo/bar.txt} est équivalent à
\texttt{\textasciitilde/foo/bar.txt}.\par 

Sur des systèmes MS-Windows et MS-DOS, où l'utilisateur n'a pas
toujours de repertoire personnel, Emacs utilise plusieurs
alternatives. Pour MS-Windows, voir section\cfaps{G}{5} [Windows
HOME], page\cfapsg{G}{5}; pour MS-DOS, voir section <<MS-DOS File
Names>> dans \emph{la version numérique du manuel Emacs}. Sur ces
systèmes, le \texttt{\textasciitilde user-id/} construit est supporté
uniquement pour l'utilisateur en court i.e, seulement si
\emph{user-id} est l'identifiant de l'utilisateur en court.\par

Pour prévenir Emacs de l'insertion du répertoire par défaut lors de la
lecture de noms de fichiers, modifier la variable
\texttt{insert-default-directory} à \texttt{nil}. Dans ce cas, le
mini-tampon démarre vide. Néanmoins, les arguments de noms de fichiers
sont encore interprétés basés sur le même répertoire par défaut.\par

Vous pouvez aussi saisir des noms de fichiers distants dans le
mini-tampon. Voir section\cfchs{15}{13} [Fichiers distants],
page\cfchsg{15}{13}.\par 

\section{\'Edition dans le minibuffer}\etchs{5}{3}

Le mini-tampon est un tampon Emacs, quoique particulier, et les
commandes Emacs usuelles sont disponibles pour l'édition du texte de
l'argument. (L'invite, toutefois, est \emph{en lecture seule}, et ne
peut pas être modifiée.)\par 

Dès que \RET dans le mini-tampon soumet l'argument, vous ne
pouvez pas l'utiliser pour insérer une nouvelle ligne. Vous pouvez
faire ça avec \rec{C}{q}{C}{j}, qui insère un caractère de contrôle
\rep{C}{j}, qui est formellement équivalent à un caractère de
nouvelle ligne (voir section\cfchs{4}{1} [Insertion de texte],
page\cfchs{4}{1}). Alternativement, vous pouvez utiliser la commande
\texttt{C-o (open-line)} (voir section\cfchs{4}{7} [Lignes vierges],
page\cfchs{4}{7}). \par

Dans le mini-tampon, les touches \TAB,\SPC et \texttt{?} sont
souvent liées à \emph{des commmandes de complétion}, qui vous
permettent de facilement remplir le texte désiré sans le taper en
entier. Voir section\cfchs{5}{4} [Complétion], page\cfchs{5}{4}. Comme
avec \RET, vous pouvez utiliser \texttt{C-q} pour insérer un caractère
\TAB,\SPC ou \texttt{?}.\par

Par commodité, \texttt{C-a (moving-beginning-of-line)} dans le
mini-tampon déplace le point au début de l'argument de texte, mais pas
au début de l'invite. Par exemple, cela vous permet d'effacer
l'argument entier avec \rec{C}{a}{C}{k}.\par

Quand le mini-tampon est actif, la zone d'écho est traitée plus comme
une fenêtre Emacs ordinaire. Par exemple, vous pouvez changer pour une
autre fenêtre (avec \texttt{C-x o}), éditer du texte là, ensuite
revenir à la fenêtre du mini-tampon pour finir l'argument. Vous pouvez
même supprimer du texte dans une autre fenêtre, revenir dans la
fenêtre du mini-tampon, et coller le texte dans l'argument. Il y a
quelques restrictions dans la fenêtre du mini-tampon, toutefois: par
exemple, vous ne pouvez pas la découper. Voir chapitre\cfch{17}
[Fenêtres], page\cfchg{17}.\par

Normalement, la fenêtre de mini-tapon occupe une seule ligne
d'écran. Toutefois, si vous ajouter deux ou plusieurs lignes de texte
dans le mini-tampon, sa taille augmente automatiquement de façon à
s'adapter au texte. La variable \texttt{resize-mini-windows} contrôle
la redimension du mini-tampon. La valeur par défaut est
\texttt{grow-only}, qui signifie le comportement que nous venons juste
de décrire. Si la valeur est \texttt{t}, la fenêtre du mini-tampon
sera aussi automatiquement rétractable si vous effacez des lignes de
texte du mini-tampon, jusqu'au minimum d'une ligne d'écran. Si la
valeur est \texttt{nil}, la fenêtre du mini-tampon ne changera jamais
de taille automatiquement, mais vous pouvez utiliser la commande
usuelle de redimensionnement de la fenêtre (voir chapitre\cfch{17}
[Fenêtres], page\cfchg{17}). \par

La variable \texttt{max-mini-window-height} contrôle la hauteur
maximale de redimensionnement de la fenêtre du mini-tampon. Un nombre
à virgule spécifie une fraction de la hauteur du cadre; un entier
spécifie le nombre maximal de lignes; \texttt{nil} signifie : ne pas
redimensionner la fenêtre du mini-tampon automatiquement. La valeur
par défaut est $0.25$.\par 

La commande \texttt{C-M-v} dans le mini-tampon fait défiler le texte
d'aide de commandes qui affiche l'aide de texte de toute sorte
dans une autre fenêtre. Vous pouvez aussi faire défiler l'aide de
texte avec \rep{M}{prior} et \rep{M}{next} (ou, de façon
équivalente, \rep{M}{PageUp} et \rep{M}{PageDown}). C'est
particulièrement utile avec des listes longues de complétions
possible. Voir section\cfchs{17}{3} [Autre fenêtre],
page\cfchs{17}{3}.\par

Emacs normalement ne permet pas la plupart des commandes qui utilisent
le mini-tampon pendant que le mini-tampon est actif. Pour autoriser de
telles commandes dans le mini-tampon, initialiser la variable
\texttt{enable-recursive-minibuffers} à \texttt{t}.\par

Quand il n'est pas actif, le mini-tampon est en
\texttt{minibuffer-inactive-mode}, et en cliquant \rep{Mouse}{1}
cela montre le tampon \texttt{*Messages*}. Si vous utilisez un cadre
dédié pour le mini-tampon, Emacs reconnaît aussi certaines touches là,
par exemple \texttt{n} pour créer un nouveau cadre.\par


\section{Complétion (ou complémentation)}\etchs{5}{4}
Vous pouvez souvent utiliser la fonction appelée \emph{complétion} qui
aide à saisir les arguments. Cela signifie qu'après que vous tapiez
une partie de l'argument, Emacs peut remplir le reste, ou une partie,
basée sur ce qui a déjà été rempli.\par

Quand la complétion est disponible, certaines touches (usuellement
\TAB, \RET, et \SPC) sont réservées dans le mini-tampon
pour des commandes de complétions spéciales (voir
section\cfchss{5}{4}{2} [Commandes de complétion],
page\cfchss{5}{4}{2}). Ces commandes attendent de compléter le texte
dans le mini-tampon, basées sur un \emph{ensemble de complétion
  alternatives} fournies par la commande qui requiert un
argument. Vous pouvez généralement taper \texttt{?} pour voir une
liste de complétions alternatives.\par

Bien que la complétion est généralement faite dans le mini-tampon, la
fonction est parfois disponible dans les tampons ordinaires
aussi. Voir section\cfchs{23}{8} [Symbole de complétion],
page\cfchsg{23}{8}.\par

\subsection{Exemple de complétion}\etchs{5}{4subsec1}
Un exemple simple peut aider ici. \rep{M}{x} utilise le minitampon
pour lire le nom d'une commande, donc la complétion fonctionne par
correspondance du texte du minitampon avec les noms des commandes
Emacs existantes. Supposons que vous souhaitiez lancer la commande
\texttt{auto-fill-mode}. Vous pouvez le faire en tapant \texttt{M-x
  auto-fill-mode RET}, mais il est plus facile d'utiliser la
complétion.\par

Si vous tapez \texttt{M-x a u TAB}, la \TAB cherche les
complétions alternatives (dans ce cas, les noms de commandes) qui
démarrent par <<\texttt{au}>>. Il y en a plusieurs, incluant
\texttt{auto-fill-mode} et \texttt{autoconf-mode}, mais elles
commencent toutes avec \texttt{auto}, donc le <<\texttt{au}>> dans le
minitampon se complète en <<\texttt{auto}>>. (Plus de commandes
peuvent être définies dans votre session Emacs. Par exemple, si une
commande appelée \texttt{authorize-me} était définie, Emacs pourrait
seulement compléter jusqu'à <<\texttt{aut}>>.)\par

Si vous tapez \TAB encore immédiatement, il ne peut pas
déterminer le prochain caractère; cela pourrait être <<\texttt{-}>>,
<<\texttt{a}>>, ou <<\texttt{c}>>. Donc il n'ajoute aucun caractère;
au lieu, \TAB affiche une liste de toutes les complétions
possibles dans une autre fenêtre.\par

Ensuite, tapez \texttt{-f}. Le minitampon contient désormais
<<\texttt{auto-f}>>, et le seul nom de commande qui démarre avec ça
est <<\texttt{auto-fill-mode}>>. Si vous tapez maintenant
\TAB, la complétion rempli le reste de l'argument
<<\texttt{auto-fill-mode}>> dans le minitampon. \par

D'où, taper juste \texttt{a u TAB -f TAB} vous permet d'entrer
<<\texttt{auto-fill-mode}>>. \par

\subsection{Commandes de complétion}\etchs{5}{4subsec2}

Voici une liste des commandes de complétions définies dans le
minitampon la complétion est permise.

\begin{tabular}[m]{>{\ttfamily}lp{13cm}}
  TAB & Complète le texte dans le minitampon autant que possible; si
  pas possible de compléter, affiche une liste des complétions
  possibles (\texttt{minibuffer-complete}). \\
  SPC & Complète jusqu'à un mot du texte du minitampon avant le point
  (\texttt{minibuffer-complete-word}). Cette commande n'est pas
  disponible pour des arguments qui incluent souvent des espaces,
  comme des noms de fichiers. \\
  RET & Présente le texte dans le minitampon comme argument,
  complétant éventuellement le premier
  (\texttt{minibuffer-complete-and-exit}). Voir section\cfchss{5}{4}{3}
  [Sortie de complétion], page\cfchssg{5}{4}{3}.  \\
  ? & Affiche une liste des complétions
  (\texttt{minibuffer-completion-help}). 
\end{tabular}

\TAB(\texttt{minibuffer-complete}) est la commande de
complétion la plus fondamentale. Elle cherche toutes les complétions
possibles qui correspondent au texte existant dans le minitampon, et
tente de le compléter autant que possible. Voir
section\cfchss{5}{4}{4}[Styles de complétion], page\cfchssg{5}{4}{4},
pour savoir comment les complétions alternatives sont choisies.\par

\SPC (\texttt{minibuffer-complete-word}) complète comme
\TAB, mais seulement jusqu'au prochain tiret ou espace. Si
vous avez <<\texttt{auto-f}>> dans le minitampon et tapez
\SPC, cela trouvera la complétion <<\texttt{auto-fill-mode}>>,
mais cela insère seulement <<\texttt{ill-}>>, donnant
<<\texttt{auto-fill}>>. Un autre \SPC à cet endroit complète
le reste jusqu'à <<\texttt{auto-fill-mode}>>.\par

Si \TAB ou \SPC n'arrive pas à compléter, cela affiche
une liste des complétions alternatives correspondantes (s'il y en a)
dans une autre fenêtre. Vous pouvez afficher la même liste avec
\texttt{?} (\texttt{minibuffer-completion-help}). Les commandes
suivantes peuvent être utilisées avec la liste de complétion :

\begin{tabular}[m]{>{\ttfamily}lp{13cm}}
  Mouse-1 & \\
  Mouse-2 & Cliquer sur le bouton 1 ou 2 sur une complétion
  alternative la choisis (\texttt{mouse-choose-completion}). \\
  M-v & \\
  PageUp & \\
  prior & Taper \texttt{M-v}, dans le minitampon, sélectionne la
  fenêtre montrant la liste de complétion
  (\texttt{switch-to-completions}). Cela ouvre la voie à l'aide des
  commandes ci-dessous. \texttt{PageUp} ou \texttt{prior} fait la
  même. Vous pouvez aussi sélectionner la fenêtre d'autres manières
  (voir \cfch{17}[Fenêtres], page\cfchg{17}). \\
  RET & Dans le tampon de la liste de complétion, cela choisit la
  complétion au point (\texttt{choose-completion}). \\
  Right & Dans le tampon de la liste de complétion, cela déplace le
  point à la complétion alternative suivante
  (\texttt{next-completion}). \\
  Left & Dans le tampon de la liste de complétion, cela déplace le
  point à la complétion alternative précédente
  (\texttt{previous-completion}). 
\end{tabular}

\subsection{Sortie de complétion}\etchs{5}{4subsec3}

Quand une commande lit un argument utilisant le minitampon avec
complétion, cela contrôle aussi ce qui se passe lorsque vous tapez
\RET (\texttt{minibuffer-complete-and-exit}) pour présenter
l'argument. Il y a quatre types de comportements :
\begin{itemize}
\item \textit{Complétion stricte} accepte seulement les complétions
  correspondantes exactes. Taper \RET sort du minitampon
  seulement si le texte du minitampon est une correspondance exacte,
  ou la complète en une. Sinon, Emacs refuse de quitter le minitampon;
  au lieu de cela il essaie de compléter, et si aucune complétion ne
  peut être faite il affiche momentanément <<[\texttt{No match}]>>
  après le texte du minitampon. (Vous pouvez toujours quitter le
  minitampon en tapant \rep{C}{g} pour annuler la commande.)

  Un exemple d'une commande qui utilise ce comportement est
  \rep{M}{x}, car ça n'a pas de sens pour elle d'accepter un de
  commande inexistante.
\item \textit{La complétion prudente} est comme la complétion stricte,
  sauf que \RET sort seulement si le texte est déjà une
  correspondance exacte. Si le texte complète en une correspondance
  exacte, \RET accompli cette complétion mais ne sort pas
  encore; vous devez taper \RET une seconde fois pour sortir.

  La complétion prudente est utilisée pour lire les noms de fichiers
  de fichiers qui existent déjà, par exemple.
\item \textit{La complétion permissive} autorise n'importe qu'elle
  saisie; les candidats à la complétion sont justes des
  suggestions. Taper \RET ne complète pas, ça présente juste
  l'argument comme vous l'avez entré.
\item \textit{La complétion permissive avec confirmation} est comme la
  complétion permissive, à une exception : si vous tapez \TAB
  et que ça complète le texte jusqu'à un état intermédiaire (i.e, un
  qui n'est pas une complétion correspondante exacte), taper
  \RET juste après ne présente pas l'argument. Au lieu de ça,
  Emacs demande une confirmation en affichant momentanément
  <<[\texttt{Confirm}]>> après le texte; taper \RET encore
  pour confirmer et présenter le texte. Cela empêche une erreur
  classique, dans laquelle une frappe \RET avant de realiser
  que \TAB n'avait pas complété aussi loin que souhaité.

  Vous pouvez modifier le comportement de la confirmation par la
  personnalisation de la variable
  \texttt{confirm-nonexistent-file-or-buffer}. La valeur par défaut,
  \texttt{after-completion}, donne le comportement que nous venons de
  décrire. Si vous la changez à \texttt{nil}, Emacs ne demandera pas
  de confirmation, retombant en complétion permissive. Si vous la
  changez en n'importe qu'elle autre valeur non\texttt{-nil}, Emacs
  demandera une confirmation que la précédente comme soit \TAB
  ou pas.

  Ce comportement est utilisé par la plupart des commandes qui lisent
  des noms de fichiers, comme \rec{C}{x}{C}{f}, et les commandes qui
  lisent les noms de tampons, comme \texttt{C-x b}.
\end{itemize}

\subsection{Comment sont choisies les alternatives de
  complétion}\etchs{5}{4subsec4} 

Les commandes de complétion fonctionnent par rétrécissement d'une
grande liste de complétions alternatives possibles vers un sous
ensemble qui <<correspond>> avec ce que vous avez tapé dans le mini
tampon. Dans la section \cfchss{5}{4}{1} [Exemple de complétion],
page\cfchssg{5}{4}{1}, nous avons donné un exemple simple d'une telle
correspondance. Emacs tente d'offrir les complétions plausibles dans
la plupart des cas.

Emacs réalise les complétions en utilisant un ou plusieurs
\textit{styles de complétion}---ensembles de critères de
correspondance pour le texte du mini tampon pour les complétions
alternatives. Durant la complétion, Emacs essaie chaque style de
complétion tour à tour. Si un style donne une ou plusieurs
correspondances, qui est utilisée comme liste de complétions
alternatives. Si un style ne produit pas de correspondance, Emacs
retombe sur le style suivant.

Cette variable liste \texttt{completion-styles} spécifie les styles de
complétions à utiliser. Chaque élément de la liste est le nom d'un
style de complétion (un symbole Lisp). Les styles de complétion par
défaut sont (par ordre):
\begin{description}
\item[\texttt{basic}] une complétion alternative doit avoir le même début que
  le texte dans le mini-tampon avant le point. En outre, s'il y a un
  texte dans le mini-tampon après le point, le reste de la complétion
  alternative doit contenir ce texte comme sous chaîne.
\item[\texttt{partial-completion}] 

Ce style de complétion agressif divise le texte du mini-tampon en mots
séparés par des tirets ou des espaces, et termine chaque mot
séparément. (Par exemple, quand il complète les noms de commandes,
<<\texttt{em-l-m}>> complète en <<\texttt{emacs-lisp-mode}>>.)

En outre, une <<$\star$>> dans le texte du mini-tampon est traité
comme un \texttt{joker}---ça correspond à n'importe quel caractère à
la position associée dans la complétion alternative.
\item[\texttt{emacs22}] Ce style de complétion est similaire du
  \texttt{basic}, sauf qu'il ignore le texte dans le mini-tampon après
  le point. Il est appelé ainsi parce qu'il correspond au comportement
  de complétion dans Emacs 22.
\end{description}
Les styles de complétion supplémentaires suivants sont aussi définis,
et vous pouvez les ajouter au \texttt{completion-styles} si vous le
souhaitez (voir Chapitre \cfch{33} [Personnalisation], page
\cfchg{33}):

% petite macro maison
\newcommand{\citcom}[1]{<<\texttt{#1}>>}

\begin{description}
\item[\texttt{substring}] une correspondance de complétion alternative doit
  contenir le texte dans le mini-tampon avant le point, et le texte
  dans le mini-tampon après le point, comme chaînes (dans le même
  ordre).

Alors, si le texte dans le mini-tampon est \citcom{foobar}, avec
le point entre \citcom{foo} et \citcom{bar}, qui correspond à
\citcom{afoo\textit{b}barc}, où \textit{a},\textit{b}, et \textit{c}
peuvent être n'importe quelle chaîne incluant la chaîne vide.
\item[\texttt{initials}] Ce style de complétion très agressif tente de
  compléter les acronymes et sigles. Par exemple, quand il complète
  les noms de commandes, il fait correspondre \citcom{lch} avec
  \citcom{list-command-history}. 
\end{description}
Il y a aussi un style de complétion très simple appelé
\texttt{emacs21}. Dans ce style, si le texte dans le mini-tampon est
\citcom{foobar}, seules les correspondances commençant avec
\citcom{foobar} sont considérées.

Vous pouvez utiliser des styles de complétion différents dans
différentes situations, en réglant la variable
\texttt{completion-category-overrides}. Par exemple, le réglage par
défaut dit d'utiliser seulement la complétion \texttt{basic} et
\texttt{substring} pour les noms de tampons.

\subsection{Options de  complétion}\etchs{5}{4subsec5} 

Un cas est important au moment de remplir les arguments de la
casse, comme par exemple les noms de commandes, '\texttt{AU}' ne se
complète pas en '\texttt{auto-fill-mode}'. 
Différents cas sont ignorés au moment de remplir des arguments dans ce
cas ce n'est pas grave.

Lorsque vous remplissez les noms de fichiers, les différences de casse
sont ignorées si la variable
\texttt{read-file-name-completion-ignore-case} est
non-\texttt{nil}. La valeur par défaut est \texttt{nil} sur les
systèmes qui ont des noms de fichiers sensibles à la casse, comme
GNU/Linux ; il est non-\texttt{nil} sur les systèmes qui sont
insensibles à la casse, comme Microsoft Windows. Lorsque vous
remplissez les noms de tampons, les différences de casse sont ignorées
si la variable \texttt{read-buffer-completion-ignore-case} est
non-\texttt{nil}; par défaut elle vaut \texttt{nil}.

Lorsque vous remplissez les noms de fichiers, Emacs omet généralement
certaines alternatives qui sont considérées peu probable d'être
choisies, telle que déterminée par la variable liste
\texttt{completion-ignored-extensions}. Chaque élément de la liste
doit être une chaîne ; n'importe quel nom de fichier se terminant par
une telle chaîne est ignoré comme alternative d'achèvement. Tout
élément se terminant par une barre oblique (\slash ) représente un nom
de sous répertoire. La valeur standard de
\texttt{completion-ignored-extensions} comporte plusieurs éléments
dont <<\texttt{.o}>>, <<\texttt{.elc}>>, et
<<\texttt{\textasciitilde}>>. Par exemple, si un répertoire contient
'\texttt{foo.c}' et '\texttt{foo.elc}', '\texttt{foo}' se complète en
'\texttt{foo.c}'. Toutefois, si toutes les complétions possibles
finissent dans les chaînes <<ignorées>>, elles ne sont pas ignorées:
dans l'exemple précédent, '\texttt{foo.e}' se complète en
'\texttt{foo.elc}'. Emacs ignore
\texttt{completion-ignored-extensions} en montrant des alternatives de
complétions dans la liste des terminaisons possibles.

Si \texttt{completion-auto-help} est réglé à \texttt{nil}, les
commandes de complétion n'affiche jamais le tampon de la liste
d'achèvement; vous devez taper \texttt{?} pour afficher la liste. Si
la valeur est \texttt{lazy}, Emacs affiche uniquement le tampon de la
liste d'achèvement pour la deuxième tentative de complétion. En
d'autres termes, s'il n'y a rien à compléter, le premier écho de
\texttt{TAB} sera '\texttt{Next char not unique}'; le second
\texttt{TAB} affichera le tampon de la liste de complétion.

Si \texttt{completion-cycle-threshold} est non-\texttt{nil}, les
commandes de complétion peuvent <<boucler>> à travers les complétions
alternatives. Normalement, s'il y a plus d'une alternative
d'achèvement pour le texte dans le mini-tampon, une commande de
complétion complète jusqu'à la plus longue sous-chaîne commune. Si
vous changez \texttt{completion-cycle-threshold} pour \texttt{t}, la
commande de complétion au lieu de compléter la première de ces
complétions alternatives; chaque appel ultérieur de la commande
de complétion remplace celle de la prochaine complétion alternative,
de manière cyclique. Si vous donnez
 à \texttt{completion-cycle-threshold} une valeur numérique $n$, les
 commandes de complétion passeront au comportement cyclique seulement
 quand il aura $n$ ou moins alternatives. 

Le mode incomplet présente un affichage constamment mis à jour qui
vous dit quelles complétions sont disponibles pour le texte que vous
avez saisi jusque là. La commande pour activer ou désactiver ce mode
mineur est \texttt{M-x incomplete-mode}.

\section{Historique du minibuffer}\etchs{5}{5}

Tous les arguments que vous entrez avec le mini-tampon sont
enregistrés dans une liste d'historique du mini-tampon de sorte que
vous pouvez facilement les utiliser plus tard. Vous pouvez utiliser
les arguments suivants pour aller chercher rapidement un argument plus
tôt dans le mini-tampon :
\begin{center}
  \begin{tabular}[m]{>{\ttfamily}lp{.85\linewidth}}
    M-p & \\
    Up & Déplacez à l'élément précédent dans l'historique du
    mini-tampon, un argument plus tôt
    (\texttt{previous-history-element}). \\
    M-n & \\
    Down & Déplacez à l'élément suivant dans l'historique du
    mini-tampon (\texttt{next-history-element}). \\
    M-r \textit{regexp} RET & Déplacer vers un point plus tôt dans l'histoire
    du mini-tampon qui correspond à l'expression régulière \par
    (\texttt{previous-matching-history-element})\\
    M-s \textit{regexp} RET & Déplacer vers un point plus tard dans l'histoire
    du mini-tampon qui correspond à l'expression régulière \par
    (\texttt{next-matching-history-element})
  \end{tabular}
\end{center}

Alors que dans le mini-tampon, \texttt{M-p} ou \texttt{Up}
(\texttt{previous-history-element}) se déplace à travers la liste
d'historique du mini-tampon, un élément à la fois. Chaque \texttt{M-p}
récupère un élément plus tôt dans la liste de l'historique du
mini-tampon, remplaçant son contenu existant. Taper \texttt{M-n} ou
\texttt{Down} (\texttt{next-history-element}) se déplace à travers la
liste d'historique du mini-tampon dans la direction opposée, aller
chercher les entrées plus tard dans le mini-tampon.

Si vous tapez \texttt{M-n} dans le mini-tampon lorsqu'il n'y a pas
d'entrée plus tard dans l'historique du mini-tampon (par exemple, si
vous n'avez pas déjà tapé \texttt{M-p}), Emacs essaie d'aller chercher
dans une liste d'arguments par défaut : les valeurs que vous êtes
susceptibles d'entrer. Vous pouvez penser à cela comme un déplacement
à travers <<le futur historique>>.

Si vous modifiez le texte inséré par les commandes d'historique du
mini-tampon \texttt{M-p} ou \texttt{M-n}, cela ne change pas son
entrée dans la liste de l'historique. Toutefois, l'argument édité va à
la fin de l'historique lorsque vous le soumettez.

Vous pouvez utiliser \texttt{M-r}
(\texttt{previous-matching-history-element}) pour chercher à travers
les anciens éléments dans la liste de l'historique, et \texttt{M-s}
(\texttt{next-matching-history-element}) pour chercher à travers les
entrées plus récentes. Chacune de ces commandes demande une expression
régulière comme un argument, et récupère la première entrée
correspondante dans le mini-tampon. 

Voir section\cfchs{12}{6} [Regexps], page\cfchsg{12}{6}, pour une
explication des expressions régulières. Un argument numérique préfixé
par $n$ signifie chercher la $n$-ième entrée correspondante. Ces
commandes sont inhabituelles, en ce qu'elles utilisent le mini-tampon
pour lire l'argument de l'expression régulière, même si elles sont
invoquées à partir du mini-tampon. Un lettre majuscule dans
l'expression régulière rend la recherche sensible à la casse (voir
section\cfchs{12}{9} [Recherche sensible à la casse],
page\cfchsg{12}{9}.

Vous pouvez également rechercher à travers l'historique en utilisant
une recherche incrémentale. Voir
section\cfchss{12}{1}{7} [Recherche incrémentale dans le mini-tampon],
page\cfchssg{12}{1}{7}. 

Emacs conserve des listes d'historique séparées pour plusieurs types
d'arguments. Par exemple, il y a une liste des noms de fichiers,
utilisés par toutes les commandes qui lisent des noms de
fichiers. Les autres listes d'historiques comprennent les noms de
tampons, noms de commandes (utilisés par \texttt{M-x}), et les
arguments de commande (utilisés par des commandes comme
\texttt{query-replace}). 

La variable \texttt{history-length} spécifie la longueur maximale
d'une liste d'historique du mini-tampon; l'ajout d'un nouvel élément
supprime l'élément le plus ancien si la liste est trop longue. Si la
valeur est \texttt{t}, il n'y a pas de longueur maximale.

La variable \texttt{history-delete-duplicates} indique s'il faut
supprimer les doublons dans l'historique. S'il est non-\texttt{nil},
l'ajout d'un nouvel élément dans la liste supprime tous les autres
éléments qui lui sont égaux. La valeur par défaut est \texttt{nil}. 

\section{Répétition des commandes du minibuffer}\etchs{5}{6}

Chaque commande qui utilise le mini-tampon une fois est enregistrée
sur une liste d'historique particulière, l'historique des commandes,
ainsi que les valeurs de ses arguments, de sorte que vous pouvez
répéter la commande entière. En particulier, chaque utilisation de
\texttt{M-x} est enregistrée là, puisque \texttt{M-x} utilise le
mini-tampon pour lire le nom de la commande.

\begin{center}
  \begin{tabular}[m]{>{\ttfamily}lp{.85\linewidth}}
    C-x ESC ESC & Ré-exécute une commande de mini-tampon récente dans
    l'historique de commande (\texttt{repeat-complex-command}) \\
    M-x list-command-history & Afficher tout l'historique de commande,
    montrant toutes les commandes \texttt{C-x ESC ESC} peuvent
    se répéter, les plus récentes apparaissant en premier. 
  \end{tabular}
\end{center}

\texttt{C-x ESC ESC} ré-exécute une commande récente qui a utilisé le
    mini-tampon. Sans argument, elle répète telle quelle la dernière
    commande. Un argument numérique spécifie quelle commande répéter ;
    1 signifie le dernier, 2 le précédent, et ainsi de suite.

\texttt{C-x ESC ESC} fonctionne en tournant la commande précédente en
une expression Lisp puis en entrant un mini-tampon initialisé avec le
texte de cette expression. Même si vous ne connaissez pas Lisp, il
sera probablement évident de voir quelle commande s'affiche pour la
répétition. Si vous tapez juste \RET, qui répète la commande
inchangée. Vous pouvez également modifier la commande en modifiant
l'expression Lisp avant de l'exécuter. La commande exécutée est
ajoutée à l'avant de l'historique des commandes que si elle est
identique à l'élément le plus récent.

Une fois à l'intérieur du mini-tampon pour \texttt{C-x ESC ESC}, vous
pouvez utiliser les commandes habituelles d'historique du mini-tampon (voir
section\cfchs{5}{5} [Historique du mini-tampon], page\cfchsg{5}{5})
pour vous déplacer dans la liste de l'histoire. Après avoir trouvé la
commande précédente désirée, vous pouvez modifier son expression comme
d'habitude, puis l'exécuter en tapant \RET .

La recherche incrémentale, à proprement parler, n'utilise pas le
mini-tampon. Par conséquent, même si elle se comporte comme une
commande complexe, elle n'apparaît normalement pas dans la liste de
l'historique pour \texttt{C-x ESC ESC}. Vous pouvez faire apparaître des
commandes de recherche supplémentaires dans l'historique en
configurant \texttt{isearch-resume-in-command-history} à une valeur
non-\texttt{nil}. Voir section\cfchs{12}{1} [Recherche incrémentale],
page\cfchsg{12}{1}.

La liste des commandes précédentes utilisant le mini-tampon est
stockée comme une liste Lisp dans la variable
\texttt{command-history}. Chaque élément est une expression Lisp qui
décrit une commande et ses arguments. Les programmes Lisp peuvent
ré-exécuter une commande en appelant \texttt{eval} avec l'élément
\texttt{command-history}. 

\section{Création de mot de passe}\etchs{5}{7}

Parfois, vous devrez peut-être un mot de passe dans Emacs. Par
exemple, quand vous dites à Emacs de visiter un fichier sur une autre
machine via un protocole FTP, vous avez souvent besoin de fournir un
mot de passe pour accéder à la machine (voir section\cfchs{15}{13}
[Fichiers distants], page\cfchsg{15}{13}).

La saisie d'un mot de passe est similaire à l'aide d'un
mini-tampon. Emacs affiche une invite dans la zone écho (comme
'\texttt{Password}'); apès avoir tapé le mot de passe requis, appuyez
sur \RET pour le soumettre. Pour empêcher les autres de voir votre mot
de passe, chaque caractère que vous tapez est affiché comme un point
('\texttt{.}') au lieu de sa forme habituelle.

La plupart des fonctions et commandes associées avec le mini-tampon ne
peut être utilisé lors de la saisie d'un mot de passe. Il n'y a pas
d'historique de la complétion, et vous ne pouvez pas changer de
fenêtre ou effectuer toute autre action avec Emacs jusqu'à ce que vous
ayez soumis le mot de passe.

Pendant que vous tapez le mot de passe, vous pouvez appuyer sur \DEL
pour effacer en arrière, enlever le dernier caractère
saisi. \texttt{C-u} supprime tout ce que vous avez saisi
jusqu'ici. \texttt{C-g} quitte l'invite de mot de passe (voir
section\cfchs{34}{1} [Quitter], page\cfchsg{34}{1}). \texttt{C-y}
insère la suppression courante dans le mot de passe (voir
section\cfch{9} [Suppression], page\cfchg{9}). Vous pouvez taper soit
\RET soit \ESC pour soumettre le mot de passe. Toute autre touche de
caractère auto-insertion insère le caractère associé dans le mot de
passe, et toute autre entrée est ignorée. 

\section{Oui ou Non du prompt (invite de commande)}\etchs{5}{8}

Une commande Emacs peut vous demander de répondre à une question <<oui
ou non>> au cours de son exécution. Ces requêtes sont disponibles en
deux variétés principales. 

Pour le premier type de requête <<oui ou non>>, l'invite se termine
par '\texttt{(y or n)}'. Une telle requête n'utilise pas le
mini-tampon ; l'invite apparaît dans la zone écho, et que vous
répondez en tapant '\texttt{y}' ou '\texttt{n}', qui fournit
immédiatement la réponse. Par exemple, si vous tapez \texttt{C-x C-w}
(\texttt{(write-file}) pour enregistrer un tampon, et entrez le nom
d'un fichier existant, Emacs émet une invite comme ceci :
\begin{center}
  \begin{scriptsize}
    {\ttfamily File 'foo.el' exists; overwrite? (y or n)}
  \end{scriptsize}
\end{center}

Parce que cette requête n'utilise pas le mini-tampon, les commandes
d'édition usuelles du mini-tampon ne peuvent être
utilisées. Toutefois, vous pouvez effectuer certaines opérations de
défilement de fenêtre pendant que la requête est active : \texttt{C-l}
recentre la fenêtre sélectionnée ; \texttt{M-v} (ou \texttt{PageDown}
ou \texttt{next}) fait défiler vers l'avant ; \texttt{C-v} (ou
\texttt{PageUp} ou \texttt{prior}) défile en arrière ; \texttt{C-M-v}
défile en avant dans la prochaine fenêtre ; et \texttt{C-M-S-v} défile
en arrière dans la fenêtre suivante. Taper \texttt{C-g} annule la
requête, et quitte la commande qui l'a délivré (voir
section\cfchs{34}{1}[Quitter], page\cfchsg{34}{1}).

Le deuxième type de requête <<oui ou non>> est généralement utilisé si
donner la mauvaise réponse aurait de graves conséquences; il utilise
le mini-tampon, et dispose d'une fin rapide avec '\texttt{(yes or
  no)}'. Par exemple, si vous invoquez \texttt{C-x k}
(\texttt{kill-buffer}) sur un tampon de fichier visiter avec des
modifications non enregistrées, Emacs active le mini-tampon avec une
invite comme ceci :
\begin{center}
  \begin{scriptsize}
    {\ttfamily Buffer foo.el modified; kill anyway? (yes or no)}
  \end{scriptsize}
\end{center}

Pour répondre, vous devez taper '\texttt{yes}' ou '\texttt{no}' dans
le mini-tampon, suivi par \RET. Le mini-tampon se comporte comme
décrit dans les sections précédentes; vous pouvez passer à une autre
fenêtre avec \texttt{C-x o}, utiliser l'historique des commandes
\texttt{M-p} et \texttt{M-f}, etc. Tapez \texttt{C-g} quitte le
mini-tampon et la commande d'interrogation.