\chapter{Lancer des commandes par leur nom}\etch{6}


Chaque commande Emacs a un nom que vous pouvez utiliser pour
l'exécuter. Pour plus de commodité, de nombreuses commandes ont
également des raccourcis clavier. Vous pouvez exécuter ces commandes
en tapant ces touches, ou les exécuter par leurs noms. La plupart des
commandes Emacs n'ont pas de raccourcis clavier, la seule façon de les
lancer est par leurs noms. (Voir section\cfchs{33}{3} [Raccourcis
clavier], page\cfchsg{33}{3}, pour savoir comment mettre en place les
raccourcis clavier.).

Par convention, un nom de commande se compose d'un ou plusieurs mots,
séparés par des tirets ; par exemple, \texttt{auto-fill-mode} ou
\texttt{manual-entry}. Les noms de commandes utilisent le plus souvent
des mots complets en anglais pour les rendre plus facile à retenir.

Pour exécuter une commande par son nom, commencez par \texttt{M-x},
tapez le nom de la commande, puis terminer par \RET. \texttt{M-x}
utilise le mini-tampon pour lire le nom de la commande. La chaîne
'\texttt{M-x}' apparaît au début du mini-tampon comme invite pour vous
rappeler d'entrer un nom de commande à exécuter. \RET sort le
mini-tampon et exécute la commande. Voir le chapitre\cfch{5}
[Mini-tampon], page\cfchg{5}, pour plus d'informations sur le
mini-tampon.

Vous pouvez utiliser la complétion pour entrer le nom de la
commande. Par exemple, pour appeler la commande \texttt{forward-char},
vous pouvez taper \texttt{M-x forward-char RET} ou \texttt{M-x forw
  TAB c RET}.

Notez que \texttt{forward-char} est la même commande que vous invoquez
avec la touche \texttt{C-f}. L'existence d'un raccourci ne vous
empêche pas d'exécuter la commande par son nom. 

Pour annuler le \texttt{M-x} et ne pas lancer une commande, tapez
\texttt{C-g} lieu d'entrer le nom de la commande. Cela vous ramène au
niveau de la commande.

Pour passer un argument numérique pour la commande que vous invoquez
avec \texttt{M-x}, spécifiez l'argument numérique avant
\texttt{M-x}. La valeur de l'argument apparaît dans l'invite alors que
le nom de la commande est en cours de lecture, et enfin \texttt{M-x}
passe l'argument à cette commande. 

Lorsque la commande que vous exécutez avec \texttt{M-x} dispose d'un
raccourci, Emacs le mentionne dans la zone écho après l'exécution de
la commande. Par exemple, si vous tapez \texttt{M-x forward-word}, le
message indique que vous pouvez exécuter la même commande en tapant
\texttt{M-f}. Vous pouvez désactiver ces messages en définissant la
variable \texttt{suggest-key-bindings} à \texttt{nil}.

Dans ce manuel, lorsque nous parlons de l'exécution d'une commande par
son nom, nous omettons souvent le \RET qui termine le nom. Ainsi, nous
pourrions dire \texttt{M-x auto-fill-mode RET}. Nous mentionnons le
\RET seulement pour l'accent, comme lorsque la commande est suivie par
des arguments. 

\texttt{M-x} fonctionne en exécutant la commande
\texttt{execute-extended-command}, qui est chargée de lire le nom d'une
autre commande et l'invoquer.

