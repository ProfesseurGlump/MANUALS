\newcommand{\gnue}{GNU Emacs\xspace}
\newcommand{\el}{Emacs Lisp\xspace}
\newcommand{\MIT}{Massachussetts Institute of Technology\xspace}
\newcommand{\cl}{Common Lisp\xspace}

\subsection*{Préface}

La plupart des environnements de développement intégrés \gnue  sont
écrits dans le langage de programmation \el . Le code écrit dans
ce langage de programmation est le logiciel---l'ensemble des
instructions---qui disent à l'ordinateur quoi faire quand vous lui
donnez des commandes. Emacs est conçu de sorte que vous pouvez écrire
du nouveau code \el  et l'installer simplement comme une
extension de l'éditeur.

(\gnue  est parfois appelé <<l'éditeur modifiable>>, mais il fait
bien plus que fournir des possibilités d'édition. C'est mieux de se
référer à Emacs comme un <<environnement de développement
modifiable>>. Toutefois, cette phrase est un peu pompeuse. C'est plus
simple de se référer à Emacs tout simplement comme un éditeur. De
plus, tout ce que vous faîtes avec Emacs---trouver une date du
calendrier Maya et les phases de la lune, simplifier des polynômes,
déboguer du code, gérer des fichiers, lire des lettres, écrire des
livres---toutes ces activités sont des façons d'éditer dans le sens le
plus général du mot.)

Aussi \el  est habituellement pensé seulement en association
avec Emacs, c'est un langage de programmation complet.Vous pouvez
utiliser \el  comme vous le feriez pour n'importe quel autre
langage de programmation.

Peut être vous voulez comprendre la programmation; peut être vous
voulez modifier Emacs; ou peut être vous voulez devenir un
programmeur. Cette introduction à \el  est conçue pour vous
faire démarrer: vous guider dans l'apprentissage des fondamentaux de
la programmation, et plus important, vous montrer comment apprendre à
aller plus loin.

\subsection*{En lisant ce texte}

\`A travers ce document, vous verrez un petit échantillon de
programmes que vous pouvez lancer dans Emacs. Si vous lisez ce
document dans Info dans \gnue , vous pouvez lancer les programmes
comme ils apparaissent. (C'est facile à faire et c'est expliqué
lorsque les programmes sont présentés.) Alternativement, vous pouvez
lire cette introduction comme un livre imprimé pendant que vous êtes
assis devant un ordinateur faisant tourner Emacs. (C'est ce que j'aime
faire; j'aime les livre imprimés.) Si vous n'avez pas un Emacs qui
tourne devant vous, vous pouvez toujours lire ce livre, mais dans ce
cas, le mieux c'est de le traiter comme une nouvelle ou un guide de
voyage d'un pays pas encore visité: intéressant, mais pas pareil que
d'y être.

La plupart de cette introduction est dédiée aux walkthroughs ou tours
guidés du code utilisé dans \gnue . Ces tours sont conçus avec deux
objectifs : d'abord, vous familiariser avec le réel, le code de
travail (le code utilisé tous les jours); et, ensuite, vous
familiariser avec la façon dont Emacs fonctionne. C'est très
intéressant de voir comment un environnement de travail est
implémenté. Aussi, j'espère que vous prendrez l'habitude de naviguer
dans le code source. Vous pouvez y apprendre et emprunter des
idées. Avoir \gnue est comme avoir une cave de dragons pleine de
trésors.

En supplément d'en apprendre sur Emacs comme un éditeur et \el comme
un langage de programmation, les exemples et les tours guidés vous
donneront une opportunité d'avoir de l'acointance avec Emacs comme un
environnement de programmation Lisp. \gnue supporte la programmation et
fournit des outils que vous voudrez pour devenir à l'aise dans l'usage
comme \texttt{M-}. (la touche invoque la commande
\texttt{find-tag}). Vous apprendrez aussi à propos des tampons et
autre objets qui font partie de l'environnement. En apprendre au sujet
de ces fonctionnalités d'Emacs est comme apprendre les nouvelles
routes autour de votre ville.

Finalement, j'espère convey une partie des compétences pour utiliser
Emacs et apprendre les aspects de la programmation que vous ne
connaissez pas. Vous pouvez utiliser souvent Emacs pour vous aider à
comprendre ce qui vous tracasse ou pour trouver comment faire quelque
chose de nouveau. Cette auto-reliance n'est pas seulement un plaisir,
mais aussi un avantage.

\subsection*{\`A qui s'adresse ceci}

Ce texte est écrit comme une introduction élémentaire pour des gens
qui ne sont pas programmeurs. Si vous êtes un programmeur, vous pouvez
ne pas être satisfait par ceci. La raison est que vous êtes
certainement devenu un expert de la lecture des manuels de référence
et exaspéré par la façon dont ce texte est organisé.

Un programmeur expert qui a relu ce texte m'a dit:

{\it Je préfère apprendre avec les manuels de référence. Je <<plonge
dans>> chaque paragraphe et <<ressors prendre de l'air>> entre les
paragraphes.

Quand j'arrive à la fin d'un paragraphe, je suppose que ce sujet est
traité, fini, que je sais tout ce dont j'ai besoin (avec la possible
exception du cas où le paragraphe suivant démarre en en parlant avec
plus de détail). Je m'attends à ce qu'un bon manuel de référence
n'aura pas beaucoup de redondance, et qu'il aura d'excellent pointeurs
vers l'endroit où trouver l'information que je souhaite.
}

Cette introduction n'est pas écrite pour cette personne!

D'abord, j'essaie de dire toute chose au moins trois fois: un, pour
l'introduire; deux, pour la montrer dans le contexte; et trois, pour
la montrer dans un contexte différent, ou pour la revoir.

Ensuite, j'ai toujours difficilement mis toute l'information à propos
d'un sujet  dans un endroit, dans moins qu'un paragraphe. Dans ma
façon de penser, cela impose un pavet trop lourd pour le lecteur. \`A
la place j'essaie d'expliquer seulement ce dont vous avez besoin de
savoir au moment idoine. (Parfois j'inclus un peu d'information
supplémentaire donc vous ne serez pas surpris plus tard quand
l'information supplémentaire sera formellement introduite.)

Quand vous lisez ce texte, vous ne vous attendez pas à apprendre tout
la première fois. Souvent, vous aurez seulement besoin de faire, comme
c'était, un <<nodding acquaintance>> avec certains des items
mentionés. Mon espoir est d'avoir structuré le texte et de vous avoir
donné assez d'indice qui vous alerteront sur ce qui est important, et
que vous vous concentrerez deçu.

Vous aurez besoin de <<plonger dans>> certains paragrpahes; il n'y a
pas d'autre façon de les lire. Mais j'ai essayé de réduire le nombre
de ces paragraphes. Ce livre est intended comme une colline
approchable, plutôt que comme une montagne escarpée.

Cette introduction \textit{Programming in \el} a un document
compagnon, \textit{The \gnue Lisp Reference Manual}. Le manuel de
référence a beaucoup plus de détails que cette introduction. Dans le
manuel de référence, toute l'information à propos d'un sujet est
concentrée dans un endroit. Vous devriez y aller si vous êtes comme le
programmeur cité plus haut. Et, bien sûr, après avoir lu cette
\textit{Introduction}, vous trouverez le \textit{Reference Manual}
utile wuand vous écrirez vos propres programmes.

\subsection*{Histoire de Lisp}

Lisp a été développé pour la première fois à la fin des années 50 au
\MIT pour la recherche en intelligence artificielle. La super
puissance du langage Lisp qui le rend supérieur pour d'autres sujets
est tant, l'écriture de commandes d'édition et les environnements
intégrés.

\gnue Lisp est largement inspiré de Maclisp, qui a été écrit au MIT
dans les années 60. C'est en partie ce qui a inspiré \cl, qui
est devenu un standard dans les années 80. Toutefois, \el est beaucoup
plus simple que \cl. (La distribution standard d'Emacs
contient une option d'extension de fichier, \texttt{cl.el} qui ajoute
beaucoup de fonctionnalités de \cl à \el .)

\subsection*{Remarque pour les novices}

Si vous ne connaissez pas \gnue , vous pouvez toujours lire ce
document avec profit. Toutefois, je recommande d'apprendre Emacs, au
moins d'apprendre comment vous déplacer sur l'écran de votre
ordinateur. Vous pouvez apprendre seul comment utilier Emacs avec le
tutoriel en ligne. Pour l'utiliser, taper \texttt{C-h t}. (Cela
signifie pressez et relachez la touche \texttt{CTRL} et le \texttt{h}
en même temps, et puis pressez et relachez \texttt{t}.)

Aussi, je ferais souvent référence à l'une des commandes standard
d'Emacs en listant les touches que vous presserez pour invoquer la
commande et ensuite en donnant le nom de la commande entre
parenthèses, comme ça : \texttt{M-C-\textbackslash} (\texttt{indent-region}). Ce
qui signifie que la commande \texttt{indent-region} est
personnellement invoquée en tapant \texttt{M-C-\textbackslash}. (Vous pouvez, si
vous souhaitez, changer les touches qui sont tapées pour invoquer la
commande; cela est appelé \textit{redéfinir}. Voir section
\cfchs{16}{8} <<Keymaps>>, page \cfchsg{16}{8}.) L'abréviation
\texttt{M-C-\textbackslash} signifie que vous tapez votre touche \texttt{META},
touche \texttt{CTRL} end touche \texttt{\textbackslash} toutes en même temps. (Sur
de nombreux claviers modernes la touche \texttt{META} est étiquetée
\texttt{ALT}.) Parfois une combinaison comme celle-là est appelée un
accord de touche, de façon similaire à un accord de piano. Si votre
clavier n'a pas de touche \texttt{META}, la touche \texttt{ESC}
préfixée est utilisée à la place. Dans ce cas, \texttt{M-C-\textbackslash} signifie
que vous pressez puis relachez la touche \texttt{ESC} et après tapez
la touche \texttt{CTRL} et la touche \texttt{\textbackslash} en même temps. Mais
usuellement \texttt{M-C-\textbackslash} signifie pressez la touche \texttt{CTRL}
along avec la touche nommée \texttt{ALT} et, en même temps, pressez la
touche \texttt{\textbackslash}.

De plus taper a lone accord de touche, vous pouvez préfixer ce que
vous tapez avec \texttt{C-u}, qui est appelé <<l'argument
universel>>. L'accord de touche \texttt{C-u} passe un argument à la
commande subsequent. Alors, pour indenter une région de texte plein
par 6 espaces, marquez la région, et ensuite tapez \texttt{C-u 6
M-C\textbackslash}. (Si vous ne spécifiez pas un nombre, Emacs passera le nombre 4
à la commande ou alors lancera la commande différemment que ce que
vous souhaitiez.) Voir section <<arguments numériques>> dans
\textit{The \gnue Manual}.

Si vous êtes en train de lire ça dans Info utilisant \gnue , vous
pouvez lire à travers ça tout le document juste en pressant la barre
espace, \texttt{SPC}. (Pour en apprendre sur Info tapez \texttt{C-h i}
et ensuite sélectionnez Info.)

Une remarque sur la terminologie: quand j'utilise le mot Lisp tout
seul, je fais souvent référence aux variantes des dialectes de Lisp en
général, mais quand je parle de \el , je fais référence à \gnue Lisp
en particulier.

\subsection*{Remerciements}

Mes remerciements à tous ceux qui m'ont aidé pour ce livre. Un
remerciement spécial à Jim Blandy, Noah Friedman, Jim Kingdon, Roland
McGrath, Frank Ritter, Randy Smith, Richard M. Stallman, et Melissa
Weisshaus. Mes remerciements également à Philip Johnson et David
Stampe pour leur encouragement. Mes erreurs sont miennes.

\begin{flushright}
Robert J. Chassell

\url{bob@gnu.org}
\end{flushright}