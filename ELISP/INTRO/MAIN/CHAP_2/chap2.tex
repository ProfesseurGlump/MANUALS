\chapter{\'Evaluation pratique}

Avant d'apprendre à écrire une définition de fonction dans Emacs Lisp,
il est utile de passer un peu de temps à évaluer diverses expressions
qui ont déjà été écrites. Ces expressions seront des listes avec des
fonctions pour premier (et souvent seul) élément. Puisque certaines des
fonctions associées avec des tampons sont à la fois simples et
intéressantes, nous allons commencer avec elles. Dans cette section,
nous allons en évaluer quelques-unes. Dans une autre section, nous
étudierons le code de plusieurs autres fonctions reliées à des
tampons, pour voir comment elles ont été écrites. 

\textit{Chaque fois que vous donnez une commande d'édition} à Emacs
Lisp, comme la commande pour déplacer le curseur ou pour faire défiler
l'écran, \textit{vous évaluez une expression}, le premier élément est
une fonction. \textit{C'est comme cela qu'Emacs fonctionne.}

Lorsque vous tapez des touches, vous provoquez l'interprète Lisp pour
évaluer une expression et c'est comme cela que vous obtenez un
résultat. Même taper le texte brut implique une évaluation d'une
fonction Emacs Lisp, dans ce cas, celle qui utilise
\tm{self-insert-command}, qui insère simplement le caractère que vous
avez saisi. Les fonctions que vous évaluez en tapant des raccourcis
sont appelées fonctions \textit{interactives}, ou \textit{commandes};
comment vous faire une fonction interactive sera illustré dans le
chapitre comment écrire des définitions de fonctions. Voir Section
\cfchs{3}{3} ``Faire une fonction interactive'', page \cfchsg{3}{3}.

En plus de la saisie des commandes clavier, nous avons vu une seconde
façon d'évaluer une expression : en positionnant le curseur après une
liste et en tapant \tm{C-x C-e}. C'est ce que nous allons faire dans
le reste de cette section. Il y a d'autres façons d'évaluer une
expression; celles-ci seront décrites lorsque nous viendront à elles.

En plus d'être utilisées pour pratiquer l'évaluation, les fonctions
présentées dans les prochaines sections sont importantes en
elles-mêmes. Une étude de ces fonctions opère une nette distinction
entre les tampons et les fichiers, comme changer de tampon, et comment
déterminer un emplacement à l'intérieur.

\section{Noms des tampons}\etchs{2}{1}

Les deux fonctions, \tm{buffer-name} et \tm{buffer-file-name},
montrent la différence entre un fichier et un tampon. Lorsque vous
évaluez l'expression suivante, \tm{(buffer-name)}, le nom du tampon
apparaît dans la zone d'écho. Quand vous évaluez
\tm{(buffer-file-name)}, le nom du fichier auquel se réfère le tampon
est affiché dans la zone d'écho. Habituellement, le nom renvoyé par
\tm{(buffer-name)}  est le même que le nom de fichier auquel il fait
référence, et le nom renvoyé par \tm{(buffer-file-name)} est le nom de
chemin complet du fichier. 

Un fichier et un tampon sont deux entités différentes. Un fichier est
une information enregistrée de façon permanente dans l'ordinateur (à
moins que vous ne l'effaciez). Un tampon, d'un autre côté, est une
information à l'intérieur d'Emacs qui va disparaître à la fin de la
session (ou quand vous <<tuez>> le tampon). Habituellement, un tampon
contient des informations que vous avez copié à partir d'un fichier;
nous disons que le tampon \textit{visite} ce fichier. Cette copie est
ce que vous travaillez et modifiez. Les changements du tampon ne
change pas le fichier, jusqu'à ce que vous enregistrez le
tampon. Lorsque vous sauvegardez le tampon, le tampon est copié dans
le fichier et est donc sauvegardé de façon permanente.

Si vous lisez ceici dans Info à l'intérieur de \gem , vous pouvez
évaluer chacune des expressions suivantes en positionnant le curseur
après et en tapant \tm{C-x C-e}. 

\tm{(buffer-name)}

\tm{(buffer-file-name)}

Quand je fais cela dans Info, la valeur renvoyée par l'évaluation de
\tm{(buffer-name)} est \tm{''*info*''}, et la valeur renvoyée par
\tm{(buffer-file-name)} est \tm{nil}.

D'autre part, pendant que je vous écris ce document, la valeur
renvoyée par l'évaluation de \tm{(buffer-name)} est
\tm{''introduction.texinfo''}, et la valeur renvoyée par l'évaluation
de \tm{(buffer-file-name)} est
\tm{''/gnu/work/intro/introduction.texinfo''}.

Le premier est le nom du tampon et le dernier est le nom du
fichier. Dans Info, le nom du tampon est \tm{''*info*''}. Info ne
pointe vers aucun fichier, donc le résultat de l'évaluation de
\tm{(buffer-file-name)} est \tm{nil}. Le symbole \tm{nil} vient du mot
Latin pour 'rien'; dans ce cas, cela signifie que le tampon n'est
associé à aucun fichier. (En Lisp, \tm{nil} est aussi utilisé pour
signifié 'faux' et est un synonyme pour la liste vide, \tm{()}.)

Lorsque j'écris, le nom de mon tampon est
\tm{''introduction.texinfo''}. Le nom du fichier vers lequel il pointe
est \tm{''/gnu/work/intro/introduction.texinfo''}.

(Dans ces expressions, les parenthèses disent à l'interprète Lisp de
traiter \tm{buffer-name} et \tm{buffer-file-name} comme des fonctions;
sans les parenthèses, l'interprète serait tenté d'évaluer les symboles
comme des variables. Voir Section \cfchs{1}{7} ''Variables'', page
\cfchsg{1}{7}.) 

En dépit de la distinction entre les fichiers et les tampons, vous
trouverez souvent que les gens se réfèrent à un fichier quand ils
veulent dire tampon et vice-versa. En effet, la plupart des gens
disent : <<Je suis dans l'édition d'un fichier>>, plutôt que de dire :
<<Je suis dans l'édition d'un tampon que je vais bientôt enregistrer
dans un fichier.>> Il est presque toujours clair à partir du contexte
de deviner ce que les gens veulent dire. Lorsque vous traitez avec des
programmes informatiques, cependant, il est important de garder à
l'esprit la distinction, puisque l'ordinateur n'est pas aussi
intelligent qu'une personne (pour deviner).

Le mot <<tampon>>, en passant, vient de la signification du mot
coussin qui amortit la force d'une collision. Dans les premiers
ordinateurs, un tampon amortissait l'interaction entre les fichiers et
l'unité centrale de traitement de l'ordinateur. Les tambours ou bandes
qui tenaient un fichier et l'unité centrale de traitement étaient des
pièces d'équipement qui étaient très différentes les unes des autres,
à travailler à leurs propres vitesses, par à-coups. Le tampon a rendu
possible pour eux de travailler ensemble efficacement. Finalement, le
tampon est passé du statut d'intermédiaire, un lieu de détention
temporaire, à celui de lieu où le travail se fait. Cette
transformation est un peu comme celle d'un petit port qui a grandi
dans une grande ville : au début il était simplement le lieu où des
marchandises étaint entreposées temporairement avant d'être chargées
sur les bateaux; puis il est devenu un centre commercial et culturel. 

Pas tous les tampons sont associés à des fichiers. Par exemple, un
tampon \tm{*scratch*} ne visite aucun fichier. De même, un tampon
\tm{*Help*} n'est associé à aucun fichier. 

Dans les anciens temps, lorsque vous aviez manqué un fichier
\tm{~/.emacs} et commencé une session Emacs en tapant la commande
\tm{emacs} seule, sans nommer aucun fichier, Emacs démarrait avec le
tampon \tm{*scratch*}. De nos jours, vous verrez un écran de
démarrage. Vous pouvez suivre l'une des commandes proposées sur
l'écran de démarrage, visiter un fichier ou appuyer sur la barre
d'espace pour atteindre le tampon \tm{*scratch*}.

Si vous passez au tampon \tm{*scratch*}, tapez \tm{(buffer-name)},
positionnez le curseur à la suite, et tapez \tm{C-x C-e} pour évaluer
l'expression. Le nom \tm{''*scratch*''} sera renvoyé et apparaîtra
dans la zone d'écho. \tm{''*scratch*''} est le nom du tampon. Quand
vous tapez \tm{(buffer-file-name)} dans le tampon \tm{*scratch*} et
évaluez ça, \tm{nil} apparaît dans la zone d'écho, juste comme ça le
fait lorsque vous évaluez \tm{(buffer-file-name)} dans Info.

Incidemment, si vous êtes dans le tampon \tm{*scratch*} et souhaitez
que la valeur renvoyée par une expression apparaisse dans le tampon
\tm{*scratch*} lui-même au lieu de la zone d'écho, tapez \tm{C-u C-x
  C-e} au lieu de \tm{C-x C-e}. Cela provoque l'apparition de la
valeur renvoyée après l'expression. Le tampon ressemblera à ça :

\tm{(buffer-name)''*scratch*''}

Vous ne pouvez pas faire ça dans Info puisque Info est en lecture
seule et ne vous autorisera pas à changer le contenu de son
tampon. Mais vous pouvez faire cela dans n'importe quel tampon que
vous éditez; et quand vous écrivez du code ou une documentation (comme
ce livre), c'est fonction sont très utiles.



\section{Obtenir un tampon}\etchs{2}{2}

La fonction \tm{buffer-name} renvoie le \textit{nom} du tampon; pour
obtenir le tampon \textit{lui-même}, une fonction différente est
nécessaire : la fonction \tm{current-buffer}. Si vous utilisez cette
fonction dans un code, ce que vous obtenez est le tampon lui-même.

Un nom d'objet et l'objet ou l'entité à laquelle se réfère le nom sont
différents entre eux. Vous n'êtes pas votre nom. Vous êtes une
personne à laquelle les autres font référence par votre nom. Si vous
demandez à parler à George et quelqu'un vous donne une carte avec les
lettres \tm{'G', 'e', 'o', 'r', 'g', 'e'} écrites dessus, vous serez
peut être amusé, mais vous ne serez pas satisfait. Vous ne voulez pas
parler au nom, mais à la personne à laquelle le nom fait
référence. Un tampon est similaire : le nom du tampon scratch est
\tm{*scratch*}, mais le nom n'est pas le tampon. Pour obtenir le
tampon lui-même vous avez besoin d'utiliser une fonction comme
\tm{current-buffer}.

Cependant, il y a une légère complication : si vous évaluez
\tm{current-buffer} dans une expression, comme nous le ferons ici, ce
que vous voyez est une représentation imprimée du nom du tampon sans
le contenu du tampon. Emacs travaille de cette façon pour deux raisons
: la tampon peut être des milliers de lignes de long---trop long pour
être affiché; et, un autre tampon peut avoir le même contenu mais un
nom différent, et il est important de pouvoir les distinguer.

Voici une expression contenant la fonction :

\tm{(current-buffer)}

Si vous évaluez cette expression dans Info dans Emacs de la façon
habituelle, \tm{\#<buffer *info*>} apparaîtra dans la zone d'écho. Le
format spécial indique que le tampon lui-même est renvoyé, plutôt que
juste son nom.

Incidemment, pendant que vous pouvez taper un nombre ou symbole dans
un programme, vous ne pouvez pas faire ça avec une représentation
imprimée d'un tampon : la seule façon d'obtenir un tampon lui-même
c'est avec une fonction telle que \tm{current-buffer}.

Une fonction liée est \tm{other-buffer}. Elle renvoie le tampon le
plus récemment sélectionné autre que celui sur lequel vous êtes en ce
moment, pas une représentation imprimée de son nom. Si vous avez
récemment changé depuis le tampon \tm{*scratch*}, \tm{other-buffer}
renverra ce tampon. 

Vous pouvez voir ça en évaluant l'expression :

\tm{(other-buffer)}

Vous devriez voir \tm{\#<buffer *scratch*>} apparaître dans la zone
d'écho, ou le nom de n'importe quel autre tampon que vous avez changé
récemment.\footnote{En fait, par défaut, si le tampon à partir duquel
  vous venez de changer est visible pour vous dans une autre fenêtre,
  un autre tampon sera choisi, le plus récent que vous ne pouvez pas
  voir; ceci est une subtilité que j'oublie souvent.}


\section{Changer de tampon}\etchs{2}{3}

La fonction \tm{other-buffer} fournit en fait un tampon quand il est
utilisé comme un argument pour une fonction qui en exige un. Nous
pouvons voir cela en utilisant \tm{other-buffer} et
\tm{switch-to-buffer} pour passer à un tampon différent.

Mais d'abord, une brève introduction à la fonction
\tm{switch-to-buffer}. Lorsque vous passez du tampon Info à celui de
\tm{*scratch*} pour évaluer \tm{(buffer-name)}, vous avez probablement
tapé \tm{C-x b} puis \tm{*scratch*}\footnote{Ou plutôt, économiser la
  frappe, vous avez probablement tapé \RET si le tampon par défaut
  était \tm{*scratch*}, ou si c'était différent, alors vous avez juste
tapé une partie du nom, telle que \tm{*sc}, pressé la touche \TAB pour
provoquer la complétion, et ensuite tapé \RET} lorsque vous êtiez
invité à saisir le nom du tampon vers lequel vous souhaitiez aller
dans le mini-tampon. Le raccourci, \tm{C-x b}, provoque l'interprète
Lisp pour évaluer la fonction interactive \tm{switch-to-buffer}. Comme
on l'a dit plus tôt, voilà comment Emacs fonctionne : différent
raccourcis appellent ou exécutent différentes fonctions. Par exemple,
\tm{C-f} appelle \tm{forward-char}, \tm{M-e} appelle
\tm{forward-sentence}, et ainsi de suite.

En écrivant \tm{switch-to-buffer}, et en lui donnant un tampon pour
basculer, nous pouvons changer de tampon juste de la même façon que
fait \tm{C-x b} : 

\tm{(switch-to-buffer (other-buffer))}

Le symbole \tm{switch-to-buffer} est le premier élément de la liste,
de sorte que l'interprète Lisp va le traiter une fonction et exécuter
les instructions qui y sont attachées. Mais avant de faire cela,
l'interprète notera que \tm{other-buffer} est entre parenthèses à
l'intérieur et le travail sur ce premier symbole. \tm{other-buffer}
est le premier (et dans ce cas, le seul) élément de cette liste, donc
l'interprète Lisp appelle et exécute la fonction. Cela renvoie un
autre tampon. Ensuite, l'interprète lance \tm{switch-to-buffer}, en
lui passant, comme un argument, l'autre tampon, qui est celui que
Emacs basculera vers. Si vous lisez ceci dans Info, essayez
maintenant. \'Evaluer l'expression. (Pour revenir, tapez \tm{C-x b
  \RET}.)\footnote{Rappelez-vous, cette expression vous déplacera dans
l'autre tampon le plus récent que vous ne pouvez pas voir. Si vous
voulez vraiment aller vers le tampon le plus récent sélectionné, même
si vous pouvez le voir, vous avez besoin d'évaluer l'expression plus
complexe suivante :

\tm{(switch-to-buffer (other-buffer (current-buffer) t))}

Dans ce cas, le premier argument de \tm{other-buffer} lui dit quel
tampon à sauter---celui en court---et le second argument inqique à
\tm{other-buffer} que c'est OK de changer vers un tampon visible. En
utilisation régulière, \tm{switch-to-buffer} vous amène une fenêtre
invisible puisque vous utilisez le plus souvent \tm{C-x o
  (other-window)} pour aller à un autre tampon visible.}

Dans les exemples de programmation dans les sections suivantes de ce
document, vous verrez la fonction \tm{set-buffer} plus souvent que
\tm{switch-to-buffer}. Ceci est dû à une différence entre les
programmes et les humains : les humains ont des yeux et s'attendent à
voir le tampon sur lequel ils travaillent sur leurs terminaux. Cela
est si évident, que cela va presque sans dire. Toutefois, les
programmes n'ont pas d'yeux. Quand un programme fonctionne sur un
tampon, ce tampon n'a pas besoin d'être visible sur l'écran.

\tm{switch-to-buffer} est conçu pour l'homme et fait deux choses
différentes: il change vers le tampon sur lequel l'attention de Emacs
est dirigé; et il commute le tampon affiché dans la fenêtre pour le
nouveau tampon. \tm{set-buffer}, d'autre part, ne fait qu'une chose :
il passe l'attention du programme à un tampon différent. Le tampon de
l'écran reste inchangé (bien sûr, rien ne se passe normalement jusqu'à
ce que les commandes finissent leurs exécutions).

Aussi, nous venons d'introduire un autre terme de jargon, le mot
\textit{appel}. Lorsque vous évaluez une liste dans laquelle le
premier symbole est une fonction, vous appellez cette
fonction. L'utilisation du terme provient de la notion de la fonction
comme une entité qui peut faire quelque chose pour vous si vous
'l'appelez'---juste comme un plombier est une entité qui peut réparer
une fuite si vous l'appelez.

\section{Différentes options pour \texttt{interactive}}\etchs{3}{4}
Dans l'exemple, \tm{multiply-by-seven} utilisait \tm{''p''} comme un
argument pour \tm{interactive}. Cet argument disait à Emacs
d'interpréter votre frappe soit \tm{C-u} suivi d'un nombre soit
\tm{META} suivi d'un nombre comme une commande à passer ce nombre à la
fonction comme son argument. Emacs a plus de vingt caractères
prédéfinis pour l'utilisation avec \tm{interactive}. Dans presque tous
les cas, une de ces options vous permettra de passer la bonne
information interactivement à la fonction. (Voir Section <<Caractère
de code pour \tm{interactive}>> dans \textit{The GNU Emacs Lisp
  Reference Manual}.)
\begin{center}
  Considérez la fonction \tm{zap-to-char}. Son expression interactive
  est

  \tm{(interactive ''p\textbackslash~nZap to char: '')}
\end{center}

La première partie de l'argument de \tm{interactive} est \tm{'p'},
avec ce dont vous êtes déjà familier. Cet argument dit à Emacs
d'interpréter un <<préfixe>>, comme un nombre qui doit être passé à la
fonction. Vous pouvez spécifier un préfixe soit en tapant \tm{C-u}
suivi d'un nombre soit en tapant \tm{META} suivi par un nombre. Le
préfixe est le nombre de caractères spécifiés. Ainsi, si votre préfixe
est trois et le caractère spécifié est \tm{'x'}, alors vous effacerez
tout le texte jusqu'à et y compris la troisième occurence de
\tm{'x'}. Si vous n'initialisez pas un préfixe, alors vous effacerez
tout le texte jusqu'à et y compris le caractère spécifié, mais pas
plus. 

Le \tm{'c'} indique à la fonction le nom du caractère qu'elle doit
supprimer.

Plus formellement, une fonction avec plusieurs arguments peut avoir
des informations passées à chacun des argument en ajoutant des
parties à la chaîne qui suit \tm{interactive}. Lorsque vous faîtes ça,
l'information est passée à chaque argument dans le même ordre qu'elle
est spécifiée dans la liste \tm{interactive}. Dans la chaîne, chaque
partie est séparée de la suivante par un \tm{'\textbackslash'}, qui
est une nouvelle ligne. Par exemple, vous pouvez suivre \tm{'p'} avec
un \tm{'\textbackslash~n'} et un \tm{'cZap to char: '}. Cela incite
Emacs à passer la valeur de l'argument préfixé (s'il y en a un) et le
caractère. 

Dans ce cas, la définition ressemble à la suivante, où \tm{arg} et
\tm{char} sont les symbole auxquels \tm{interactive} lie l'argument
préfixé et le caractère spécifié :
\begin{center}
  \tm{(defun \textit{name-of-function} (arg char)}

  \tm{''\textit{documentation}\dots''}

  \tm{(interactive ''p\textbackslash~ncZap to char; '')}
  
  \tm{\textit{body-of-function}\dots)}
\end{center}
(L'espace après les deux points dans l'invite lui donne une meilleure
apparence lorsque vous êtes invité. Voir Section \cfchs{5}{1} <<La
définition de \tm{copy-to-buffer}>>, page \cfchsg{5}{1}, par exemple.)

Quand une fonction ne prend pas d'arguments, \tm{interactive} n'en
nécessite aucun. Une telle fonction contient l'expression simple
\tm{(interactive)}. La fonction \tm{mark-whole-buffer} est comme ça.

Alternativement, si les lettres-codes ne sont pas bonnes pour votre
application, vous pouvez passer vos propres arguments à
\tm{interactive} comme une liste.

Voir Section \cfchs{4}{4} <<La définition de \tm{append-to-buffer}>>,
page \cfchsg{4}{4}, par exemple. Voir Section <<Utilisation de
\tm{Interactive}>> dans \textit{The GNU Emacs Lisp Reference Manual},
pour une explication plus complète à propos de cette technique. 

\section{Installer du code de façon permanente}\etchs{3}{5}
