\chapter{\'Evaluation pratique}

Avant d'apprendre à écrire une définition de fonction dans Emacs Lisp,
il est utile de passer un peu de temps à évaluer diverses expressions
qui ont déjà été écrites. Ces expressions seront des listes avec des
fonctions pour premier (et souvent seul) élément. Puisque certaines des
fonctions associées avec des tampons sont à la fois simples et
intéressantes, nous allons commencer avec elles. Dans cette section,
nous allons en évaluer quelques-unes. Dans une autre section, nous
étudierons le code de plusieurs autres fonctions reliées à des
tampons, pour voir comment elles ont été écrites. 

\textit{Chaque fois que vous donnez une commande d'édition} à Emacs
Lisp, comme la commande pour déplacer le curseur ou pour faire défiler
l'écran, \textit{vous évaluez une expression}, le premier élément est
une fonction. \textit{C'est comme cela qu'Emacs fonctionne.}

Lorsque vous tapez des touches, vous provoquez l'interprète Lisp pour
évaluer une expression et c'est comme cela que vous obtenez un
résultat. Même taper le texte brut implique une évaluation d'une
fonction Emacs Lisp, dans ce cas, celle qui utilise
\tm{self-insert-command}, qui insère simplement le caractère que vous
avez saisi. Les fonctions que vous évaluez en tapant des raccourcis
sont appelées fonctions \textit{interactives}, ou \textit{commandes};
comment vous faire une fonction interactive sera illustré dans le
chapitre comment écrire des définitions de fonctions. Voir Section
\cfchs{3}{3} ``Faire une fonction interactive'', page \cfchsg{3}{3}.

En plus de la saisie des commandes clavier, nous avons vu une seconde
façon d'évaluer une expression : en positionnant le curseur après une
liste et en tapant \tm{C-x C-e}. C'est ce que nous allons faire dans
le reste de cette section. Il y a d'autres façons d'évaluer une
expression; celles-ci seront décrites lorsque nous viendront à elles.

En plus d'être utilisées pour pratiquer l'évaluation, les fonctions
présentées dans les prochaines sections sont importantes en
elles-mêmes. Une étude de ces fonctions opère une nette distinction
entre les tampons et les fichiers, comme changer de tampon, et comment
déterminer un emplacement à l'intérieur.

\section{Point}\etchs{1}{1}
Le curseur de la fenêtre sélectionnée indique l'emplacement où la
plupart des commandes d'édition prennent effet, qui est appelé
point\footnote{Le terme <<point>> vient du caractère <<.>>, qui était la
  commande TECO (le langage dans lequel l'Emacs original a été
  écrit) pour accéder à la position d'édition.}. Beaucoup
de commandes Emacs déplacent le point à différents endroits dans le
tampon; par exemple, vous pouvez placer en cliquant sur le bouton 1 de
la souris (usuellement le gauche) à l'endroit désiré. Par défaut, le
curseur de la fenêtre sélectionnée est dessiné comme un bloc plein et
paraît être un caractère, mais vous devrez penser le point comme entre
deux caractères; il est situé avant le caractère sous le curseur. Par
exemple; si votre texte ressemble à <<frob>> avec le curseur sur <<b>>,
alors le point est entre le <<o>> et le <<b>>. Si vous insérez le
caractère <<!>> à cette position, le résultat sera <<fro!b>>, avec le
point entre le <<!>> et le <<b>>. Ainsi, le curseur reste au-dessus du
<<b>>, comme précédemment.\par 

Si vous éditez plusieurs fichiers dans Emacs, chacun dans son propre
tampon, chaque tampon a sa propre valeur du point. Un tampon qui n'est
pas actuellement affiché se souvient encore de sa valeur du point si vous
l'afficher plus tard. En outre, si un tampon est affiché dans
plusieurs fenêtres, chacune de ces fenêtres a sa propre valeur du
point.\par

Voir la section\cfchs{11}{20} [Affichage Curseur],
page\cfchsg{11}{20}, pour les options qui contrôlent la façon
dont Emacs affiche le curseur.\par 


\section{Installer une définition de fonction}\etchs{3}{2}

Si vous lisez ceci à l'intérieur d'Info dans Emacs, vous pouvez
essayer la fonction \tm{multiply-by-seven} en évaluant d'abord la
définition de fonction puis en évaluant \tm{(multiply-by-seven
  3)}. Une copie de la définition de la fonction suivra. Placez le
curseur après la dernière parenthèse de la définition de fonction et
tapez \tm{C-x C-e}. Lorsque vous faites cela, \tm{multiply-by-seven}
apparaîtra dans la zone d'écho. (Ce que cela signifie c'est que
lorsqu'une définition de fonction est évaluée, la valeur renvoyée est
le nom de la fonction définie.) En même temps, cette action installe
la définition de fonction. 

\tm{(defun multiply-by-seven (number)}

\tm{   ``Multiply NUMBER by seven.''}

\tm{ (* 7 number))}

En évaluant ce \tm{defun}, vous venez d'installer
\tm{multiply-by-seven} dans Emacs. La fonction fait désormais autant
partie de Emacs que \tm{forward-word} ou toute autre fonction
d'édition que vous utilisez. (\tm{multiply-by-seven} restera installé
jusqu'à ce que vous quittiez Emacs. Pour recharger le code
automatiquement chaque fois que vous démarrez Emacs, voir Section
\cfchs{3}{5} ``Installation permanente de code'', page \cfchs{3}{5}.)

Vous pouvez voir l'effet de l'installation de \tm{multiply-by-seven}
en évaluant l'échantillon suivant. Placez le curseur après
l'expression suivante et tapez \tm{C-x C-e}. Le nombre 21 apparaîtra
dans la zone d'écho.

\tm{(multiply-by-seven 3)}

Si vous le souhaitez, vous pouvez lire la documentation de la fonction
en tapant \tm{C-h f (describe-function)} puis le nom de la fonction,
\tm{multiply-by-seven}. Lorsque vous faîtes cela, une fenêtre
\tm{*Help*} apparaîtra sur votre écran qui dit :

\tm{multiply-by-seven is a Lisp function.}

\tm{(multiply-by-seven NUMBER)}

\tm{Multiply NUMBER by seven.}

(Pour revenir à une fenêtre simple sur votre écran, tapez \tm{C-x 1}.)
 
\subsection{Changer une définition de fonction}\etchss{3}{2}{1}

Si vous souhaitez modifier le code dans \tm{multiply-by-seven}, alors
réécrivez-le. Pour installer la nouvelle version à la place de
l'ancienne, évaluez à nouveau la définition de fonction. Ceci est la
façon dont vous modifiez le code dans Emacs. C'est très simple.

\`A titre d'exemple, vous pouvez modifier la fonction
\tm{multiply-by-seven} de sorte qu'elle ajoute le nombre à lui-même
sept fois au lieu de multiplier le nombre par sept. Cela produit le
même résultat, mais par un chemin différent. En même temps, nous
ajouterons un commentaire à ce code; un commentaire est un texte que
l'interprète Lisp ignore, mais qu'un lecteur humain peut trouver utile
ou instructif. Le commentaire est <<seconde version>>.

\tm{(defun multiply-by-seven (number)   ;} Second version.

\tm{ ``Multiply NUMBER by seven.''}

\tm{ (+ number number number number number number number))}

Le commentaire suit un point-virgule, '\tm{;}'. En Lisp, tout sur une
ligne qui suit un point-virgule est un commentaire. La fin de ligne
est la fin du commentaire. Pour étirer un commentaire sur deux ou
plusieurs lignes, chaque ligne doit commencer par un point-virgule.

Voir Section 16.3 ``Commencer un fichier \tm{.emacs}'', page 187, et
Section ``Commentaires'' dans \textit{The GNU Emacs Lisp Reference
  Manual}, pour plus de renseignements à propos des commentaires.

Vous pouvez installer cette version de la fonction
\tm{multiply-by-seven} en l'évaluant de la même manière que vous avez
évalué la première fonction : placez le curseur après la dernière
parenthèse et tapez \tm{C-x C-e}.

En résumé, voici comment vous écrivez du code en Emacs Lisp: vous
écrivez une fonction; vous l'installez; vous la testez; et ensuite
vous faites des corrections ou améliorations et vous l'installez à
nouveau. 

\section{Générer un message d'erreur}\etchs{1}{3}

Partant de sorte que vous ne vous inquiétez pas si vous le faites
accidentellement, nous allons maintenant donner un ordre à
l'interprète Lisp qui génère un message d'erreur. C'est une activité
inoffensive; et en effet, nous allons souvent essayer de générer des
messages d'erreur intentionnellement. Une fois que vous comprenez le
jargon, les messages d'erreur peuvent être informatif. Au lieu d'être
appelé ``messages d'erreur'', ils devraient être appelés messages
``d'aide''. Ils sont comme des panneaux pour un voyageur dans un pays
étranger ; les déchiffrer peut être difficile, mais une fois compris,
ils peuvent indiquer la voie. 

Le message d'erreur est généré par un débogueur \gem intégré. Nous
allons <<entrer dans le débogueur>>. Pour sortir du débogueur taper
\tm{q}.

Ce que nous allons faire c'est d'évaluer une liste qui n'a pas de
quote la précédant et qui n'a pas de commande significative comme
premier élément. Voici une liste presque exactement la même que celle
que nous avons utilisé, mais sans la quote la précédant. Placez le
curseur juste après et taper \rec{C}{x}{C}{e}:
\begin{center}
  \tm{(this is an unquoted list)}
\end{center}

(this is an unquoted list)
Une fenêtre \tm{*Backtrace*} s'ouvrira et vous devriez voir ce qui suit:
{\ttfamily
\begin{flushleft}
  Debugger entered--Lisp error: (void-function this)
  
  (this is an unquoted list)

  eval((this is an unquoted list))

  eval-last-sexp-1(nil)

  eval-last-sexp(nil)

  call-interactively(eval-last-sexp)
\end{flushleft}}

Votre curseur sera dans cette fenêtre (vous pouvez avoir à attendre
quelques secondes avant qu'il ne devienne visible). Pour quitter le
débogueur et faire la fenêtre du débogueur s'en aller, tapez: \tm{q}.

S'il vous plaît tapez \tm{q} maintenant, alors vous devenez plus
confiant et vous pouvez sortir du débogueur. Ensuite, tapez
\rec{C}{x}{C}{e} de nouveau pour y rentrer.

Sur la base de ce que nous savons déjà, nous pouvons presque lire ce
message d'erreur.

Vous avez lu le tampon \tm{*Backtrace*} de bas en haut; il vous dit ce
que Emacs fait. Lorsque vous avez tapé \rec{C}{x}{C}{e}, vous avez
fait un appel à la commande \tm{eval-last-sexp}. \tm{eval} est une
abréviation pour <<évaluer>> et \tm{sexp} est une abréviation pour
<<expression symbolique>>. La signification de cette commande est
<<évaluer la dernière expression symbolique>>, qui est l'expression
juste avant le curseur.

Chaque ligne ci-dessus vous indique ce que l'interprète Lisp évalue
après. L'action la plus récente est en haut. Le tampon est appelé
\tm{*Backtrace*} car il vous permet de suivre Emacs en arrière.

En haut du tampon \tm{*Backtrace*}, vous voyez la ligne :
\begin{center}
  \tm{Debugger entered--Lisp error: (void-function this)}
\end{center}

L'interprète Lisp a tenté d'évaluer le premier atome de la liste, le
mot <<this>>. C'est cette action qui a généré le message d'erreur
<<void-finction this>>. 

Le message contient les mots <<void-function>> et <<this>>. Le mot
<<function>> a été mentionné une seule fois auparavant. C'est un mot
très important.

Pour nos fins, nous pouvons la définir en disant qu'une fonction est
un ensemble d'instructions données à l'ordinateur qui signale à
l'oridnateur de faire quelque chose. 

Maintenant, nous pouvons commencer à comprendre le message d'erreur:
<<\tm{void-function this}>>. La fonction (ce mot est le mot
<<\tm{this}>>) n'a pas de définition d'un ensemble d'instructions que
l'ordinateur peut mener à bien. 

Le mot un peu bizarre, <<\tm{void-function}>>, est conçu pour couvrir
la façon dont Emacs Lisp est mis en \oe{}uvre, qui est que lorsqu'un
symbole n'a pas de définition de fonction attaché à lui, la place qui
doit contenir les instructions est <<\tm{void}>>.

D'autre part, puisque nous avons pu ajouter 2 plus 2 avec succès, en
évaluant \tm{(+ 2 2)} nous pouvons en déduire que le symbole \tm{+}
doit avoir un ensemble d'instructions pour l'ordinateur qui doit obéir
et ces instructions doivent être d'ajouter les nombres qui suivent le
\tm{+}.

Il est possible de prévenir Emacs d'entrer dans le débogueur dans de
tels cas. Nous n'expliquons pas comment faire ici, mais nous allons
parler de ce à quoi le résultat ressemble, parce que vous pouvez
rencontrer une situation similaire s'il y a un bogue dans un code
Emacs que vous utilisez. Dans de tels cas, vous verrez seulement une
ligne de message d'erreur ; il apparaîtra dans la zone écho et
ressemblera à ceci:
\begin{center}
  \tm{Symbol's function definition is void: this}
\end{center}

Le message disparaît dès que vous tapez une touche, même juste pour
déplacer le curseur. Nous connaissons le sens du mot
<<\tm{Symbol}>>. Il se réfère au premier atome de la liste, le mot
<<\tm{this}>>. Le mot <<\tm{function}>> se réfère aux instructions qui
indiquent à l'ordinateur ce qu'il faut faire. (Techniquement, le
symbole indique à l'ordinateur où trouver les instructions, mais c'est
une complication que nous pouvons ignorer pour le moment.) Le message
d'erreur peut être compris: <<\tm{Symbol's function definition is
  void: this}>>. Le symbole (c'est le mot <<\tm{this}>>) manque
d'instructions pour que l'ordinateur les mènent à bien. 




\section{Taille du tampon et localisation du point}\etchs{2}{4}

Enfin, regardons plusieurs fonctions assez simples, \tm{buffer-size,
  point-min} et \tm{point-max}. Celles-ci donnent des informations sur
la taille d'un tampon et l'emplacement du point en son sein.

La fonction \tm{buffer-size} vous indique la taille du tampon courant;
autrement dit, la fonction renvoie un compte du nombre de caractères
dans le tampon.

\tm{(buffer-size)}

Vous pouvez évaluer cela de façon habituelle, en positionnant le
curseur après l'expression et en tapant \tm{C-x C-e}.

Dans Emacs, la position actuelle du curseur est appelée le
\textit{point}. L'expression \tm{(point)} renvoie un nombre qui vous
indique où se trouve le curseur comme un comptage de nombre de
caractères à partir du début du tampon jusqu'au point.

Vous pouvez voir le nombre de caractères pour le point dans ce tampon
en évaluant l'expression suivante de la manière usuelle :

\tm{(point)}

Pendant que j'écris ceci, la valeur du point est 65724. La fonction
\tm{point} est utilisée fréquemment dans certains exemples plus loin
dans ce livre. 

La valeur du point dépend, bien sûr, de son emplacement dans le
tampon. Si vous évaluez le point à cet endroit, le nombre sera plus
grand :

\tm{(point)}

Pour moi, la valeur du point à cet endroit est 66043, ce qui signifie
qu'il y a 319 caractères (espaces compris) entre les deux
expressions. (Sans doute, vous verrez des nombres différents, puisque
j'aurais édité depuis ma première évaluation du point.)

La fonction \tm{point-min} est quelque peu semblable à un \tm{point}, mais
elle renvoie la valeur de la valeur minimale admissible du point dans
le tampon courant. C'est le nombre 1 sauf si le \textit{réduction} est
en vigueur. (La réduction est un mécanisme par lequel vous pouvez vous
limiter, ou un programme, d'opérations sur seulement une partie d'un
tampon. Voir le chapitre \cfch{6} ``Réduction et élargissement'', page
\cfchg{6}.) De même, la fonction \tm{point-max} renvoie la valeur de
la valeur maximale admissible du point dans le tampon courant.


\section{Interpréteur Lisp}\etchs{1}{5}

Sur la base de ce que nous avons vu, nous pouvons maintenant commencer
à comprendre ce que l'interprète Lisp fait lorsque nous commandons à
évaluer une liste. Tout d'abord, il semble pour voir s'il y a une
quote avant que la liste ; s'il y a, l'interprète nous donne juste
la liste. D'un autre côté, s'il n'y a pas de quote, l'interprète
regarde le premier élément de la liste et voit si elle a une
définition de fonction. Si c'est le cas, l'interpréteur exécute les
instructions dans la définition de fonction. Sinon, l'interpréteur
affiche un message d'erreur. 

C'est ainsi que fonctionne Lisp simple. Il y a des complications
supplémentaires que nous allons obtenir dans une minute, mais ce sont
les fondamentaux. Bien sûr, pour écrire des programmes Lisp, vous
devez savoir comment écrire les définitions de fonctions et leur
donner des noms, et comment le faire sans confondre vous-même ou
l'ordinateur. 

Maintenant, pour la première complication, en plus de la liste,
l'interpréteur Lisp peut évaluer un symbole qui n'a pas de quote et
qui n'a pas de parenthèses autour de lui. L'interprète Lisp tentera de
déterminer la valeur du symbole comme une variable. Cette situation
est décrite dans la section sur les variables. (Voir la
section\cfchs{1}{7} <<Variables>>, page\cfchsg{1}{7}.)

La deuxième complication se produit parce que certaines fonctions sont
inhabituelles et ne fonctionnent pas de manière habituelle. Celles qui
ne le font pas sont appelées formes spéciales. Elles sont utilisées
pour des tâches spéciales, comme la définition d'une fonction, et il
n'y en a pas beaucoup. Dans les prochains chapitres, vous serez initié
à plusieurs de ces formes spéciales les plus importantes. 

La troisième et dernière complication est la suivante: si la fonction
que l'interprète Lisp regarde n'est pas une forme particulière, et si
elle fait partie d'une liste, l'interprète Lisp regarde pour voir si
la liste a une liste à l'intérieur de celui-ci. S'il y a une liste
interne, l'interprète Lisp décide ce qu'il doit faire avec la
liste intérieur en premier, puis il travaille sur la liste
extérieure. S'il y a encore une autre liste incorporée à l'intérieur
de la liste intérieure, cela fonctionne sur celui-là en premier, et
ainsi de suite. Il travaille toujours sur la liste la plus profonde en
premier. L'interprète travaille sur la liste la plus interne d'abord,
pour évaluer le résultat de cette liste. Le résultat peut être utilisé
par l'expression l'encapsulant.

Sinon, l'interprète travaille de gauche à droite, d'une expression à
l'autre.

\subsection{Compilation d'octet}\etchss{1}{5}{1}

Un autre aspect de l'interprétation : l'interprète Lisp est capable
d'interpréter deux types d'entités : du code lisible par des humains,
sur lequel nous nous concentrerons exclusivement, et du code
spécialement traité, appelé byte code compilé, qui n'est pas lisible
par les humains. Le byte code compilé s'exécute plus rapidement que le
code lisible par les humains. 

Vous pouvez transformer un code lisible pour les humains en un byte
code compilé en exécutant l'une des commandes de compilation telles
que \tm{byte-compile-file}. Le byte code compilé est généralement
stocké dans un fichier qui se termine par une extension \tm{.elc}
plutôt qu'une extension \tm{.el}. Vous verrez deux types de fichier
dans le répertoire \tm{emacs/lisp} ; les fichiers à lire sont ceux qui
portent l'extension \tm{.el}.

En pratique, pour la plupart des choses que vous pourriez faire pour
personnaliser ou étendre Emacs, vous n'avez pas besoin d'une
compilation de byte code ; et je ne vais pas en parler ici. Voir la
section <<Byte compilation>> dans le manuel Emacs Lisp reference, pour
une description complète de la byte compilation.
 


