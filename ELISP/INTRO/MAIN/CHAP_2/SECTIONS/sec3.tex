\section{Changer de tampon}\etchs{2}{3}

La fonction \tm{other-buffer} fournit en fait un tampon quand il est
utilisé comme un argument pour une fonction qui en exige un. Nous
pouvons voir cela en utilisant \tm{other-buffer} et
\tm{switch-to-buffer} pour passer à un tampon différent.

Mais d'abord, une brève introduction à la fonction
\tm{switch-to-buffer}. Lorsque vous passez du tampon Info à celui de
\tm{*scratch*} pour évaluer \tm{(buffer-name)}, vous avez probablement
tapé \tm{C-x b} puis \tm{*scratch*}\footnote{Ou plutôt, économiser la
  frappe, vous avez probablement tapé \RET si le tampon par défaut
  était \tm{*scratch*}, ou si c'était différent, alors vous avez juste
tapé une partie du nom, telle que \tm{*sc}, pressé la touche \TAB pour
provoquer la complétion, et ensuite tapé \RET} lorsque vous êtiez
invité à saisir le nom du tampon vers lequel vous souhaitiez aller
dans le mini-tampon. Le raccourci, \tm{C-x b}, provoque l'interprète
Lisp pour évaluer la fonction interactive \tm{switch-to-buffer}. Comme
on l'a dit plus tôt, voilà comment Emacs fonctionne : différent
raccourcis appellent ou exécutent différentes fonctions. Par exemple,
\tm{C-f} appelle \tm{forward-char}, \tm{M-e} appelle
\tm{forward-sentence}, et ainsi de suite.

En écrivant \tm{switch-to-buffer}, et en lui donnant un tampon pour
basculer, nous pouvons changer de tampon juste de la même façon que
fait \tm{C-x b} : 

\tm{(switch-to-buffer (other-buffer))}

Le symbole \tm{switch-to-buffer} est le premier élément de la liste,
de sorte que l'interprète Lisp va le traiter une fonction et exécuter
les instructions qui y sont attachées. Mais avant de faire cela,
l'interprète notera que \tm{other-buffer} est entre parenthèses à
l'intérieur et le travail sur ce premier symbole. \tm{other-buffer}
est le premier (et dans ce cas, le seul) élément de cette liste, donc
l'interprète Lisp appelle et exécute la fonction. Cela renvoie un
autre tampon. Ensuite, l'interprète lance \tm{switch-to-buffer}, en
lui passant, comme un argument, l'autre tampon, qui est celui que
Emacs basculera vers. Si vous lisez ceci dans Info, essayez
maintenant. \'Evaluer l'expression. (Pour revenir, tapez \tm{C-x b
  \RET}.)\footnote{Rappelez-vous, cette expression vous déplacera dans
l'autre tampon le plus récent que vous ne pouvez pas voir. Si vous
voulez vraiment aller vers le tampon le plus récent sélectionné, même
si vous pouvez le voir, vous avez besoin d'évaluer l'expression plus
complexe suivante :

\tm{(switch-to-buffer (other-buffer (current-buffer) t))}

Dans ce cas, le premier argument de \tm{other-buffer} lui dit quel
tampon à sauter---celui en court---et le second argument inqique à
\tm{other-buffer} que c'est OK de changer vers un tampon visible. En
utilisation régulière, \tm{switch-to-buffer} vous amène une fenêtre
invisible puisque vous utilisez le plus souvent \tm{C-x o
  (other-window)} pour aller à un autre tampon visible.}

Dans les exemples de programmation dans les sections suivantes de ce
document, vous verrez la fonction \tm{set-buffer} plus souvent que
\tm{switch-to-buffer}. Ceci est dû à une différence entre les
programmes et les humains : les humains ont des yeux et s'attendent à
voir le tampon sur lequel ils travaillent sur leurs terminaux. Cela
est si évident, que cela va presque sans dire. Toutefois, les
programmes n'ont pas d'yeux. Quand un programme fonctionne sur un
tampon, ce tampon n'a pas besoin d'être visible sur l'écran.

\tm{switch-to-buffer} est conçu pour l'homme et fait deux choses
différentes: il change vers le tampon sur lequel l'attention de Emacs
est dirigé; et il commute le tampon affiché dans la fenêtre pour le
nouveau tampon. \tm{set-buffer}, d'autre part, ne fait qu'une chose :
il passe l'attention du programme à un tampon différent. Le tampon de
l'écran reste inchangé (bien sûr, rien ne se passe normalement jusqu'à
ce que les commandes finissent leurs exécutions).

Aussi, nous venons d'introduire un autre terme de jargon, le mot
\textit{appel}. Lorsque vous évaluez une liste dans laquelle le
premier symbole est une fonction, vous appellez cette
fonction. L'utilisation du terme provient de la notion de la fonction
comme une entité qui peut faire quelque chose pour vous si vous
'l'appelez'---juste comme un plombier est une entité qui peut réparer
une fuite si vous l'appelez.