\section{Noms des tampons}\etchs{2}{1}

Les deux fonctions, \tm{buffer-name} et \tm{buffer-file-name},
montrent la différence entre un fichier et un tampon. Lorsque vous
évaluez l'expression suivante, \tm{(buffer-name)}, le nom du tampon
apparaît dans la zone d'écho. Quand vous évaluez
\tm{(buffer-file-name)}, le nom du fichier auquel se réfère le tampon
est affiché dans la zone d'écho. Habituellement, le nom renvoyé par
\tm{(buffer-name)}  est le même que le nom de fichier auquel il fait
référence, et le nom renvoyé par \tm{(buffer-file-name)} est le nom de
chemin complet du fichier. 

Un fichier et un tampon sont deux entités différentes. Un fichier est
une information enregistrée de façon permanente dans l'ordinateur (à
moins que vous ne l'effaciez). Un tampon, d'un autre côté, est une
information à l'intérieur d'Emacs qui va disparaître à la fin de la
session (ou quand vous <<tuez>> le tampon). Habituellement, un tampon
contient des informations que vous avez copié à partir d'un fichier;
nous disons que le tampon \textit{visite} ce fichier. Cette copie est
ce que vous travaillez et modifiez. Les changements du tampon ne
change pas le fichier, jusqu'à ce que vous enregistrez le
tampon. Lorsque vous sauvegardez le tampon, le tampon est copié dans
le fichier et est donc sauvegardé de façon permanente.

Si vous lisez ceici dans Info à l'intérieur de \gem , vous pouvez
évaluer chacune des expressions suivantes en positionnant le curseur
après et en tapant \tm{C-x C-e}. 

\tm{(buffer-name)}

\tm{(buffer-file-name)}

Quand je fais cela dans Info, la valeur renvoyée par l'évaluation de
\tm{(buffer-name)} est \tm{''*info*''}, et la valeur renvoyée par
\tm{(buffer-file-name)} est \tm{nil}.

D'autre part, pendant que je vous écris ce document, la valeur
renvoyée par l'évaluation de \tm{(buffer-name)} est
\tm{''introduction.texinfo''}, et la valeur renvoyée par l'évaluation
de \tm{(buffer-file-name)} est
\tm{''/gnu/work/intro/introduction.texinfo''}.

Le premier est le nom du tampon et le dernier est le nom du
fichier. Dans Info, le nom du tampon est \tm{''*info*''}. Info ne
pointe vers aucun fichier, donc le résultat de l'évaluation de
\tm{(buffer-file-name)} est \tm{nil}. Le symbole \tm{nil} vient du mot
Latin pour 'rien'; dans ce cas, cela signifie que le tampon n'est
associé à aucun fichier. (En Lisp, \tm{nil} est aussi utilisé pour
signifié 'faux' et est un synonyme pour la liste vide, \tm{()}.)

Lorsque j'écris, le nom de mon tampon est
\tm{''introduction.texinfo''}. Le nom du fichier vers lequel il pointe
est \tm{''/gnu/work/intro/introduction.texinfo''}.

(Dans ces expressions, les parenthèses disent à l'interprète Lisp de
traiter \tm{buffer-name} et \tm{buffer-file-name} comme des fonctions;
sans les parenthèses, l'interprète serait tenté d'évaluer les symboles
comme des variables. Voir Section \cfchs{1}{7} ''Variables'', page
\cfchsg{1}{7}.) 

En dépit de la distinction entre les fichiers et les tampons, vous
trouverez souvent que les gens se réfèrent à un fichier quand ils
veulent dire tampon et vice-versa. En effet, la plupart des gens
disent : <<Je suis dans l'édition d'un fichier>>, plutôt que de dire :
<<Je suis dans l'édition d'un tampon que je vais bientôt enregistrer
dans un fichier.>> Il est presque toujours clair à partir du contexte
de deviner ce que les gens veulent dire. Lorsque vous traitez avec des
programmes informatiques, cependant, il est important de garder à
l'esprit la distinction, puisque l'ordinateur n'est pas aussi
intelligent qu'une personne (pour deviner).

Le mot <<tampon>>, en passant, vient de la signification du mot
coussin qui amortit la force d'une collision. Dans les premiers
ordinateurs, un tampon amortissait l'interaction entre les fichiers et
l'unité centrale de traitement de l'ordinateur. Les tambours ou bandes
qui tenaient un fichier et l'unité centrale de traitement étaient des
pièces d'équipement qui étaient très différentes les unes des autres,
à travailler à leurs propres vitesses, par à-coups. Le tampon a rendu
possible pour eux de travailler ensemble efficacement. Finalement, le
tampon est passé du statut d'intermédiaire, un lieu de détention
temporaire, à celui de lieu où le travail se fait. Cette
transformation est un peu comme celle d'un petit port qui a grandi
dans une grande ville : au début il était simplement le lieu où des
marchandises étaint entreposées temporairement avant d'être chargées
sur les bateaux; puis il est devenu un centre commercial et culturel. 

Pas tous les tampons sont associés à des fichiers. Par exemple, un
tampon \tm{*scratch*} ne visite aucun fichier. De même, un tampon
\tm{*Help*} n'est associé à aucun fichier. 

Dans les anciens temps, lorsque vous aviez manqué un fichier
\tm{~/.emacs} et commencé une session Emacs en tapant la commande
\tm{emacs} seule, sans nommer aucun fichier, Emacs démarrait avec le
tampon \tm{*scratch*}. De nos jours, vous verrez un écran de
démarrage. Vous pouvez suivre l'une des commandes proposées sur
l'écran de démarrage, visiter un fichier ou appuyer sur la barre
d'espace pour atteindre le tampon \tm{*scratch*}.

Si vous passez au tampon \tm{*scratch*}, tapez \tm{(buffer-name)},
positionnez le curseur à la suite, et tapez \tm{C-x C-e} pour évaluer
l'expression. Le nom \tm{''*scratch*''} sera renvoyé et apparaîtra
dans la zone d'écho. \tm{''*scratch*''} est le nom du tampon. Quand
vous tapez \tm{(buffer-file-name)} dans le tampon \tm{*scratch*} et
évaluez ça, \tm{nil} apparaît dans la zone d'écho, juste comme ça le
fait lorsque vous évaluez \tm{(buffer-file-name)} dans Info.

Incidemment, si vous êtes dans le tampon \tm{*scratch*} et souhaitez
que la valeur renvoyée par une expression apparaisse dans le tampon
\tm{*scratch*} lui-même au lieu de la zone d'écho, tapez \tm{C-u C-x
  C-e} au lieu de \tm{C-x C-e}. Cela provoque l'apparition de la
valeur renvoyée après l'expression. Le tampon ressemblera à ça :

\tm{(buffer-name)''*scratch*''}

Vous ne pouvez pas faire ça dans Info puisque Info est en lecture
seule et ne vous autorisera pas à changer le contenu de son
tampon. Mais vous pouvez faire cela dans n'importe quel tampon que
vous éditez; et quand vous écrivez du code ou une documentation (comme
ce livre), c'est fonction sont très utiles.

