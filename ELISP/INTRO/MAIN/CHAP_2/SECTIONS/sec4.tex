\section{Taille du tampon et localisation du point}\etchs{2}{4}

Enfin, regardons plusieurs fonctions assez simples, \tm{buffer-size,
  point-min} et \tm{point-max}. Celles-ci donnent des informations sur
la taille d'un tampon et l'emplacement du point en son sein.

La fonction \tm{buffer-size} vous indique la taille du tampon courant;
autrement dit, la fonction renvoie un compte du nombre de caractères
dans le tampon.

\tm{(buffer-size)}

Vous pouvez évaluer cela de façon habituelle, en positionnant le
curseur après l'expression et en tapant \tm{C-x C-e}.

Dans Emacs, la position actuelle du curseur est appelée le
\textit{point}. L'expression \tm{(point)} renvoie un nombre qui vous
indique où se trouve le curseur comme un comptage de nombre de
caractères à partir du début du tampon jusqu'au point.

Vous pouvez voir le nombre de caractères pour le point dans ce tampon
en évaluant l'expression suivante de la manière usuelle :

\tm{(point)}

Pendant que j'écris ceci, la valeur du point est 65724. La fonction
\tm{point} est utilisée fréquemment dans certains exemples plus loin
dans ce livre. 

La valeur du point dépend, bien sûr, de son emplacement dans le
tampon. Si vous évaluez le point à cet endroit, le nombre sera plus
grand :

\tm{(point)}

Pour moi, la valeur du point à cet endroit est 66043, ce qui signifie
qu'il y a 319 caractères (espaces compris) entre les deux
expressions. (Sans doute, vous verrez des nombres différents, puisque
j'aurais édité depuis ma première évaluation du point.)

La fonction \tm{point-min} est quelque peu semblable à un \tm{point}, mais
elle renvoie la valeur de la valeur minimale admissible du point dans
le tampon courant. C'est le nombre 1 sauf si le \textit{réduction} est
en vigueur. (La réduction est un mécanisme par lequel vous pouvez vous
limiter, ou un programme, d'opérations sur seulement une partie d'un
tampon. Voir le chapitre \cfch{6} ``Réduction et élargissement'', page
\cfchg{6}.) De même, la fonction \tm{point-max} renvoie la valeur de
la valeur maximale admissible du point dans le tampon courant.
