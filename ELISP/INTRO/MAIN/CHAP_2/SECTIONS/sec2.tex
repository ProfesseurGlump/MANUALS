\section{Obtenir un tampon}\etchs{2}{2}

La fonction \tm{buffer-name} renvoie le \textit{nom} du tampon; pour
obtenir le tampon \textit{lui-même}, une fonction différente est
nécessaire : la fonction \tm{current-buffer}. Si vous utilisez cette
fonction dans un code, ce que vous obtenez est le tampon lui-même.

Un nom d'objet et l'objet ou l'entité à laquelle se réfère le nom sont
différents entre eux. Vous n'êtes pas votre nom. Vous êtes une
personne à laquelle les autres font référence par votre nom. Si vous
demandez à parler à George et quelqu'un vous donne une carte avec les
lettres \tm{'G', 'e', 'o', 'r', 'g', 'e'} écrites dessus, vous serez
peut être amusé, mais vous ne serez pas satisfait. Vous ne voulez pas
parler au nom, mais à la personne à laquelle le nom fait
référence. Un tampon est similaire : le nom du tampon scratch est
\tm{*scratch*}, mais le nom n'est pas le tampon. Pour obtenir le
tampon lui-même vous avez besoin d'utiliser une fonction comme
\tm{current-buffer}.

Cependant, il y a une légère complication : si vous évaluez
\tm{current-buffer} dans une expression, comme nous le ferons ici, ce
que vous voyez est une représentation imprimée du nom du tampon sans
le contenu du tampon. Emacs travaille de cette façon pour deux raisons
: la tampon peut être des milliers de lignes de long---trop long pour
être affiché; et, un autre tampon peut avoir le même contenu mais un
nom différent, et il est important de pouvoir les distinguer.

Voici une expression contenant la fonction :

\tm{(current-buffer)}

Si vous évaluez cette expression dans Info dans Emacs de la façon
habituelle, \tm{\#<buffer *info*>} apparaîtra dans la zone d'écho. Le
format spécial indique que le tampon lui-même est renvoyé, plutôt que
juste son nom.

Incidemment, pendant que vous pouvez taper un nombre ou symbole dans
un programme, vous ne pouvez pas faire ça avec une représentation
imprimée d'un tampon : la seule façon d'obtenir un tampon lui-même
c'est avec une fonction telle que \tm{current-buffer}.

Une fonction liée est \tm{other-buffer}. Elle renvoie le tampon le
plus récemment sélectionné autre que celui sur lequel vous êtes en ce
moment, pas une représentation imprimée de son nom. Si vous avez
récemment changé depuis le tampon \tm{*scratch*}, \tm{other-buffer}
renverra ce tampon. 

Vous pouvez voir ça en évaluant l'expression :

\tm{(other-buffer)}

Vous devriez voir \tm{\#<buffer *scratch*>} apparaître dans la zone
d'écho, ou le nom de n'importe quel autre tampon que vous avez changé
récemment.\footnote{En fait, par défaut, si le tampon à partir duquel
  vous venez de changer est visible pour vous dans une autre fenêtre,
  un autre tampon sera choisi, le plus récent que vous ne pouvez pas
  voir; ceci est une subtilité que j'oublie souvent.}
