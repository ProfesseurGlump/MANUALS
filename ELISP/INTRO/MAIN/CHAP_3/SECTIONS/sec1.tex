\section{La forme spéciale \texttt{defun}}\etchs{3}{1}

En Lisp, un symbole tel que \tm{mark-whole-buffer} a un code qui lui
est attaché et qui dit à l'ordinateur que faire lorsque la fonction
est appélée. Ce code est appelé la \textit{définition de fonction} et est
créé par l'évaluation d'une expression Lisp qui commence par le
symbole \tm{defun} ( qui est une abréviation pour \textit{define
  function}\footnote{définition de fonction}). Parce que \tm{defun}
n'évalue pas ses arguments de la manière habituelle, elle est appelée
une \textit{forme spéciale}.

Dans les sections suivantes, nous allons examiner les définitions de
fonctions à partir du code source Emacs, comme
\tm{mark-whole-buffer}. Dans cette section, nous allons décrire une
définition de fonction simple de sorte que vous puissiez voir à quoi
ça ressemble. Cette définition de fonction utilise l'arithmétique, car
cela en fait un exemple simple. Certaines personnes détestent les
exemples utilisant l'arithmétique; toutefois, si vous êtes une telle
personne, ne désespérez pas. Pratiquement tout le code, que nous
allons étudier par la suite de cette introdcution implique
l'arithmétique ou les mathématiques. Les exemples concernent
essentiellement du texte d'une manière ou d'une autre.

Une définition de fonction a jusqu'à cinq parties suivant le mot
\tm{defun} :
\begin{enumerate}
\item Le nom du symbole à laquelle la définition de fonction doit être
  jointe.
\item Une liste des arguments qui seront passés à la fonction. Si
  aucun argument ne seront passés à la fonction, ceci est une liste
  vide, \tm{()}.
\item Une documentation décrivant la fonction. (Techniquement
  facultatif, mais fortement recommandé.)
\item \'Eventuellement, une expression pour rendre la fonction
  interactive de sorte que vous pouvez l'utiliser en tapant \tm{M-x},
  puis le nom de la fonction; ou en tapant sur une touche ou une
  combinaison de touches appropriées.
\item Le code qui indique à l'ordinateur ce qu'il doit faire: le
  \textit{corps} de la fonction.
\end{enumerate}

Il est utile de penser aux cinq parties d'une définition de fonction
comme étant organisé dans un modèle, avec des fentes pour chaque
partie:

\tm{(defun \textit{function-name} (\textit{arguments} \dots)}

\tm{``\textit{optional-documentation}\dots''}

\tm{(interactive \textit{argument-passing-info})} ; optional

\tm{\textit{body}\dots)}

\`A titre d'exemple, voici le code pour une fonction qui multiplie son
argument par 7. (Cet exemple n'est pas interactif. Voir la section
\cfchs{3}{3} ``Faire une fonction interactive'', page \cfchsg{3}{3},
pour cette information.)

\tm{(defun muliply-by-seven (number)}

\tm{``Multiply NUMBER by seven.''}

\tm{(* 7 number))}

Cette définition commence par une parenthèse et le symbole \tm{defun},
suivi du nom de la fonction.

Le nom de la fonction est suivie par une liste qui contient les
arguments qui seront passés à la fonction. Cette liste est appelée la
\textit{liste des arguments}. Dans cet exemple, la liste ne comporte
qu'un seul élément, le symbole, \tm{number}. Lorsque la fonction est
utilisée, le symbole sera lié à la valeur qui est utilisée comme
argument de la fonction. 

Au lieu de choisir le nombre de mot pour le nom de l'argument,
j'aurais pu choisir un autre nom. Par exemple, je pourrais avoir
choisi le mot \tm{multicplicand}. J'ai pris le mot <<nombre>> parce
qu'il dit bien quel type de valeur est destinée à cet emplacement;
mais je pourrais tout aussi bien avoir choisi le mot <<multiplicande>>
pour indiquer le rôle que la valeur placée à cet endroit va jouer dans
le fonctionnement de la fonction. J'aurais pu l'appeler \tm{foogle},
mais cela aurait été un mauvais choix, car il ne renseignerait pas sur
sa signification. Le choix du nom revient au programmeur et doit être
fait pour rendre le sens de la fonction le plus clair possible.

En effet, vous pouvez choisir le nom que vous souhaitez pour un
symbole dans une liste d'arguments, même le nom d'un symbole utilisé
dans une autre fonction : le nom que vous utilisez dans une liste
d'arguments est réservé à cette définition particulière. Dans cette
définition, le nom se réfère à une entité différente de n'importe
qu'elle utilisation du même nom en dehors de cette définition de
fonction. Supposons que vous avez un surnom <<shorty>> dans votre
famille; lorsque un membre de votre famille se réfère à <<shorty>>, il
parle de vous. Mais en dehors de votre famille, dans un film, par
exemple, le nom <<shorty>> se réfère à quelqu'un d'autre. Parce qu'un
nom dans une liste d'argument est réservé à cette définition de
fonction, vous pouvez changer la valeur d'un tel symbole à l'intérieur
du corps d'une fonction sans changer sa valeur en dehors de la
fonction. L'effet est similaire à celui produit par une expression
\tm{let}. (Voir Section \cfchs{3}{6} ``let'', page \cfchsg{3}{6}.)

La liste des arguments est suivie par la documentation qui décrit la
fonction. Ceci est ce que vous voyez lorsque vous tapez \tm{C-h f} et
le nom d'une fonction. Incidemment, lorsque vous écrivez une
documentation de ce genre, vous devriez faire la première ligne une
phrase complète depuis quelques commandes, comme \tm{apropos},
imprimer uniquement la première ligne d'une documentation
multi-ligne. En outre, vous ne devriez pas indenter la deuxième ligne
d'une documentation, si vous en avez une, parce que ça aurait l'air
bizarre quand vous utilisez \tm{C-h f (describe-function)}. La
documentation est facultative, mais elle est si utile, elle devrait
être inclus dans presque toutes les fonctions que vous écrivez.

La troisième ligne de l'exemple est constituée du corps de la
définition de fonction. (La plupart des définitions de fonctions, bien
sûr, sont plus longues que celle-là.) Dans cette fonction, le corps
est la liste, \tm{(* 7 number)}, qui dit de multiplier la valeur du
nombre de 7. (Dans Emacs Lisp, \tm{*} est la fonction de
multiplication, comme \tm{+} est la fonction d'addition.)

Lorsque vous utilisez la fonction \tm{multiply-by-seven}, l'argument
\tm{number} évalue le nombre que vous avez utilisé. Voici un exemple
qui montre comment \tm{multiply-by-seven} est utilisé; mais n'essayez
pas encore de l'évaluer!

\tm{(multiply-by-seven 3)}

Le symbole \tm{number}, spécifié dans la définition de fonction de la
prochaine section, est donné ou <<lié à>> la valeur 3 dans cet exemple
d'utilisation. Notez que bien que \tm{number} était entre parenthèses
à l'intérieur dans la définition de fonction, l'argument passé à la
fonction \tm{(multiply-by-seven 3)} n'est pas entre parenthèses. Les
parenthèses sont écrites dans la définition de fonction afin que
l'ordinateur comprenne où se termine la liste d'arguments et le reste
de la définition commence. 

Si vous évaluez cet exemple, vous êtes susceptible d'obtenir un
message d'erreur. (Allez-y, essayez!) C'est parce que nous avons écrit
la définition de fonction, mais pas encore dit à l'ordinateur à propos
de la définition---nous n'avons pas encore installé (ou <<chargé>>) la
définition de fonction dans Emacs. L'installation d'une fonction est
le processus qui dit à l'interprète Lisp qu'elle est la définition de
fonction. L'installation est décrite dans la section suivante. 