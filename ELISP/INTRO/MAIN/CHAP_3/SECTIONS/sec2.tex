\section{Installer une définition de fonction}\etchs{3}{2}

Si vous lisez ceci à l'intérieur d'Info dans Emacs, vous pouvez
essayer la fonction \tm{multiply-by-seven} en évaluant d'abord la
définition de fonction puis en évaluant \tm{(multiply-by-seven
  3)}. Une copie de la définition de la fonction suivra. Placez le
curseur après la dernière parenthèse de la définition de fonction et
tapez \tm{C-x C-e}. Lorsque vous faites cela, \tm{multiply-by-seven}
apparaîtra dans la zone d'écho. (Ce que cela signifie c'est que
lorsqu'une définition de fonction est évaluée, la valeur renvoyée est
le nom de la fonction définie.) En même temps, cette action installe
la définition de fonction. 

\tm{(defun multiply-by-seven (number)}

\tm{   ``Multiply NUMBER by seven.''}

\tm{ (* 7 number))}

En évaluant ce \tm{defun}, vous venez d'installer
\tm{multiply-by-seven} dans Emacs. La fonction fait désormais autant
partie de Emacs que \tm{forward-word} ou toute autre fonction
d'édition que vous utilisez. (\tm{multiply-by-seven} restera installé
jusqu'à ce que vous quittiez Emacs. Pour recharger le code
automatiquement chaque fois que vous démarrez Emacs, voir Section
\cfchs{3}{5} ``Installation permanente de code'', page \cfchs{3}{5}.)

Vous pouvez voir l'effet de l'installation de \tm{multiply-by-seven}
en évaluant l'échantillon suivant. Placez le curseur après
l'expression suivante et tapez \tm{C-x C-e}. Le nombre 21 apparaîtra
dans la zone d'écho.

\tm{(multiply-by-seven 3)}

Si vous le souhaitez, vous pouvez lire la documentation de la fonction
en tapant \tm{C-h f (describe-function)} puis le nom de la fonction,
\tm{multiply-by-seven}. Lorsque vous faîtes cela, une fenêtre
\tm{*Help*} apparaîtra sur votre écran qui dit :

\tm{multiply-by-seven is a Lisp function.}

\tm{(multiply-by-seven NUMBER)}

\tm{Multiply NUMBER by seven.}

(Pour revenir à une fenêtre simple sur votre écran, tapez \tm{C-x 1}.)
 
\subsection{Changer une définition de fonction}\etchss{3}{2}{1}

Si vous souhaitez modifier le code dans \tm{multiply-by-seven}, alors
réécrivez-le. Pour installer la nouvelle version à la place de
l'ancienne, évaluez à nouveau la définition de fonction. Ceci est la
façon dont vous modifiez le code dans Emacs. C'est très simple.

\`A titre d'exemple, vous pouvez modifier la fonction
\tm{multiply-by-seven} de sorte qu'elle ajoute le nombre à lui-même
sept fois au lieu de multiplier le nombre par sept. Cela produit le
même résultat, mais par un chemin différent. En même temps, nous
ajouterons un commentaire à ce code; un commentaire est un texte que
l'interprète Lisp ignore, mais qu'un lecteur humain peut trouver utile
ou instructif. Le commentaire est <<seconde version>>.

\tm{(defun multiply-by-seven (number)   ;} Second version.

\tm{ ``Multiply NUMBER by seven.''}

\tm{ (+ number number number number number number number))}

Le commentaire suit un point-virgule, '\tm{;}'. En Lisp, tout sur une
ligne qui suit un point-virgule est un commentaire. La fin de ligne
est la fin du commentaire. Pour étirer un commentaire sur deux ou
plusieurs lignes, chaque ligne doit commencer par un point-virgule.

Voir Section 16.3 ``Commencer un fichier \tm{.emacs}'', page 187, et
Section ``Commentaires'' dans \textit{The GNU Emacs Lisp Reference
  Manual}, pour plus de renseignements à propos des commentaires.

Vous pouvez installer cette version de la fonction
\tm{multiply-by-seven} en l'évaluant de la même manière que vous avez
évalué la première fonction : placez le curseur après la dernière
parenthèse et tapez \tm{C-x C-e}.

En résumé, voici comment vous écrivez du code en Emacs Lisp: vous
écrivez une fonction; vous l'installez; vous la testez; et ensuite
vous faites des corrections ou améliorations et vous l'installez à
nouveau. 