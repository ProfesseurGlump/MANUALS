\section{Rendre une fonction intéractive}\etchs{3}{3}

Vous créez une fonction interactive en plaçant une liste qui commence
par la forme spéciale \tm{interactive} immédiatement après la
documentation. Un utilisateur peut invoquer une fonction interactive
en tapant \tm{M-x} et ensuite le nom de la fonction; ou en tapant la
combinaison de touches (raccourci clavier) à laquelle cette fonction
est liée, par exemple, en tapant \tm{C-n} pour \tm{next-line} ou
\tm{C-x h} pour \tm{mark-whole-buffer}. 

Curieusement, lorsque vous appelez une fonction interactive de
façon interactive, la valeur renvoyée n'est pas automatiquement
affichée dans la zone d'écho. Cela est dû au fait que vous appelez
souvent une fonction interactive pour ses effets secondaires, tels que
se déplacer vers l'avant par mot ou par ligne, et non pour la valeur
renvoyée. Si la valeur renvoyée était affichée dans la zone d'écho
chaque fois que vous tapez un raccourci, cela serait très distrayant.

Tant l'utilisation de la forme spéciale \tm{interactive} que la façon
d'afficher une valeur dans la zone d'écho peuvent être illustrées par
la création d'une version interactive de \tm{multiply-by-seven}.

Voici le code :

\tm{(defun multiply-by-seven (number)}   ; Interactive version.

\tm{ ``Multiply NUMBER by seven.''}

\tm{ (interactive ``p'')}

\tm{ (message ``The result is \%d'' (* 7 number)))}

Vous pouvez installer ce code en plaçant votre curseur après (la
dernière parenthèse) et en tapant \tm{C-x C-e}. Le nom de la fonction
apparaîtra dans votre zone d'écho. Ensuite, vous pouvez utiliser ce
code en tapant \tm{C-u} et un nombre et ensuite en tapant \tm{M-x
  multiply-by-seven} et appuer sur \RET. La phrase \tm{'The result is
  \dots'} suivie du produit appraîtra dans la zone d'écho.

De façon plus générale, vous invoquez une fonction de ce genre de
l'une des deux manières suivantes :
\begin{enumerate}
\item En tapant un argument préfixé qui contient le nombre qui être
  passé, et ensuite en tapant \tm{M-x} et le nom de la fonction, comme
  avec \tm{C-u 3 M-x forward-sentence}; ou,
\item En tapant la touche ou raccourci auquel la fonction est liée,
  comme avec \tm{C-u 3 M-e}.
\end{enumerate}

Les deux exemples qui viennent d'être mentionnés fonctionnent
identiquement pour déplacer le point vers l'avant de trois
phrases. (Puisque \tm{multiply-by-seven} n'est liée à aucune touche
(ni raccourci), elle ne pouvait pas être utilisée comme un exemple de
raccourci de liaison.)

(Voir Section \cfchs{16}{7} ``Quelques raccourcis'', page
\cfchsg{16}{7}, pour apprendre comment lier une commande à une
combinaison de touches.)

Un argument préfixé est passé à une fonction interactive en tapant la
touche \tm{META} suivi d'un numéro, par exemple, \tm{M-3 M-e}, ou en
tapant \tm{C-u} et ensuite le nombre par exemple, \tm{C-u 3 M-e} (si
vous tapez \tm{C-u} sans aucun nombre, par défaut ce sera 4). 


\subsection{Une intéractive
  \texttt{multiply-by-seven}}\etchss{3}{3}{1}
Regardons l'utilisation de la forme spéciale \tm{interactive} et
ensuite la fonction \tm{message} dans la version interactive de
\tm{multiply-by-seven}. Vous vous souviendrez que la définition de
fonction ressemble à cela :

 \tm{(defun multiply-by-seven (number)}   ; Interactive version.

\tm{ ``Multiply NUMBER by seven.''}

\tm{ (interactive ``p'')}

\tm{ (message ``The result is \%d'' (* 7 number)))}

Dans cette fonction, l'expression, \tm{(interactive ``p'')}, est une
liste à deux éléments. Le \tm{``p''} dit à Emacs de passer l'argument
préfixé à la fonction et d'utiliser sa valeur pour l'argument de la
fonction. 

L'argument sera un nombre. Cela signifie que le symbole \tm{number}
sera lié au nombre dans la ligne :

\tm{ (message ``The result is \%d'' (* 7 number))}

Par exemple, si votre argument préfixé est 5, l'interprète Lisp
évaluera la ligne comme si elle était :

\tm{ (message ``The result is \%d'' (* 7 5))}

(Si vous lisez ceci dans GNU Emacs, vous pouvez évaluer cette
expression vous-même.) Tout d'abord, l'interprète évaluera la liste
intérieure, qui est \tm{(* 7 5)}. Cela renvoie une valeur qui est
35. Ensuite, il évaluera la liste extérieure, en passant les valeurs
des deuxième éléments et les suivants de la liste à la fonction
\tm{message}.

Comme nous l'avons vu, \tm{message} est une fonction Emacs Lisp
spécialement conçue pour l'envoi de message d'une ligne pour
l'utilisateur. (Voir Section \cfchss{1}{8}{5} <<La fonction
\tm{message}>>, page \cfchss{1}{8}{5}.) En résumé, la fonction
\tm{message} imprime son premier argument dans la zone d'écho tel
quel, sauf pour les occurrences de \tm{'\%d'} ou \tm{'\%s'} (et les
diverses autres \%-séquences que nous n'avons pas encore
mentionnées). Quand elle voit une séquence de contrôle, la fonction
recherche au deuxième ou arguments suivants et imprime la valeur de
l'argument à l'emplacement dans la chaîne où la séquence de contrôle
est située. 

Dans la fonction interactive \tm{multiply-by-seven}, la chaîne de
contrôle est \tm{'\%d'}, qui requiert un nombre, et la valeur renvoyée
par l'évaluation de \tm{(* 7 5)} est le nombre 35. En conséquence, le
nombre 35 est imprimé à la place de \tm{'\%d'} et le message est
\tm{'The result is 35'}.

(Notez que lorsque vous appelez la fonction \tm{multiply-by-seven}, le
message est imprimé sans guillemets, mais quand vous appelez
\tm{message}, le texte est imprimé avec guillemets. C'est parce que la
valeur renvoyée par \tm{message} est ce qui appraît dans la zone
d'écho quand vous évaluez une expression dont le premier élément est
\tm{message}; mais lorsqu'il est incorporé dans une fonction,
\tm{message} imprime le texte comme un effet secondaire sans guillemets.)