\chapter{Comment écrire des définitions de fonctions}\etch{3}

Lorsque l'interprète Lisp évalue une liste, il regarde si le premier
symbole sur la liste a une définition de fonction attachée à lui; ou,
en d'autres termes, si le symbole pointe vers une définition de
fonction. Dans le cas contraire, l'ordinateur exécute les instructions
dans la définition. Un symbole qui a une définition de fonction est
appelé, tout simplement, une fonction (même si, à proprement parler,
la définition est la fonction et le symbole se réfère à elle.)

Toutes les fonctions sont définies en termes d'autres fonctions, à
l'exception de quelques fonctions primitives qui sont écrites dans le
langage de programmation C. Lorsque vous écrivez les définitions de
fonctions, vous les écrirez en Emacs Lisp et utiliserez d'autres
fonctions comme blocs de construction. Certaines des fonctions que
vous utiliserez seront elles-mêmes écrites en Emacs Lisp (peut être
par vous) et certaines seront des primitives écrites en C. Les
fonctions primitives sont utilisées exactement comme celles écrites en
Emacs Lisp et se comportent comme elles. Elles sont écrites en C afin
que nous puissions facilement lancer \gem sur tout ordinateur d'une
puissance suffisante pour faire fonctionner C.

Permettez-moi de souligner de nouveau ceci: quand vous écrivez du code
en Emacs Lisp, vous ne distinguez pas celles écrites en C de celles
écrites en Emacs Lisp. La différence est sans importance. Je mentionne
la distinction seulement parce qu'il est intéressant de le savoir. En
effet, à moins que vous ne l'étudiez, vous ne saurez pas si une
fonction déjà écrite est écrite en Emacs Lisp ou en C.

\section{Noms des tampons}\etchs{2}{1}

Les deux fonctions, \tm{buffer-name} et \tm{buffer-file-name},
montrent la différence entre un fichier et un tampon. Lorsque vous
évaluez l'expression suivante, \tm{(buffer-name)}, le nom du tampon
apparaît dans la zone d'écho. Quand vous évaluez
\tm{(buffer-file-name)}, le nom du fichier auquel se réfère le tampon
est affiché dans la zone d'écho. Habituellement, le nom renvoyé par
\tm{(buffer-name)}  est le même que le nom de fichier auquel il fait
référence, et le nom renvoyé par \tm{(buffer-file-name)} est le nom de
chemin complet du fichier. 

Un fichier et un tampon sont deux entités différentes. Un fichier est
une information enregistrée de façon permanente dans l'ordinateur (à
moins que vous ne l'effaciez). Un tampon, d'un autre côté, est une
information à l'intérieur d'Emacs qui va disparaître à la fin de la
session (ou quand vous <<tuez>> le tampon). Habituellement, un tampon
contient des informations que vous avez copié à partir d'un fichier;
nous disons que le tampon \textit{visite} ce fichier. Cette copie est
ce que vous travaillez et modifiez. Les changements du tampon ne
change pas le fichier, jusqu'à ce que vous enregistrez le
tampon. Lorsque vous sauvegardez le tampon, le tampon est copié dans
le fichier et est donc sauvegardé de façon permanente.

Si vous lisez ceici dans Info à l'intérieur de \gem , vous pouvez
évaluer chacune des expressions suivantes en positionnant le curseur
après et en tapant \tm{C-x C-e}. 

\tm{(buffer-name)}

\tm{(buffer-file-name)}

Quand je fais cela dans Info, la valeur renvoyée par l'évaluation de
\tm{(buffer-name)} est \tm{''*info*''}, et la valeur renvoyée par
\tm{(buffer-file-name)} est \tm{nil}.

D'autre part, pendant que je vous écris ce document, la valeur
renvoyée par l'évaluation de \tm{(buffer-name)} est
\tm{''introduction.texinfo''}, et la valeur renvoyée par l'évaluation
de \tm{(buffer-file-name)} est
\tm{''/gnu/work/intro/introduction.texinfo''}.

Le premier est le nom du tampon et le dernier est le nom du
fichier. Dans Info, le nom du tampon est \tm{''*info*''}. Info ne
pointe vers aucun fichier, donc le résultat de l'évaluation de
\tm{(buffer-file-name)} est \tm{nil}. Le symbole \tm{nil} vient du mot
Latin pour 'rien'; dans ce cas, cela signifie que le tampon n'est
associé à aucun fichier. (En Lisp, \tm{nil} est aussi utilisé pour
signifié 'faux' et est un synonyme pour la liste vide, \tm{()}.)

Lorsque j'écris, le nom de mon tampon est
\tm{''introduction.texinfo''}. Le nom du fichier vers lequel il pointe
est \tm{''/gnu/work/intro/introduction.texinfo''}.

(Dans ces expressions, les parenthèses disent à l'interprète Lisp de
traiter \tm{buffer-name} et \tm{buffer-file-name} comme des fonctions;
sans les parenthèses, l'interprète serait tenté d'évaluer les symboles
comme des variables. Voir Section \cfchs{1}{7} ''Variables'', page
\cfchsg{1}{7}.) 

En dépit de la distinction entre les fichiers et les tampons, vous
trouverez souvent que les gens se réfèrent à un fichier quand ils
veulent dire tampon et vice-versa. En effet, la plupart des gens
disent : <<Je suis dans l'édition d'un fichier>>, plutôt que de dire :
<<Je suis dans l'édition d'un tampon que je vais bientôt enregistrer
dans un fichier.>> Il est presque toujours clair à partir du contexte
de deviner ce que les gens veulent dire. Lorsque vous traitez avec des
programmes informatiques, cependant, il est important de garder à
l'esprit la distinction, puisque l'ordinateur n'est pas aussi
intelligent qu'une personne (pour deviner).

Le mot <<tampon>>, en passant, vient de la signification du mot
coussin qui amortit la force d'une collision. Dans les premiers
ordinateurs, un tampon amortissait l'interaction entre les fichiers et
l'unité centrale de traitement de l'ordinateur. Les tambours ou bandes
qui tenaient un fichier et l'unité centrale de traitement étaient des
pièces d'équipement qui étaient très différentes les unes des autres,
à travailler à leurs propres vitesses, par à-coups. Le tampon a rendu
possible pour eux de travailler ensemble efficacement. Finalement, le
tampon est passé du statut d'intermédiaire, un lieu de détention
temporaire, à celui de lieu où le travail se fait. Cette
transformation est un peu comme celle d'un petit port qui a grandi
dans une grande ville : au début il était simplement le lieu où des
marchandises étaint entreposées temporairement avant d'être chargées
sur les bateaux; puis il est devenu un centre commercial et culturel. 

Pas tous les tampons sont associés à des fichiers. Par exemple, un
tampon \tm{*scratch*} ne visite aucun fichier. De même, un tampon
\tm{*Help*} n'est associé à aucun fichier. 

Dans les anciens temps, lorsque vous aviez manqué un fichier
\tm{~/.emacs} et commencé une session Emacs en tapant la commande
\tm{emacs} seule, sans nommer aucun fichier, Emacs démarrait avec le
tampon \tm{*scratch*}. De nos jours, vous verrez un écran de
démarrage. Vous pouvez suivre l'une des commandes proposées sur
l'écran de démarrage, visiter un fichier ou appuyer sur la barre
d'espace pour atteindre le tampon \tm{*scratch*}.

Si vous passez au tampon \tm{*scratch*}, tapez \tm{(buffer-name)},
positionnez le curseur à la suite, et tapez \tm{C-x C-e} pour évaluer
l'expression. Le nom \tm{''*scratch*''} sera renvoyé et apparaîtra
dans la zone d'écho. \tm{''*scratch*''} est le nom du tampon. Quand
vous tapez \tm{(buffer-file-name)} dans le tampon \tm{*scratch*} et
évaluez ça, \tm{nil} apparaît dans la zone d'écho, juste comme ça le
fait lorsque vous évaluez \tm{(buffer-file-name)} dans Info.

Incidemment, si vous êtes dans le tampon \tm{*scratch*} et souhaitez
que la valeur renvoyée par une expression apparaisse dans le tampon
\tm{*scratch*} lui-même au lieu de la zone d'écho, tapez \tm{C-u C-x
  C-e} au lieu de \tm{C-x C-e}. Cela provoque l'apparition de la
valeur renvoyée après l'expression. Le tampon ressemblera à ça :

\tm{(buffer-name)''*scratch*''}

Vous ne pouvez pas faire ça dans Info puisque Info est en lecture
seule et ne vous autorisera pas à changer le contenu de son
tampon. Mais vous pouvez faire cela dans n'importe quel tampon que
vous éditez; et quand vous écrivez du code ou une documentation (comme
ce livre), c'est fonction sont très utiles.



\section{Obtenir un tampon}\etchs{2}{2}

La fonction \tm{buffer-name} renvoie le \textit{nom} du tampon; pour
obtenir le tampon \textit{lui-même}, une fonction différente est
nécessaire : la fonction \tm{current-buffer}. Si vous utilisez cette
fonction dans un code, ce que vous obtenez est le tampon lui-même.

Un nom d'objet et l'objet ou l'entité à laquelle se réfère le nom sont
différents entre eux. Vous n'êtes pas votre nom. Vous êtes une
personne à laquelle les autres font référence par votre nom. Si vous
demandez à parler à George et quelqu'un vous donne une carte avec les
lettres \tm{'G', 'e', 'o', 'r', 'g', 'e'} écrites dessus, vous serez
peut être amusé, mais vous ne serez pas satisfait. Vous ne voulez pas
parler au nom, mais à la personne à laquelle le nom fait
référence. Un tampon est similaire : le nom du tampon scratch est
\tm{*scratch*}, mais le nom n'est pas le tampon. Pour obtenir le
tampon lui-même vous avez besoin d'utiliser une fonction comme
\tm{current-buffer}.

Cependant, il y a une légère complication : si vous évaluez
\tm{current-buffer} dans une expression, comme nous le ferons ici, ce
que vous voyez est une représentation imprimée du nom du tampon sans
le contenu du tampon. Emacs travaille de cette façon pour deux raisons
: la tampon peut être des milliers de lignes de long---trop long pour
être affiché; et, un autre tampon peut avoir le même contenu mais un
nom différent, et il est important de pouvoir les distinguer.

Voici une expression contenant la fonction :

\tm{(current-buffer)}

Si vous évaluez cette expression dans Info dans Emacs de la façon
habituelle, \tm{\#<buffer *info*>} apparaîtra dans la zone d'écho. Le
format spécial indique que le tampon lui-même est renvoyé, plutôt que
juste son nom.

Incidemment, pendant que vous pouvez taper un nombre ou symbole dans
un programme, vous ne pouvez pas faire ça avec une représentation
imprimée d'un tampon : la seule façon d'obtenir un tampon lui-même
c'est avec une fonction telle que \tm{current-buffer}.

Une fonction liée est \tm{other-buffer}. Elle renvoie le tampon le
plus récemment sélectionné autre que celui sur lequel vous êtes en ce
moment, pas une représentation imprimée de son nom. Si vous avez
récemment changé depuis le tampon \tm{*scratch*}, \tm{other-buffer}
renverra ce tampon. 

Vous pouvez voir ça en évaluant l'expression :

\tm{(other-buffer)}

Vous devriez voir \tm{\#<buffer *scratch*>} apparaître dans la zone
d'écho, ou le nom de n'importe quel autre tampon que vous avez changé
récemment.\footnote{En fait, par défaut, si le tampon à partir duquel
  vous venez de changer est visible pour vous dans une autre fenêtre,
  un autre tampon sera choisi, le plus récent que vous ne pouvez pas
  voir; ceci est une subtilité que j'oublie souvent.}


\section{Changer de tampon}\etchs{2}{3}

La fonction \tm{other-buffer} fournit en fait un tampon quand il est
utilisé comme un argument pour une fonction qui en exige un. Nous
pouvons voir cela en utilisant \tm{other-buffer} et
\tm{switch-to-buffer} pour passer à un tampon différent.

Mais d'abord, une brève introduction à la fonction
\tm{switch-to-buffer}. Lorsque vous passez du tampon Info à celui de
\tm{*scratch*} pour évaluer \tm{(buffer-name)}, vous avez probablement
tapé \tm{C-x b} puis \tm{*scratch*}\footnote{Ou plutôt, économiser la
  frappe, vous avez probablement tapé \RET si le tampon par défaut
  était \tm{*scratch*}, ou si c'était différent, alors vous avez juste
tapé une partie du nom, telle que \tm{*sc}, pressé la touche \TAB pour
provoquer la complétion, et ensuite tapé \RET} lorsque vous êtiez
invité à saisir le nom du tampon vers lequel vous souhaitiez aller
dans le mini-tampon. Le raccourci, \tm{C-x b}, provoque l'interprète
Lisp pour évaluer la fonction interactive \tm{switch-to-buffer}. Comme
on l'a dit plus tôt, voilà comment Emacs fonctionne : différent
raccourcis appellent ou exécutent différentes fonctions. Par exemple,
\tm{C-f} appelle \tm{forward-char}, \tm{M-e} appelle
\tm{forward-sentence}, et ainsi de suite.

En écrivant \tm{switch-to-buffer}, et en lui donnant un tampon pour
basculer, nous pouvons changer de tampon juste de la même façon que
fait \tm{C-x b} : 

\tm{(switch-to-buffer (other-buffer))}

Le symbole \tm{switch-to-buffer} est le premier élément de la liste,
de sorte que l'interprète Lisp va le traiter une fonction et exécuter
les instructions qui y sont attachées. Mais avant de faire cela,
l'interprète notera que \tm{other-buffer} est entre parenthèses à
l'intérieur et le travail sur ce premier symbole. \tm{other-buffer}
est le premier (et dans ce cas, le seul) élément de cette liste, donc
l'interprète Lisp appelle et exécute la fonction. Cela renvoie un
autre tampon. Ensuite, l'interprète lance \tm{switch-to-buffer}, en
lui passant, comme un argument, l'autre tampon, qui est celui que
Emacs basculera vers. Si vous lisez ceci dans Info, essayez
maintenant. \'Evaluer l'expression. (Pour revenir, tapez \tm{C-x b
  \RET}.)\footnote{Rappelez-vous, cette expression vous déplacera dans
l'autre tampon le plus récent que vous ne pouvez pas voir. Si vous
voulez vraiment aller vers le tampon le plus récent sélectionné, même
si vous pouvez le voir, vous avez besoin d'évaluer l'expression plus
complexe suivante :

\tm{(switch-to-buffer (other-buffer (current-buffer) t))}

Dans ce cas, le premier argument de \tm{other-buffer} lui dit quel
tampon à sauter---celui en court---et le second argument inqique à
\tm{other-buffer} que c'est OK de changer vers un tampon visible. En
utilisation régulière, \tm{switch-to-buffer} vous amène une fenêtre
invisible puisque vous utilisez le plus souvent \tm{C-x o
  (other-window)} pour aller à un autre tampon visible.}

Dans les exemples de programmation dans les sections suivantes de ce
document, vous verrez la fonction \tm{set-buffer} plus souvent que
\tm{switch-to-buffer}. Ceci est dû à une différence entre les
programmes et les humains : les humains ont des yeux et s'attendent à
voir le tampon sur lequel ils travaillent sur leurs terminaux. Cela
est si évident, que cela va presque sans dire. Toutefois, les
programmes n'ont pas d'yeux. Quand un programme fonctionne sur un
tampon, ce tampon n'a pas besoin d'être visible sur l'écran.

\tm{switch-to-buffer} est conçu pour l'homme et fait deux choses
différentes: il change vers le tampon sur lequel l'attention de Emacs
est dirigé; et il commute le tampon affiché dans la fenêtre pour le
nouveau tampon. \tm{set-buffer}, d'autre part, ne fait qu'une chose :
il passe l'attention du programme à un tampon différent. Le tampon de
l'écran reste inchangé (bien sûr, rien ne se passe normalement jusqu'à
ce que les commandes finissent leurs exécutions).

Aussi, nous venons d'introduire un autre terme de jargon, le mot
\textit{appel}. Lorsque vous évaluez une liste dans laquelle le
premier symbole est une fonction, vous appellez cette
fonction. L'utilisation du terme provient de la notion de la fonction
comme une entité qui peut faire quelque chose pour vous si vous
'l'appelez'---juste comme un plombier est une entité qui peut réparer
une fuite si vous l'appelez.

\section{Différentes options pour \texttt{interactive}}\etchs{3}{4}
Dans l'exemple, \tm{multiply-by-seven} utilisait \tm{''p''} comme un
argument pour \tm{interactive}. Cet argument disait à Emacs
d'interpréter votre frappe soit \tm{C-u} suivi d'un nombre soit
\tm{META} suivi d'un nombre comme une commande à passer ce nombre à la
fonction comme son argument. Emacs a plus de vingt caractères
prédéfinis pour l'utilisation avec \tm{interactive}. Dans presque tous
les cas, une de ces options vous permettra de passer la bonne
information interactivement à la fonction. (Voir Section <<Caractère
de code pour \tm{interactive}>> dans \textit{The GNU Emacs Lisp
  Reference Manual}.)
\begin{center}
  Considérez la fonction \tm{zap-to-char}. Son expression interactive
  est

  \tm{(interactive ''p\textbackslash~nZap to char: '')}
\end{center}

La première partie de l'argument de \tm{interactive} est \tm{'p'},
avec ce dont vous êtes déjà familier. Cet argument dit à Emacs
d'interpréter un <<préfixe>>, comme un nombre qui doit être passé à la
fonction. Vous pouvez spécifier un préfixe soit en tapant \tm{C-u}
suivi d'un nombre soit en tapant \tm{META} suivi par un nombre. Le
préfixe est le nombre de caractères spécifiés. Ainsi, si votre préfixe
est trois et le caractère spécifié est \tm{'x'}, alors vous effacerez
tout le texte jusqu'à et y compris la troisième occurence de
\tm{'x'}. Si vous n'initialisez pas un préfixe, alors vous effacerez
tout le texte jusqu'à et y compris le caractère spécifié, mais pas
plus. 

Le \tm{'c'} indique à la fonction le nom du caractère qu'elle doit
supprimer.

Plus formellement, une fonction avec plusieurs arguments peut avoir
des informations passées à chacun des argument en ajoutant des
parties à la chaîne qui suit \tm{interactive}. Lorsque vous faîtes ça,
l'information est passée à chaque argument dans le même ordre qu'elle
est spécifiée dans la liste \tm{interactive}. Dans la chaîne, chaque
partie est séparée de la suivante par un \tm{'\textbackslash'}, qui
est une nouvelle ligne. Par exemple, vous pouvez suivre \tm{'p'} avec
un \tm{'\textbackslash~n'} et un \tm{'cZap to char: '}. Cela incite
Emacs à passer la valeur de l'argument préfixé (s'il y en a un) et le
caractère. 

Dans ce cas, la définition ressemble à la suivante, où \tm{arg} et
\tm{char} sont les symbole auxquels \tm{interactive} lie l'argument
préfixé et le caractère spécifié :
\begin{center}
  \tm{(defun \textit{name-of-function} (arg char)}

  \tm{''\textit{documentation}\dots''}

  \tm{(interactive ''p\textbackslash~ncZap to char; '')}
  
  \tm{\textit{body-of-function}\dots)}
\end{center}
(L'espace après les deux points dans l'invite lui donne une meilleure
apparence lorsque vous êtes invité. Voir Section \cfchs{5}{1} <<La
définition de \tm{copy-to-buffer}>>, page \cfchsg{5}{1}, par exemple.)

Quand une fonction ne prend pas d'arguments, \tm{interactive} n'en
nécessite aucun. Une telle fonction contient l'expression simple
\tm{(interactive)}. La fonction \tm{mark-whole-buffer} est comme ça.

Alternativement, si les lettres-codes ne sont pas bonnes pour votre
application, vous pouvez passer vos propres arguments à
\tm{interactive} comme une liste.

Voir Section \cfchs{4}{4} <<La définition de \tm{append-to-buffer}>>,
page \cfchsg{4}{4}, par exemple. Voir Section <<Utilisation de
\tm{Interactive}>> dans \textit{The GNU Emacs Lisp Reference Manual},
pour une explication plus complète à propos de cette technique. 

\section{Installer du code de façon permanente}\etchs{3}{5}

\section{\texttt{let}}\etchs{3}{6}

\subsection{Les parties d'une expression \texttt{let}}\etchss{3}{6}{1}

\subsection{\'Echantillon d'expression \texttt{let}}\etchss{3}{6}{2}

\subsection{Variables non initialisées dans une instruction
  \texttt{let}}\etchss{3}{6}{3} 

\section{Variables}\etchs{1}{7}

Dans Emacs Lisp, un symbole peut avoir une valeur associée à lui tout
comme il peut avoir une définition de fonction associée. Les deux sont
différents. La définition de la fonction est un ensemble
d'instructions auquelles un ordinateur obéira. Une valeur, d'autre
part, est quelque chose, comme un nombre ou un nom, qui peut varier
(c'est pourquoi un tel symbole est appelé variable). La valeur d'un
symbole peut être n'importe quelle expression Lisp, comme un symbole,
un nombre, une liste, ou une chaîne. Un symbole qui a une valeur est
souvent appelé une variable. 

Un symbole peut avoir à la fois une définition de fonction et une
valeur fixée à lui en même temps. Ou il peut avoir l'un ou
l'autre. Les deux sont séparés. C'est un peu similaire à la façon dont
le nom Cambridge peut référer à la ville du Massachussets et avoir
quelques informations attachées au nom ainsi, comme <<grand centre de
programmation>>. 

Une autre façon de penser à ce sujet est d'imaginer un symbole comme
étant une commode. La définition de la fonction est mise dans un
tiroir, la valeur dans un autre et ainsi de suite. Ce qui est mis dans
le tiroir contenant la valeur peut être modifié sans affecter le
contenu du tiroir maintenant la définition de la fonciton, et
vice-versa. 

La variable \tm{fill-column} illustre un symbole avec une valeur
attachée à elle : dans chaque tampon \gem , ce symbole est réglé à une
valeur, généralement 72 ou 70, mais parfois à une autre valeur. Pour
trouver la valeur de ce symbole, d'évaluer par lui-même. Si vous lisez
ceci dans Info à l'intérieur de \gem , vous pouvez le faire en le
curseur après le symbole et en tapant \rec{C}{x}{C}{e} :
\begin{center}
  \tm{fill-column}
\end{center}

Après que j'ai tapé \rec{C}{x}{C}{e}, Emacs a imprimé le nombre 70
dans ma zone d'écho. c'est la valeur pour laquelle \tm{fill-column}
est réglée pour moi qui écris ceci. Il peut être différent pour vous
dans votre tampon Info. Notez que la valeur renvoyée comme une
variable est imprimée exactement de la même manière que la valeur
renvoyée par la fonction d'exécution des instructions. Du point de vue
de l'interprète Lisp, une valeur renvoyée est une valeur
renvoyée. Peu importe le genre d'expression une fois que la valeur est
connue.

Un symbole peut avoir n'importe quelle valeur attachée à lui ou, pour
utiliser le jargon, on peut lier la variable à une valeur : à un
certain nombre, comme \tm{72} ; à une chaîne, <<\tm{telle que ça}>>, à une
liste, comme \tm{(spruce pine oak)}; nous pouvons même lier une
variable à une définition de fonction. 

Un symbole peut être lié à une valeur de plusieurs façons. Voir la
section\cfchs{1}{9} <<Réglage de la valeur d'une variable>>,
page\cfchsg{1}{9}, pour des informations sur une façon de le faire.  

\subsection{Message d'erreur pour un symbole sans
  fonction}\etchss{1}{7}{1}

Lorsque nous avons évalué \tm{fill-column} pour trouver sa valeur en
tant que variable, nous n'avons pas placé des parenthèses autour du
mot. C'est parce que nous n'avons pas l'intention de l'utiliser comme
un nom de fonction.

Si \tm{fill-column} était le premier ou le seul élément d'une liste,
l'interprète Lisp tenterait de trouver la définition de fonction
attachée à elle. Mais \tm{fill-column} n'a pas de définition de
fonction. Essayez d'évaluer ceci :
\begin{center}
  \tm{(fill-column)}
\end{center}

Vous allez créé un tampon \tm{*Backtrace*} disant ceci:
{\ttfamily
\begin{flushleft}
  Debugger entered--Lisp error: (void-function fill-column)

  (fill-column)

  eval((fill-column))

  eval-last-sexp-1(nil)

  eval-last-sexp(nil)

  call-interactively(eval-last-sexp)
\end{flushleft}
}

(Rappelez-vous, pour quitter le débogueur et faire la fenêtre du
débogueur s'en aller, tapez \tm{q} dans le tampon \tm{*Backtrace*}.)


\subsection{Message d'erreur pour un symbole sans
  valeur}\etchss{1}{7}{2}

Si vous essayez d'évaluer un symbole qui ne possède pas une valeur qui
lui est liée, vous recevrez un message d'erreur. Vous pouvez voir cela
en expérimentant avec notre addition 2 plus 2. Dans l'expression
suivante, placez votre curseur à droite après le \tm{+}, avant le
premier nombre 2, tapez \rec{C}{x}{C}{e} :
\begin{center}
  \tm{(+ 2 2)}
\end{center}
Vous allez créé un tampon \tm{*Backtrace*} disant ceci:
{\ttfamily
\begin{flushleft}
  Debugger entered--Lisp error: (void-variable +)

  eval(+)

  eval-last-sexp-1(nil)

  eval-last-sexp(nil)

  call-interactively(eval-last-sexp)
\end{flushleft}
}

(Encore une fois, vous pouvez quitter le débogueur en tapant \tm{q}
dans le tampon \tm{*Backtrace*}.)

Ce backtrace est différent du premier message d'erreur que nous avons
vu, qui dit <<\tm{Debugger entered--Lisp error: (void-function
  this)}>>. Dans ce cas, la fonction n'a pas de valeur en tant que
variable; tandis que dans l'autre message d'erreur, la fonction (le
mot <<\tm{this}>>) n'avait pas de définition.

Dans cette expérience avec le \tm{+}, ce que nous faisions était
provoquer l'interprète Lisp évaluer le \tm{+} et rechercher la valeur
de la variable au lieu de la définition de la fonction. Nous l'avons
fait en plaçant le curseur juste après le symbole plutôt qu'après la
parenthèse de la liste englobante que nous avons fait avant. En
conséquence, l'interprète Lisp a évalué la s-expression précédente,
qui dans ce cas était \tm{+} lui-même.

Depuis \tm{+} n'a pas de valeur liée à elle, juste la définition de
fonction, le message d'erreur rapportait que la valeur du symbole
comme variable était nulle. 



\section{Arguments}\etchs{1}{8}

Pour voir comment l'information est transmise à des fonctions,
regardons de nouveau notre vieille veille, l'ajout de deux plus
deux. En Lisp, ça s'écrit comme suit :
\begin{center}
  \tm{(+ 2 2)}
\end{center}

Si vous évaluez cette expression, le nombre 4 apparaîtra dans votre
zone écho. Ce que fait l'interprète Lisp c'est d'ajouter les nombres
qui suivent le \tm{+}.

Les nombres ajoutés par \tm{+} sont appelés les arguments de la
fonction \tm{+}. Ces nombres sont les informations que l'on donne à ou
transmis à la fonction. 

Le mot <<argument>> vient de la façon dont il est utilisé en
mathématiques et ne se réfère pas à une dispute entre deux personnes;
au contraire, il se réfère à l'information présentée à la fonction,
dans ce cas, à la fonction \tm{+}. En Lisp, les arguments d'une
fonction sont les atomes ou les listes qui suivent la fonction. Les
valeurs renvoyées par l'évaluation de ces atomes ou des listes sont
passées à la fonction. Différentes fonctions nécessitent différents
nombres d'arguments; certaines fonctions n'en exigent pas du
tout\footnote{Il est curieux de suivre le chemin par lequel le mot
  <<argument>> est venu d'avoir deux significations différentes, l'une
  en mathématiques et l'autre en anglais de tous les jours. Selon le
  Oxford English Dictionary, le mot dérive du Latin <<\tm{faire
    comprendre, prouver}>>; ainsi venu à signifier, par un fil de
  dérivation, <<la preuve présentée comme preuve>>, c'est-à-dire,
  <<l'information offerte>>, qui a conduit à sa signification en
  Lisp. Mais dans l'autre fil de dérivation, il est venu à signifier
  <<d'affirmer d'une manière contre laquelle d'autres peuvent faire
  des contre affirmations>>, qui ont conduit à la signification du mot
  comme une dispute. (Notons ici que le mot anglais a deux définitions
  différentes qui s'y rattachent en même temps. En revanche, dans
  Emacs Lisp, un symbole ne peut pas avoir deux définitions de
  fonctions différentes en même temps.)}. 

\subsection{Arguments des types de données}\etchss{1}{8}{1}

Le type de données qui doivent être transmises à une fonction dépend
de quel type d'information elle utilise. Les arguments d'une fonction
telle que \tm{+} doivent avoir des valeurs qui sont des nombres,
depuis \tm{+} ajoute des nombres. D'autres fonctions utilisent
différents types de données pour leurs arguments.

Par exemple, la fonction \tm{concat} relie ou réunit deux ou plusieurs
chaînes de caractères pour produire une chaîne. Les arguments sont des
chaînes. La concaténation des deux chaînes de caractères \tm{abc},
\tm{def} produit la chaîne \tm{abcdef} seule. Ceci peut être vu par
l'évaluation de ce qui suit : 
\begin{center}
  \tm{(concat 'abc' 'def')} % ne marche pas en TeX
\end{center}

La valeur produite en évaluant cette expression est <<\tm{abcdef}>>.

Une fonction comme sous-chaîne utilise à la fois une chaîne et des
nombres comme arguments. La fonction renvoie une partie de la chaîne,
une chaîne du premier argument. Cette fonction prend trois
arguments. Son premier argument est la chaîne de caractères, les
deuxième et troisième arguments sont des nombres qui indiquent le
début et la fin de la sous-chaîne. Les nombres sont un comptage du
nombre de caractères (ponctuation et espaces compris) depuis le début
de la chaîne.

Par exemple, si vous évaluez la suivante :
\begin{center}
  \tm{(substring ''The quick brown fox jumped.'' 16 19)}
\end{center}

vous verrez ''\tm{fox}'' apparaître dans la zone d'écho. Les arguments
sont la chaîne et les deux nombres. 

Notez que la chaîne passée à \tm{substring} est un atome, même si elle
est composée de plusieurs mots séparés par des espaces. Lisp compte
tout entre les deux guillemets dans le cadre de la chaîne, y compris
les espaces. Vous pouvez penser la fonction \tm{substring} comme une
sorte <<d'écraseur d'atome>>, car il faut un atome contraire à
l'indivisible et en extraire une partie. Cependant, \tm{substring} est
seulement capable d'extraire une sous-chaîne à partir d'un argument
qui est une chaîne, pas d'un autre type d'atome comme un nombre ou un
symbole.



\subsection{Un argument comme la valeur d'une variable ou d'une
  liste}\etchss{1}{8}{2}

Un argument peut être un symbole qui renvoie une valeur quand il est
évalué. Par exemple, lorsque le symbole de remplissage par colonne
lui-même est évalué, il renvoie un nombre. Ce nombre peut être utilisé
pour une addition. Positionner le curseur après l'expression suivante
et le type \rec{C}{x}{C}{e}:
\begin{center}
  \tm{(+ 2 fill-column)}
\end{center}

La valeur sera un nombre deux de plus que ce que vous obtenez en
évaluant \tm{fill-column} seul. Pour moi, c'est 74, parce que ma
valeur de \tm{fill-column} est 72.

Comme nous venons de le voir, un argument peut être un symbole qui
renvoie une valeur lors de l'évaluation. En outre, un argument peut
être une liste qui renvoie une valeur quand il est évalué. Par
exemple, dans l'expression suivante, les arguments de la fonction
\tm{concat} sont les chaînes <<The>> et <<red foxes.>> et la liste
\tm{(number-to-string (+ 2 fill-column))}.
\begin{center}
  \tm{(concat ''The '' (number-to-string (+ 2 fill-column)) '' red foxes.'')}
\end{center}

Si vous évaluez cette expression et si, comme avec mon Emacs,
\tm{fill-column} évalue à 72---<<The 74 red foxes.>> apparaît dans la
zone écho. (Notez que vous devez mettre des espaces après le mot
<<The>> et avant le mot <<red>> de sorte qu'ils apparaissent dans la
chaîne finale. La fonction \tm{number-to-string} convertit l'entier
que la fonction d'addition renvoie à une chaîne. \tm{number-to-string}
est également connu comme \tm{int-to-string}.)


\subsection{Nombre variable d'arguments}\etchss{1}{8}{3}

Certaines fonctions, comme \tm{concat}, \tm{+} ou \tm{*}, prendre
n'importe quel nombre d'arguments. (Le \tm{*} est le symbole de
multiplication.) Ceci peut être vu par l'évaluation de chacune des
expressions suivantes de la manière habituelle. Ce que vous verrez
dans la zone d'écho est imprimé dans ce texte après <<$\Rightarrow$>>,
que vous pouvez lire comme <<évalué à >>.

Dans la première série, les fonctions n'ont pas d'arguments:
\begin{center}
  \begin{tabular}[m]{lrl}
    \tm{(+)} &$\Rightarrow$ & 0 \\
    \tm{(*)} &$\Rightarrow$ & 1 
  \end{tabular}
\end{center}

Dans cet ensemble, les fonctions ont un argument de chaque:
\begin{center}
  \begin{tabular}[m]{lrl}
    \tm{(+ 3)} &$\Rightarrow$ & 3 \\
    \tm{(* 3)} &$\Rightarrow$ & 3 
  \end{tabular}
\end{center}

Dans cet ensemble, les fonctions ont trois arguments de chaque:
\begin{center}
  \begin{tabular}[m]{lrl}
    \tm{(+ 3 4 5)} &$\Rightarrow$ & 12 \\
    \tm{(* 3 4 5)} &$\Rightarrow$ & 60 
  \end{tabular}
\end{center}


\subsection{Utiliser le mauvais type d'objet pour un argument}\etchss{1}{8}{4}

Quand une fonction est passée un argument de type incorrect,
l'interprète Lisp produit un message d'erreur. Par exemple, la
fonction \tm{+} attend les valeurs de ses arguments pour être des
nombres. Comme une expérience, nous pouvons passer le symbole cité
bonjour au lieu d'un nombre. Positionner le curseur après l'expression
suivante et le type \rec{C}{x}{C}{e} :
\begin{center}
  \tm{(+ 2 'hello)}
\end{center}

Lorsque vous faites cela, vous allez générer un message d'erreur. Ce
qui est arrivé est que \tm{+} a essayé d'ajouter le 2 à la valeur
renvoyée par \tm{'hello}, mais la valeur renvoyée par \tm{'hello} est
le symbole \tm{hello}, pas un nombre. Seuls des chiffres peuvent être
ajoutés. Donc \tm{+} ne pouvait pas mener à bien son addition. 

Vous aller créer et entrer dans un tampon \tm{*Backtrace*} qui dit :

{\ttfamily
Debugger entered--Lisp error:
\begin{center}
  (wrong-type-argument number-or-marker-p hello)
\end{center}
+(2 hello)

eval((+ 2 (quote hello)))

eval-last-sexp-1(nil)

eval-last-sexp(nil)

call-interactively(eval-last-sexp)
}

Comme d'habitude, le message d'erreur essaie d'être utile et logique
après vous apprenez à lire\footnote{\tm{(quote hello)} est une
  expansion de l'abréviation \tm{'hello}}. 

La première partie du message d'erreur est simple; il est dit
<<\tm{wrong type argument}>>. Vient ensuite le mot de jargon
mystérieux <<\tm{number-or-marker-p}>>. Ce mot est d'essayer de vous
dire ce genre d'argument du \tm{+} prévu. 

Le symbole \tm{number-or-marker-p} dit que l'interprète Lisp essaie de
déterminer si l'information a présenté (la valeur de l'argument) est
un nombre ou un marqueur (un objet spécial représentant une position
de tampon). Ce qu'il fait est le test pour voir si le \tm{+} est étant
des nombres à ajouter. Il teste aussi de voir si l'argument est
quelque chose appelé un marqueur, qui est une caractéristique
spécifique de Emacs Lisp. (Dans Emacs, les emplacements dans une
mémoire tampon sont enregistrés en tant que marqueurs. Lorsque la
marque est réglée avec le \rep{C}{@} ou la commande \rep{C}{SPC}, sa
position est maintenue en tant que marqueur. La marque peut être
considérée comme un nombre---le nombre de caractères les emplacement
est à partir du début de la mémoire tampon). En Lisp, \tm{+} peut être
utilisé pour ajouter la valeur numérique de positions de marqueurs que
des nombres.

Le <<\tm{p}>> de <<\tm{number-or-marker-p}>> est l'incarnation d'une
pratique a commencé dans les premiers jours de la programmation
Lisp. Le <<\tm{p}>> signifie <<prédicat>>. Dans le jargon, utilisé par
les chercheurs Lisp, un prédicat renvoie à une fonction pour
déterminer si une propriété est vraie ou fausse. Ainsi, le <<\tm{p}>>
nous dit que ce \tm{number-or-marker-p} est le nom d'une fonction qui
détermine si c'est vrai ou faux que l'argument fourni est un nombre ou
un marqueur. Autres symboles Lisp qui se terminent par <<\tm{p}>>
comprennent \tm{zerop}, une fonction qui teste si son argument a la
valeur de zéro, et \tm{listp}, une fonction qui teste si son argument
est une liste. 

Enfin, la dernière partie du message d'erreur est le symbole
\tm{hello}. C'est la valeur de l'argument qui a été adopté pour
\tm{+}. Si l'addition avait été adoptée le bon type d'objet, la valeur
passée aurait été un certain nombre, comme 37, plutôt qu'un symbole
comme \tm{hello}. Mais alors vous n'auriez pas reçu le message d'erreur.

\subsection{La fonction \tm{message}}\etchss{1}{8}{5}

Comme \texttt{+}, la fonction de \texttt{message} prend un nombre
variable d'arguments. Elle est utilisée pour envoyer des messages à
l'utilisateur et est si utile que nous allons la décrire ici.

Un message est imprimé dans la zone écho. Par exemple, vous pouvez
imprimer un message dans votre zone de répercussion en évaluant la
liste suivante :   

\tm{(message ``This message appears in the echo area'')}

L'ensemble de la chaîne de caractères entre guillemets est un argument
unique et est imprimé dans sa totalité. (Notez que dans cet exemple,
le message lui-même apparaît dans la zone écho entre guillemets; cela
parce que vous voyez la valeur retournée par la fonction
\tm{message}. Dans la plupart des utilisations de \tm{message} dans
les programmes que vous écrivez, le texte sera imprimé dans la zone
d'écho comme un effet secondaire, sans les guillemets. Voir Section
\cfchss{3}{3}{1} ``multiply-by-seven in detail'' page
\cfchssg{3}{3}{1}, pour un exemple de cela).

Cependant, s'il y a un '\tm{\%s}' dans la chaîne de caractères entre
guillemets, la fonction \tm{message} n'imprimera pas le '\tm{\%s}' en
tant que tel, mais l'assimile à un argument qui suit la chaîne. Elle
évalue le second argument et imprime sa valeur à l'endroit où se situe
'\tm{\%s}' dans la chaîne.

Vous pouvez voir cela en positionnant le curseur après l'expression
suivante et en tapant \tm{C-x C-e}:

\tm{(message ``The name of this buffer is : \%s.'' (buffer-name))}

Dans Info, \tm{``The name of this buffer is :*info*.''} apparaîtra
dans la zone d'écho. La fonction \tm{buffer-name} renvoie  le nom du
tampon comme une chaîne, où la fonction \tm{message} insère à la place
de \tm{'\%s'}.

Pour imprimer une valeur comme un entier, utiliser \tm{'\%d'} de la
même manière que \tm{'\%s'}.  Par exemple, pour imprimer un message
dans la zone écho qui indique la valeur de \tm{fill-column}, évaluer les
points suivants:

\tm{(message ``The value of fill-column is: \%d.'' fill-column)}

Sur mon système, lorsque j'évalue cette list, \tm{``The value of
  fill-column is 72.''} apparaît dans ma zone d'écho.\footnote{En fait
    vous pouvez utiliser \tm{'\%s'} pour afficher un nombre. Ce n'est
    pas spécifique. \tm{\%d} n'affiche que la partie à gauche de la
    virgule d'un nombre décimal, et rien d'autre qui ne soit un
    nombre.}

S'il y a plus qu'un \tm{'\%s'} dans la chaîne entre les quotes, la
valeur du premier argument est affichée à l'endroit du premier
\tm{'\%s'}, la valeur du deuxième argument est affichée à l'endroit du
deuxième \tm{'\%s'} et ansi de suite.

Par exemple, si vous faites l'évaluation suivante : 

\tm{(message ``There are \%d \%s in the office!''}

\hspace{1.5cm}\tm{(- fill-column 14) ``pink elephants'')}

un message plutôt lunatique apparaîtra dans votre zone de
répercussion. Sur mon système, il dit, 

\tm{``The are 58 pink elephants in the office!''}.

L'expression \tm{(- fill-column 14)} est évaluée et le nombre
résultant est inséré à la place de \tm{'\%d'}; et la chaîne de
caractères \tm{``pink elephants''}, est traitée comme un argument
simple et insérée à la place de \tm{'\%s'} :

{\tt 
  \begin{center}
    \begin{tabular}[m]{*{3}l}
      (message &''He saw \%d \%s''  & \\
               & (- fill-column 32) & \\
               & (concat ``red ``   & \\
               &                    & (substring \\
               &                    & ``The quick brown foxes
                                      jumped.'' 16 21) \\
               &                    & `` leaping.''))
    \end{tabular}
  \end{center}
 }

Dans cet exemple, la fonction \tm{message} a trois arguments : la
chaîne \tm{``He saw \%d \%s''}, l'expression \tm{(- fill-column 32)}
est insérée à la place de \tm{'\%d'}; et la valeur renvoyée par l'expression
commençant par \tm{concat} est insérée à la place de \tm{'\%s'}. 

Lorsque votre colonne de remplissage est de 70 et que vous évaluer
l'expression, le message \tm{``He saw 38 red foxes leaping.''} aparaît
dans votre zone d'écho.



\section{Vérité et fausseté dans Emacs Lisp}\etchs{3}{9}

\section{Résumé}\etchs{1}{10}

L'apprentissage de Lisp est comme l'escalade d'une colline dans
laquelle la première partie est la plus dure. Vous avez maintenant
grimpé la partie la plus difficile; ce qui reste deviendra plus facile
maintenant que vous avez progressé.

En résumé,
\begin{itemize}
\item Les programmes Lisp sont constitués d'expressions, qui sont des
  listes ou des atomes individuels.
\item Les listes sont constituées de 0 ou plusieurs atomes ou de
  listes internes, séparées par des espaces et entourées par des
  parenthèses. Une liste peut être vide.
\item Les atomes sont des symboles multi-caractères, comme
  \tm{forward-paragraph}, symboles caractère simple comme \tm{+},
  chaînes de caractères entre guillemets, ou des nombres.
\item Un nombre s'évalue lui-même.
\item Une chaîne entre guillemets s'évalue également elle-même.
\item Lorsque vous évaluez un symbole par lui-même, sa valeur est
  renvoyée.
\item Lorsque vous évaluez une liste, l'interprète Lisp regarde
  d'abord le premier symbole dans la liste et ensuite la fonction liée
  à ce symbole. Ensuite les instructions dans la définition de
  fonction sont effectuées.
\item Un guillemet simple (quote), \tm{'}, dit à l'interprète Lisp
  qu'il devrait renvoyer l'expression suivante comme écrite, et ne pas
  l'évaluer comme si la quote n'était pas là.
\item Les arguments sont des information passées à la fonction. Les
  arguments d'une fonction sont calculés par évaluation du reste des
  éléments de la liste dont la fonction est le premier élément.
\item Une fonction renvoie toujours une valeur quand elle est évaluée
  ( à moins d'obtenir une erreur); en outre, il peut aussi effectuer
  une action appelée <<effet secondaire>>. Dans de nombreux cas, le
  but principal d'une fonction est de créer un effet secondaire.
\end{itemize}






\section{Rappels}\etchs{3}{11}

\section{Exercices}\etchs{3}{12}



