\chapter{Comment écrire des définitions de fonctions}\etch{3}

Lorsque l'interprète Lisp évalue une liste, il regarde si le premier
symbole sur la liste a une définition de fonction attachée à lui; ou,
en d'autres termes, si le symbole pointe vers une définition de
fonction. Dans le cas contraire, l'ordinateur exécute les instructions
dans la définition. Un symbole qui a une définition de fonction est
appelé, tout simplement, une fonction (même si, à proprement parler,
la définition est la fonction et le symbole se réfère à elle.)

Toutes les fonctions sont définies en termes d'autres fonctions, à
l'exception de quelques fonctions primitives qui sont écrites dans le
langage de programmation C. Lorsque vous écrivez les définitions de
fonctions, vous les écrirez en Emacs Lisp et utiliserez d'autres
fonctions comme blocs de construction. Certaines des fonctions que
vous utiliserez seront elles-mêmes écrites en Emacs Lisp (peut être
par vous) et certaines seront des primitives écrites en C. Les
fonctions primitives sont utilisées exactement comme celles écrites en
Emacs Lisp et se comportent comme elles. Elles sont écrites en C afin
que nous puissions facilement lancer \gem sur tout ordinateur d'une
puissance suffisante pour faire fonctionner C.

Permettez-moi de souligner de nouveau ceci: quand vous écrivez du code
en Emacs Lisp, vous ne distinguez pas celles écrites en C de celles
écrites en Emacs Lisp. La différence est sans importance. Je mentionne
la distinction seulement parce qu'il est intéressant de le savoir. En
effet, à moins que vous ne l'étudiez, vous ne saurez pas si une
fonction déjà écrite est écrite en Emacs Lisp ou en C.

\section{Point}\etchs{1}{1}
Le curseur de la fenêtre sélectionnée indique l'emplacement où la
plupart des commandes d'édition prennent effet, qui est appelé
point\footnote{Le terme <<point>> vient du caractère <<.>>, qui était la
  commande TECO (le langage dans lequel l'Emacs original a été
  écrit) pour accéder à la position d'édition.}. Beaucoup
de commandes Emacs déplacent le point à différents endroits dans le
tampon; par exemple, vous pouvez placer en cliquant sur le bouton 1 de
la souris (usuellement le gauche) à l'endroit désiré. Par défaut, le
curseur de la fenêtre sélectionnée est dessiné comme un bloc plein et
paraît être un caractère, mais vous devrez penser le point comme entre
deux caractères; il est situé avant le caractère sous le curseur. Par
exemple; si votre texte ressemble à <<frob>> avec le curseur sur <<b>>,
alors le point est entre le <<o>> et le <<b>>. Si vous insérez le
caractère <<!>> à cette position, le résultat sera <<fro!b>>, avec le
point entre le <<!>> et le <<b>>. Ainsi, le curseur reste au-dessus du
<<b>>, comme précédemment.\par 

Si vous éditez plusieurs fichiers dans Emacs, chacun dans son propre
tampon, chaque tampon a sa propre valeur du point. Un tampon qui n'est
pas actuellement affiché se souvient encore de sa valeur du point si vous
l'afficher plus tard. En outre, si un tampon est affiché dans
plusieurs fenêtres, chacune de ces fenêtres a sa propre valeur du
point.\par

Voir la section\cfchs{11}{20} [Affichage Curseur],
page\cfchsg{11}{20}, pour les options qui contrôlent la façon
dont Emacs affiche le curseur.\par 


\section{Installer une définition de fonction}\etchs{3}{2}

Si vous lisez ceci à l'intérieur d'Info dans Emacs, vous pouvez
essayer la fonction \tm{multiply-by-seven} en évaluant d'abord la
définition de fonction puis en évaluant \tm{(multiply-by-seven
  3)}. Une copie de la définition de la fonction suivra. Placez le
curseur après la dernière parenthèse de la définition de fonction et
tapez \tm{C-x C-e}. Lorsque vous faites cela, \tm{multiply-by-seven}
apparaîtra dans la zone d'écho. (Ce que cela signifie c'est que
lorsqu'une définition de fonction est évaluée, la valeur renvoyée est
le nom de la fonction définie.) En même temps, cette action installe
la définition de fonction. 

\tm{(defun multiply-by-seven (number)}

\tm{   ``Multiply NUMBER by seven.''}

\tm{ (* 7 number))}

En évaluant ce \tm{defun}, vous venez d'installer
\tm{multiply-by-seven} dans Emacs. La fonction fait désormais autant
partie de Emacs que \tm{forward-word} ou toute autre fonction
d'édition que vous utilisez. (\tm{multiply-by-seven} restera installé
jusqu'à ce que vous quittiez Emacs. Pour recharger le code
automatiquement chaque fois que vous démarrez Emacs, voir Section
\cfchs{3}{5} ``Installation permanente de code'', page \cfchs{3}{5}.)

Vous pouvez voir l'effet de l'installation de \tm{multiply-by-seven}
en évaluant l'échantillon suivant. Placez le curseur après
l'expression suivante et tapez \tm{C-x C-e}. Le nombre 21 apparaîtra
dans la zone d'écho.

\tm{(multiply-by-seven 3)}

Si vous le souhaitez, vous pouvez lire la documentation de la fonction
en tapant \tm{C-h f (describe-function)} puis le nom de la fonction,
\tm{multiply-by-seven}. Lorsque vous faîtes cela, une fenêtre
\tm{*Help*} apparaîtra sur votre écran qui dit :

\tm{multiply-by-seven is a Lisp function.}

\tm{(multiply-by-seven NUMBER)}

\tm{Multiply NUMBER by seven.}

(Pour revenir à une fenêtre simple sur votre écran, tapez \tm{C-x 1}.)
 
\subsection{Changer une définition de fonction}\etchss{3}{2}{1}

Si vous souhaitez modifier le code dans \tm{multiply-by-seven}, alors
réécrivez-le. Pour installer la nouvelle version à la place de
l'ancienne, évaluez à nouveau la définition de fonction. Ceci est la
façon dont vous modifiez le code dans Emacs. C'est très simple.

\`A titre d'exemple, vous pouvez modifier la fonction
\tm{multiply-by-seven} de sorte qu'elle ajoute le nombre à lui-même
sept fois au lieu de multiplier le nombre par sept. Cela produit le
même résultat, mais par un chemin différent. En même temps, nous
ajouterons un commentaire à ce code; un commentaire est un texte que
l'interprète Lisp ignore, mais qu'un lecteur humain peut trouver utile
ou instructif. Le commentaire est <<seconde version>>.

\tm{(defun multiply-by-seven (number)   ;} Second version.

\tm{ ``Multiply NUMBER by seven.''}

\tm{ (+ number number number number number number number))}

Le commentaire suit un point-virgule, '\tm{;}'. En Lisp, tout sur une
ligne qui suit un point-virgule est un commentaire. La fin de ligne
est la fin du commentaire. Pour étirer un commentaire sur deux ou
plusieurs lignes, chaque ligne doit commencer par un point-virgule.

Voir Section 16.3 ``Commencer un fichier \tm{.emacs}'', page 187, et
Section ``Commentaires'' dans \textit{The GNU Emacs Lisp Reference
  Manual}, pour plus de renseignements à propos des commentaires.

Vous pouvez installer cette version de la fonction
\tm{multiply-by-seven} en l'évaluant de la même manière que vous avez
évalué la première fonction : placez le curseur après la dernière
parenthèse et tapez \tm{C-x C-e}.

En résumé, voici comment vous écrivez du code en Emacs Lisp: vous
écrivez une fonction; vous l'installez; vous la testez; et ensuite
vous faites des corrections ou améliorations et vous l'installez à
nouveau. 

\section{Générer un message d'erreur}\etchs{1}{3}

Partant de sorte que vous ne vous inquiétez pas si vous le faites
accidentellement, nous allons maintenant donner un ordre à
l'interprète Lisp qui génère un message d'erreur. C'est une activité
inoffensive; et en effet, nous allons souvent essayer de générer des
messages d'erreur intentionnellement. Une fois que vous comprenez le
jargon, les messages d'erreur peuvent être informatif. Au lieu d'être
appelé ``messages d'erreur'', ils devraient être appelés messages
``d'aide''. Ils sont comme des panneaux pour un voyageur dans un pays
étranger ; les déchiffrer peut être difficile, mais une fois compris,
ils peuvent indiquer la voie. 

Le message d'erreur est généré par un débogueur \gem intégré. Nous
allons <<entrer dans le débogueur>>. Pour sortir du débogueur taper
\tm{q}.

Ce que nous allons faire c'est d'évaluer une liste qui n'a pas de
quote la précédant et qui n'a pas de commande significative comme
premier élément. Voici une liste presque exactement la même que celle
que nous avons utilisé, mais sans la quote la précédant. Placez le
curseur juste après et taper \rec{C}{x}{C}{e}:
\begin{center}
  \tm{(this is an unquoted list)}
\end{center}

(this is an unquoted list)
Une fenêtre \tm{*Backtrace*} s'ouvrira et vous devriez voir ce qui suit:
{\ttfamily
\begin{flushleft}
  Debugger entered--Lisp error: (void-function this)
  
  (this is an unquoted list)

  eval((this is an unquoted list))

  eval-last-sexp-1(nil)

  eval-last-sexp(nil)

  call-interactively(eval-last-sexp)
\end{flushleft}}

Votre curseur sera dans cette fenêtre (vous pouvez avoir à attendre
quelques secondes avant qu'il ne devienne visible). Pour quitter le
débogueur et faire la fenêtre du débogueur s'en aller, tapez: \tm{q}.

S'il vous plaît tapez \tm{q} maintenant, alors vous devenez plus
confiant et vous pouvez sortir du débogueur. Ensuite, tapez
\rec{C}{x}{C}{e} de nouveau pour y rentrer.

Sur la base de ce que nous savons déjà, nous pouvons presque lire ce
message d'erreur.

Vous avez lu le tampon \tm{*Backtrace*} de bas en haut; il vous dit ce
que Emacs fait. Lorsque vous avez tapé \rec{C}{x}{C}{e}, vous avez
fait un appel à la commande \tm{eval-last-sexp}. \tm{eval} est une
abréviation pour <<évaluer>> et \tm{sexp} est une abréviation pour
<<expression symbolique>>. La signification de cette commande est
<<évaluer la dernière expression symbolique>>, qui est l'expression
juste avant le curseur.

Chaque ligne ci-dessus vous indique ce que l'interprète Lisp évalue
après. L'action la plus récente est en haut. Le tampon est appelé
\tm{*Backtrace*} car il vous permet de suivre Emacs en arrière.

En haut du tampon \tm{*Backtrace*}, vous voyez la ligne :
\begin{center}
  \tm{Debugger entered--Lisp error: (void-function this)}
\end{center}

L'interprète Lisp a tenté d'évaluer le premier atome de la liste, le
mot <<this>>. C'est cette action qui a généré le message d'erreur
<<void-finction this>>. 

Le message contient les mots <<void-function>> et <<this>>. Le mot
<<function>> a été mentionné une seule fois auparavant. C'est un mot
très important.

Pour nos fins, nous pouvons la définir en disant qu'une fonction est
un ensemble d'instructions données à l'ordinateur qui signale à
l'oridnateur de faire quelque chose. 

Maintenant, nous pouvons commencer à comprendre le message d'erreur:
<<\tm{void-function this}>>. La fonction (ce mot est le mot
<<\tm{this}>>) n'a pas de définition d'un ensemble d'instructions que
l'ordinateur peut mener à bien. 

Le mot un peu bizarre, <<\tm{void-function}>>, est conçu pour couvrir
la façon dont Emacs Lisp est mis en \oe{}uvre, qui est que lorsqu'un
symbole n'a pas de définition de fonction attaché à lui, la place qui
doit contenir les instructions est <<\tm{void}>>.

D'autre part, puisque nous avons pu ajouter 2 plus 2 avec succès, en
évaluant \tm{(+ 2 2)} nous pouvons en déduire que le symbole \tm{+}
doit avoir un ensemble d'instructions pour l'ordinateur qui doit obéir
et ces instructions doivent être d'ajouter les nombres qui suivent le
\tm{+}.

Il est possible de prévenir Emacs d'entrer dans le débogueur dans de
tels cas. Nous n'expliquons pas comment faire ici, mais nous allons
parler de ce à quoi le résultat ressemble, parce que vous pouvez
rencontrer une situation similaire s'il y a un bogue dans un code
Emacs que vous utilisez. Dans de tels cas, vous verrez seulement une
ligne de message d'erreur ; il apparaîtra dans la zone écho et
ressemblera à ceci:
\begin{center}
  \tm{Symbol's function definition is void: this}
\end{center}

Le message disparaît dès que vous tapez une touche, même juste pour
déplacer le curseur. Nous connaissons le sens du mot
<<\tm{Symbol}>>. Il se réfère au premier atome de la liste, le mot
<<\tm{this}>>. Le mot <<\tm{function}>> se réfère aux instructions qui
indiquent à l'ordinateur ce qu'il faut faire. (Techniquement, le
symbole indique à l'ordinateur où trouver les instructions, mais c'est
une complication que nous pouvons ignorer pour le moment.) Le message
d'erreur peut être compris: <<\tm{Symbol's function definition is
  void: this}>>. Le symbole (c'est le mot <<\tm{this}>>) manque
d'instructions pour que l'ordinateur les mènent à bien. 




\section{Taille du tampon et localisation du point}\etchs{2}{4}

Enfin, regardons plusieurs fonctions assez simples, \tm{buffer-size,
  point-min} et \tm{point-max}. Celles-ci donnent des informations sur
la taille d'un tampon et l'emplacement du point en son sein.

La fonction \tm{buffer-size} vous indique la taille du tampon courant;
autrement dit, la fonction renvoie un compte du nombre de caractères
dans le tampon.

\tm{(buffer-size)}

Vous pouvez évaluer cela de façon habituelle, en positionnant le
curseur après l'expression et en tapant \tm{C-x C-e}.

Dans Emacs, la position actuelle du curseur est appelée le
\textit{point}. L'expression \tm{(point)} renvoie un nombre qui vous
indique où se trouve le curseur comme un comptage de nombre de
caractères à partir du début du tampon jusqu'au point.

Vous pouvez voir le nombre de caractères pour le point dans ce tampon
en évaluant l'expression suivante de la manière usuelle :

\tm{(point)}

Pendant que j'écris ceci, la valeur du point est 65724. La fonction
\tm{point} est utilisée fréquemment dans certains exemples plus loin
dans ce livre. 

La valeur du point dépend, bien sûr, de son emplacement dans le
tampon. Si vous évaluez le point à cet endroit, le nombre sera plus
grand :

\tm{(point)}

Pour moi, la valeur du point à cet endroit est 66043, ce qui signifie
qu'il y a 319 caractères (espaces compris) entre les deux
expressions. (Sans doute, vous verrez des nombres différents, puisque
j'aurais édité depuis ma première évaluation du point.)

La fonction \tm{point-min} est quelque peu semblable à un \tm{point}, mais
elle renvoie la valeur de la valeur minimale admissible du point dans
le tampon courant. C'est le nombre 1 sauf si le \textit{réduction} est
en vigueur. (La réduction est un mécanisme par lequel vous pouvez vous
limiter, ou un programme, d'opérations sur seulement une partie d'un
tampon. Voir le chapitre \cfch{6} ``Réduction et élargissement'', page
\cfchg{6}.) De même, la fonction \tm{point-max} renvoie la valeur de
la valeur maximale admissible du point dans le tampon courant.


\section{Interpréteur Lisp}\etchs{1}{5}

Sur la base de ce que nous avons vu, nous pouvons maintenant commencer
à comprendre ce que l'interprète Lisp fait lorsque nous commandons à
évaluer une liste. Tout d'abord, il semble pour voir s'il y a une
quote avant que la liste ; s'il y a, l'interprète nous donne juste
la liste. D'un autre côté, s'il n'y a pas de quote, l'interprète
regarde le premier élément de la liste et voit si elle a une
définition de fonction. Si c'est le cas, l'interpréteur exécute les
instructions dans la définition de fonction. Sinon, l'interpréteur
affiche un message d'erreur. 

C'est ainsi que fonctionne Lisp simple. Il y a des complications
supplémentaires que nous allons obtenir dans une minute, mais ce sont
les fondamentaux. Bien sûr, pour écrire des programmes Lisp, vous
devez savoir comment écrire les définitions de fonctions et leur
donner des noms, et comment le faire sans confondre vous-même ou
l'ordinateur. 

Maintenant, pour la première complication, en plus de la liste,
l'interpréteur Lisp peut évaluer un symbole qui n'a pas de quote et
qui n'a pas de parenthèses autour de lui. L'interprète Lisp tentera de
déterminer la valeur du symbole comme une variable. Cette situation
est décrite dans la section sur les variables. (Voir la
section\cfchs{1}{7} <<Variables>>, page\cfchsg{1}{7}.)

La deuxième complication se produit parce que certaines fonctions sont
inhabituelles et ne fonctionnent pas de manière habituelle. Celles qui
ne le font pas sont appelées formes spéciales. Elles sont utilisées
pour des tâches spéciales, comme la définition d'une fonction, et il
n'y en a pas beaucoup. Dans les prochains chapitres, vous serez initié
à plusieurs de ces formes spéciales les plus importantes. 

La troisième et dernière complication est la suivante: si la fonction
que l'interprète Lisp regarde n'est pas une forme particulière, et si
elle fait partie d'une liste, l'interprète Lisp regarde pour voir si
la liste a une liste à l'intérieur de celui-ci. S'il y a une liste
interne, l'interprète Lisp décide ce qu'il doit faire avec la
liste intérieur en premier, puis il travaille sur la liste
extérieure. S'il y a encore une autre liste incorporée à l'intérieur
de la liste intérieure, cela fonctionne sur celui-là en premier, et
ainsi de suite. Il travaille toujours sur la liste la plus profonde en
premier. L'interprète travaille sur la liste la plus interne d'abord,
pour évaluer le résultat de cette liste. Le résultat peut être utilisé
par l'expression l'encapsulant.

Sinon, l'interprète travaille de gauche à droite, d'une expression à
l'autre.

\subsection{Compilation d'octet}\etchss{1}{5}{1}

Un autre aspect de l'interprétation : l'interprète Lisp est capable
d'interpréter deux types d'entités : du code lisible par des humains,
sur lequel nous nous concentrerons exclusivement, et du code
spécialement traité, appelé byte code compilé, qui n'est pas lisible
par les humains. Le byte code compilé s'exécute plus rapidement que le
code lisible par les humains. 

Vous pouvez transformer un code lisible pour les humains en un byte
code compilé en exécutant l'une des commandes de compilation telles
que \tm{byte-compile-file}. Le byte code compilé est généralement
stocké dans un fichier qui se termine par une extension \tm{.elc}
plutôt qu'une extension \tm{.el}. Vous verrez deux types de fichier
dans le répertoire \tm{emacs/lisp} ; les fichiers à lire sont ceux qui
portent l'extension \tm{.el}.

En pratique, pour la plupart des choses que vous pourriez faire pour
personnaliser ou étendre Emacs, vous n'avez pas besoin d'une
compilation de byte code ; et je ne vais pas en parler ici. Voir la
section <<Byte compilation>> dans le manuel Emacs Lisp reference, pour
une description complète de la byte compilation.
 



\section{\'Evaluation}\etchs{1}{6}

Lorsque l'interpréteur Lisp fonctionne sur une expression, le terme de
l'activité est appelée évaluation. Nous disons que l'interprète
<<évalue l'expression>>. J'ai utilisé ce terme plusieurs fois
avant. Le mot vient de son utilisation dans le langage courant, <<pour
déterminer la valeur ou la quantité de ; d'évaluer>>, selon le Webster New
Collegiate Dictionnary.

Après avoir évalué une expression, l'interprète Lisp va probablement
renvoyer la valeur que l'ordinateur produit en effectuant les
instructions trouvées dans la définition de fonction, ou peut-être
qu'il donnera pour cette fonction un message d'erreur. (L'interprète
peut également se trouver jeté, pour ainsi dire, à une fonction
différente ou il peut tenter de répéter sans cesse ce qu'il fait à
tout jamais dans ce qu'on appelle une <<boucle infinie>>. Ces actions
sont moins fréquentes ; et nous pouvons les ignorer.) Le plus souvent,
l'interprète renvoie une valeur. 

Dans le même temps l'interprète renvoie une valeur, il peut faire
autre chose ainsi, comme déplacer un curseur ou copeir un fichier ;
cet autre type d'action est appelé un effet secondaire. Les actions que
nous les humains pensont être importantes, telles que les résultats
d'impression, sont souvent des <<effets secondaires>> à l'interprète
Lisp. Le jargon peut paraître étrange, mais il s'avère qu'il est assez
facile d'apprendre à utiliser des effets secondaires.

En résumé, l'évaluation d'une expression symbolique incite le plus souvent
l'interprète Lisp à renvoyer une valeur et peut être à réaliser un
effet secondaire ; ou bien produire une erreur.

\subsection{\'Evaluation des listes internes}\etchss{1}{6}{1}

Si l'évaluation s'applique à une liste qui est à l'intérieur d'une
autre liste, la liste extérieure peut utiliser la valeur renvoyée par
la première évaluation comme information lorsque la liste extérieure
sera évaluée. Cela explique pourquoi les expressions intérieures sont
évaluées en premier : les valeurs qu'elles renvoient sont utilisées
par les expressions extérieures.

Nous pouvons enquêter sur ce processus en évaluant un autre exemple de
plus. Placez votre curseur après l'expression suivante et tapez
\rec{C}{x}{C}{e} :
\begin{center}
  \tm{(+ 2 (+ 3 3))}
\end{center}

Le nombre 8 apparaîtra dans la zone d'écho.

Ce qui se passe c'est que l'interprète Lisp évalue d'abord
l'expression intérieure, \tm{(+ 3 3)}, pour laquelle la valeur
renvoyée est 6; il évalue l'expression externe comme s'il était écrit
\tm{(+ 2 6)}, qui renvoie la valeur 8. Comme il n'y a pas
d'expressions plus englobantes pour évaluer, l'interprète affiche
cette valeur dans la zone d'écho.

Maintenant, il est facile de comprendre le nom de la commande invoquée
par le raccourci \rec{C}{x}{C}{e} : le nom est
\tm{eval-last-sexp}. Les lettres \tm{sexp} sont une abréviation pour
<<expression symbolique>>, et \tm{eval} pour <<évaluer>>. La commande
signifie <<évaluer la dernière expression symbolique>>.

\`A titre expérimental, on peut essayer d'évaluer l'expression en
plaçant le curseur au début de la ligne suivante immédiatement après
l'expression, ou l'expression à l'intérieur. 

Voici une autre copie de l'expression : 
\begin{center}
  \tm{(+ 2 (+ 3 3))}
\end{center}

Si vous placez le cursuer au début de la ligne blanche et tapez
\rec{C}{x}{C}{e}, vous obtiendrez toujours la valeur 8 imprimée dans
la zone d'écho. Maintenant, essayez de mettre le curseur dans
l'expression. Si vous le mettez juste après l'avant-dernière
parenthèse (il apparaît donc de s'asseoir sur le dessus de la dernière
parenthèse), vous obtiendrez un 6 imprimé dans la zone écho! C'est
parce que la commande évalue l'expression \tm{(+ 3 3)}.

Maintenant, mettez le curseur immédiatement après un certain
nombre. Type \rec{C}{x}{C}{e} et vous obtiendrez le nombre
lui-même. En Lisp, si vous évaluez un certain nombre, vous obtenez le
nombre lui-même, c'est en cela que les nombres diffèrent des
symboles. Si vous évaluez une liste à partir d'un symbole comme
\tm{+}, vous aurez une valeur renvoyée c'est le résultat de
l'ordinateur après exécution des instructions de la définition de la
fonction associée à ce nom. Si un symbole en lui-même est évalué,
quelque chose de différent se produit, comme nous le verrons dans la
prochaine section.



\section{La forme spéciale \texttt{if}}\etchs{3}{7}

\subsection{La fonction \texttt{type-of-animal} en
  détail}\etchss{3}{7}{1}


\section{Arguments}\etchs{1}{8}

Pour voir comment l'information est transmise à des fonctions,
regardons de nouveau notre vieille veille, l'ajout de deux plus
deux. En Lisp, ça s'écrit comme suit :
\begin{center}
  \tm{(+ 2 2)}
\end{center}

Si vous évaluez cette expression, le nombre 4 apparaîtra dans votre
zone écho. Ce que fait l'interprète Lisp c'est d'ajouter les nombres
qui suivent le \tm{+}.

Les nombres ajoutés par \tm{+} sont appelés les arguments de la
fonction \tm{+}. Ces nombres sont les informations que l'on donne à ou
transmis à la fonction. 

Le mot <<argument>> vient de la façon dont il est utilisé en
mathématiques et ne se réfère pas à une dispute entre deux personnes;
au contraire, il se réfère à l'information présentée à la fonction,
dans ce cas, à la fonction \tm{+}. En Lisp, les arguments d'une
fonction sont les atomes ou les listes qui suivent la fonction. Les
valeurs renvoyées par l'évaluation de ces atomes ou des listes sont
passées à la fonction. Différentes fonctions nécessitent différents
nombres d'arguments; certaines fonctions n'en exigent pas du
tout\footnote{Il est curieux de suivre le chemin par lequel le mot
  <<argument>> est venu d'avoir deux significations différentes, l'une
  en mathématiques et l'autre en anglais de tous les jours. Selon le
  Oxford English Dictionary, le mot dérive du Latin <<\tm{faire
    comprendre, prouver}>>; ainsi venu à signifier, par un fil de
  dérivation, <<la preuve présentée comme preuve>>, c'est-à-dire,
  <<l'information offerte>>, qui a conduit à sa signification en
  Lisp. Mais dans l'autre fil de dérivation, il est venu à signifier
  <<d'affirmer d'une manière contre laquelle d'autres peuvent faire
  des contre affirmations>>, qui ont conduit à la signification du mot
  comme une dispute. (Notons ici que le mot anglais a deux définitions
  différentes qui s'y rattachent en même temps. En revanche, dans
  Emacs Lisp, un symbole ne peut pas avoir deux définitions de
  fonctions différentes en même temps.)}. 

\subsection{Arguments des types de données}\etchss{1}{8}{1}

Le type de données qui doivent être transmises à une fonction dépend
de quel type d'information elle utilise. Les arguments d'une fonction
telle que \tm{+} doivent avoir des valeurs qui sont des nombres,
depuis \tm{+} ajoute des nombres. D'autres fonctions utilisent
différents types de données pour leurs arguments.

Par exemple, la fonction \tm{concat} relie ou réunit deux ou plusieurs
chaînes de caractères pour produire une chaîne. Les arguments sont des
chaînes. La concaténation des deux chaînes de caractères \tm{abc},
\tm{def} produit la chaîne \tm{abcdef} seule. Ceci peut être vu par
l'évaluation de ce qui suit : 
\begin{center}
  \tm{(concat 'abc' 'def')} % ne marche pas en TeX
\end{center}

La valeur produite en évaluant cette expression est <<\tm{abcdef}>>.

Une fonction comme sous-chaîne utilise à la fois une chaîne et des
nombres comme arguments. La fonction renvoie une partie de la chaîne,
une chaîne du premier argument. Cette fonction prend trois
arguments. Son premier argument est la chaîne de caractères, les
deuxième et troisième arguments sont des nombres qui indiquent le
début et la fin de la sous-chaîne. Les nombres sont un comptage du
nombre de caractères (ponctuation et espaces compris) depuis le début
de la chaîne.

Par exemple, si vous évaluez la suivante :
\begin{center}
  \tm{(substring ''The quick brown fox jumped.'' 16 19)}
\end{center}

vous verrez ''\tm{fox}'' apparaître dans la zone d'écho. Les arguments
sont la chaîne et les deux nombres. 

Notez que la chaîne passée à \tm{substring} est un atome, même si elle
est composée de plusieurs mots séparés par des espaces. Lisp compte
tout entre les deux guillemets dans le cadre de la chaîne, y compris
les espaces. Vous pouvez penser la fonction \tm{substring} comme une
sorte <<d'écraseur d'atome>>, car il faut un atome contraire à
l'indivisible et en extraire une partie. Cependant, \tm{substring} est
seulement capable d'extraire une sous-chaîne à partir d'un argument
qui est une chaîne, pas d'un autre type d'atome comme un nombre ou un
symbole.



\subsection{Un argument comme la valeur d'une variable ou d'une
  liste}\etchss{1}{8}{2}

Un argument peut être un symbole qui renvoie une valeur quand il est
évalué. Par exemple, lorsque le symbole de remplissage par colonne
lui-même est évalué, il renvoie un nombre. Ce nombre peut être utilisé
pour une addition. Positionner le curseur après l'expression suivante
et le type \rec{C}{x}{C}{e}:
\begin{center}
  \tm{(+ 2 fill-column)}
\end{center}

La valeur sera un nombre deux de plus que ce que vous obtenez en
évaluant \tm{fill-column} seul. Pour moi, c'est 74, parce que ma
valeur de \tm{fill-column} est 72.

Comme nous venons de le voir, un argument peut être un symbole qui
renvoie une valeur lors de l'évaluation. En outre, un argument peut
être une liste qui renvoie une valeur quand il est évalué. Par
exemple, dans l'expression suivante, les arguments de la fonction
\tm{concat} sont les chaînes <<The>> et <<red foxes.>> et la liste
\tm{(number-to-string (+ 2 fill-column))}.
\begin{center}
  \tm{(concat ''The '' (number-to-string (+ 2 fill-column)) '' red foxes.'')}
\end{center}

Si vous évaluez cette expression et si, comme avec mon Emacs,
\tm{fill-column} évalue à 72---<<The 74 red foxes.>> apparaît dans la
zone écho. (Notez que vous devez mettre des espaces après le mot
<<The>> et avant le mot <<red>> de sorte qu'ils apparaissent dans la
chaîne finale. La fonction \tm{number-to-string} convertit l'entier
que la fonction d'addition renvoie à une chaîne. \tm{number-to-string}
est également connu comme \tm{int-to-string}.)


\subsection{Nombre variable d'arguments}\etchss{1}{8}{3}

Certaines fonctions, comme \tm{concat}, \tm{+} ou \tm{*}, prendre
n'importe quel nombre d'arguments. (Le \tm{*} est le symbole de
multiplication.) Ceci peut être vu par l'évaluation de chacune des
expressions suivantes de la manière habituelle. Ce que vous verrez
dans la zone d'écho est imprimé dans ce texte après <<$\Rightarrow$>>,
que vous pouvez lire comme <<évalué à >>.

Dans la première série, les fonctions n'ont pas d'arguments:
\begin{center}
  \begin{tabular}[m]{lrl}
    \tm{(+)} &$\Rightarrow$ & 0 \\
    \tm{(*)} &$\Rightarrow$ & 1 
  \end{tabular}
\end{center}

Dans cet ensemble, les fonctions ont un argument de chaque:
\begin{center}
  \begin{tabular}[m]{lrl}
    \tm{(+ 3)} &$\Rightarrow$ & 3 \\
    \tm{(* 3)} &$\Rightarrow$ & 3 
  \end{tabular}
\end{center}

Dans cet ensemble, les fonctions ont trois arguments de chaque:
\begin{center}
  \begin{tabular}[m]{lrl}
    \tm{(+ 3 4 5)} &$\Rightarrow$ & 12 \\
    \tm{(* 3 4 5)} &$\Rightarrow$ & 60 
  \end{tabular}
\end{center}


\subsection{Utiliser le mauvais type d'objet pour un argument}\etchss{1}{8}{4}

Quand une fonction est passée un argument de type incorrect,
l'interprète Lisp produit un message d'erreur. Par exemple, la
fonction \tm{+} attend les valeurs de ses arguments pour être des
nombres. Comme une expérience, nous pouvons passer le symbole cité
bonjour au lieu d'un nombre. Positionner le curseur après l'expression
suivante et le type \rec{C}{x}{C}{e} :
\begin{center}
  \tm{(+ 2 'hello)}
\end{center}

Lorsque vous faites cela, vous allez générer un message d'erreur. Ce
qui est arrivé est que \tm{+} a essayé d'ajouter le 2 à la valeur
renvoyée par \tm{'hello}, mais la valeur renvoyée par \tm{'hello} est
le symbole \tm{hello}, pas un nombre. Seuls des chiffres peuvent être
ajoutés. Donc \tm{+} ne pouvait pas mener à bien son addition. 

Vous aller créer et entrer dans un tampon \tm{*Backtrace*} qui dit :

{\ttfamily
Debugger entered--Lisp error:
\begin{center}
  (wrong-type-argument number-or-marker-p hello)
\end{center}
+(2 hello)

eval((+ 2 (quote hello)))

eval-last-sexp-1(nil)

eval-last-sexp(nil)

call-interactively(eval-last-sexp)
}

Comme d'habitude, le message d'erreur essaie d'être utile et logique
après vous apprenez à lire\footnote{\tm{(quote hello)} est une
  expansion de l'abréviation \tm{'hello}}. 

La première partie du message d'erreur est simple; il est dit
<<\tm{wrong type argument}>>. Vient ensuite le mot de jargon
mystérieux <<\tm{number-or-marker-p}>>. Ce mot est d'essayer de vous
dire ce genre d'argument du \tm{+} prévu. 

Le symbole \tm{number-or-marker-p} dit que l'interprète Lisp essaie de
déterminer si l'information a présenté (la valeur de l'argument) est
un nombre ou un marqueur (un objet spécial représentant une position
de tampon). Ce qu'il fait est le test pour voir si le \tm{+} est étant
des nombres à ajouter. Il teste aussi de voir si l'argument est
quelque chose appelé un marqueur, qui est une caractéristique
spécifique de Emacs Lisp. (Dans Emacs, les emplacements dans une
mémoire tampon sont enregistrés en tant que marqueurs. Lorsque la
marque est réglée avec le \rep{C}{@} ou la commande \rep{C}{SPC}, sa
position est maintenue en tant que marqueur. La marque peut être
considérée comme un nombre---le nombre de caractères les emplacement
est à partir du début de la mémoire tampon). En Lisp, \tm{+} peut être
utilisé pour ajouter la valeur numérique de positions de marqueurs que
des nombres.

Le <<\tm{p}>> de <<\tm{number-or-marker-p}>> est l'incarnation d'une
pratique a commencé dans les premiers jours de la programmation
Lisp. Le <<\tm{p}>> signifie <<prédicat>>. Dans le jargon, utilisé par
les chercheurs Lisp, un prédicat renvoie à une fonction pour
déterminer si une propriété est vraie ou fausse. Ainsi, le <<\tm{p}>>
nous dit que ce \tm{number-or-marker-p} est le nom d'une fonction qui
détermine si c'est vrai ou faux que l'argument fourni est un nombre ou
un marqueur. Autres symboles Lisp qui se terminent par <<\tm{p}>>
comprennent \tm{zerop}, une fonction qui teste si son argument a la
valeur de zéro, et \tm{listp}, une fonction qui teste si son argument
est une liste. 

Enfin, la dernière partie du message d'erreur est le symbole
\tm{hello}. C'est la valeur de l'argument qui a été adopté pour
\tm{+}. Si l'addition avait été adoptée le bon type d'objet, la valeur
passée aurait été un certain nombre, comme 37, plutôt qu'un symbole
comme \tm{hello}. Mais alors vous n'auriez pas reçu le message d'erreur.

\subsection{La fonction \tm{message}}\etchss{1}{8}{5}

Comme \texttt{+}, la fonction de \texttt{message} prend un nombre
variable d'arguments. Elle est utilisée pour envoyer des messages à
l'utilisateur et est si utile que nous allons la décrire ici.

Un message est imprimé dans la zone écho. Par exemple, vous pouvez
imprimer un message dans votre zone de répercussion en évaluant la
liste suivante :   

\tm{(message ``This message appears in the echo area'')}

L'ensemble de la chaîne de caractères entre guillemets est un argument
unique et est imprimé dans sa totalité. (Notez que dans cet exemple,
le message lui-même apparaît dans la zone écho entre guillemets; cela
parce que vous voyez la valeur retournée par la fonction
\tm{message}. Dans la plupart des utilisations de \tm{message} dans
les programmes que vous écrivez, le texte sera imprimé dans la zone
d'écho comme un effet secondaire, sans les guillemets. Voir Section
\cfchss{3}{3}{1} ``multiply-by-seven in detail'' page
\cfchssg{3}{3}{1}, pour un exemple de cela).

Cependant, s'il y a un '\tm{\%s}' dans la chaîne de caractères entre
guillemets, la fonction \tm{message} n'imprimera pas le '\tm{\%s}' en
tant que tel, mais l'assimile à un argument qui suit la chaîne. Elle
évalue le second argument et imprime sa valeur à l'endroit où se situe
'\tm{\%s}' dans la chaîne.

Vous pouvez voir cela en positionnant le curseur après l'expression
suivante et en tapant \tm{C-x C-e}:

\tm{(message ``The name of this buffer is : \%s.'' (buffer-name))}

Dans Info, \tm{``The name of this buffer is :*info*.''} apparaîtra
dans la zone d'écho. La fonction \tm{buffer-name} renvoie  le nom du
tampon comme une chaîne, où la fonction \tm{message} insère à la place
de \tm{'\%s'}.

Pour imprimer une valeur comme un entier, utiliser \tm{'\%d'} de la
même manière que \tm{'\%s'}.  Par exemple, pour imprimer un message
dans la zone écho qui indique la valeur de \tm{fill-column}, évaluer les
points suivants:

\tm{(message ``The value of fill-column is: \%d.'' fill-column)}

Sur mon système, lorsque j'évalue cette list, \tm{``The value of
  fill-column is 72.''} apparaît dans ma zone d'écho.\footnote{En fait
    vous pouvez utiliser \tm{'\%s'} pour afficher un nombre. Ce n'est
    pas spécifique. \tm{\%d} n'affiche que la partie à gauche de la
    virgule d'un nombre décimal, et rien d'autre qui ne soit un
    nombre.}

S'il y a plus qu'un \tm{'\%s'} dans la chaîne entre les quotes, la
valeur du premier argument est affichée à l'endroit du premier
\tm{'\%s'}, la valeur du deuxième argument est affichée à l'endroit du
deuxième \tm{'\%s'} et ansi de suite.

Par exemple, si vous faites l'évaluation suivante : 

\tm{(message ``There are \%d \%s in the office!''}

\hspace{1.5cm}\tm{(- fill-column 14) ``pink elephants'')}

un message plutôt lunatique apparaîtra dans votre zone de
répercussion. Sur mon système, il dit, 

\tm{``The are 58 pink elephants in the office!''}.

L'expression \tm{(- fill-column 14)} est évaluée et le nombre
résultant est inséré à la place de \tm{'\%d'}; et la chaîne de
caractères \tm{``pink elephants''}, est traitée comme un argument
simple et insérée à la place de \tm{'\%s'} :

{\tt 
  \begin{center}
    \begin{tabular}[m]{*{3}l}
      (message &''He saw \%d \%s''  & \\
               & (- fill-column 32) & \\
               & (concat ``red ``   & \\
               &                    & (substring \\
               &                    & ``The quick brown foxes
                                      jumped.'' 16 21) \\
               &                    & `` leaping.''))
    \end{tabular}
  \end{center}
 }

Dans cet exemple, la fonction \tm{message} a trois arguments : la
chaîne \tm{``He saw \%d \%s''}, l'expression \tm{(- fill-column 32)}
est insérée à la place de \tm{'\%d'}; et la valeur renvoyée par l'expression
commençant par \tm{concat} est insérée à la place de \tm{'\%s'}. 

Lorsque votre colonne de remplissage est de 70 et que vous évaluer
l'expression, le message \tm{``He saw 38 red foxes leaping.''} aparaît
dans votre zone d'écho.



\section{Vérité et fausseté dans Emacs Lisp}\etchs{3}{9}

\section{\texttt{save-excursion}}\etchs{3}{10}

\subsection{Modèle pour une expression
  \texttt{save-excursion}}\etchss{3}{10}{1} 


\section{Exercices}\etchs{1}{11}

Quelques exercices simples :
\begin{itemize}
\item Générer un message d'erreur en évaluant un symbole approprié qui
  n'est pas entre parenthèses. (Par exemple évaluer \tm{+})
\item Générer un message d'erreur en évaluant un symbole approprié
  entre parenthèses. (Par exemple évaluer \tm{(+ 2 bonjour)})
\item Créer un compteur qui incrémente par deux au lieu de un. (Par
  exemple 

\tm{(setq counter 0)}

\tm{setq counter (+ counter 2)}

\tm{counter}
\item \'Ecrire une expression qui affiche un message dans la zone
  d'écho quand il est évalué. (Par exemple \tm{(message ``Ceci est un
    message dans le tampon : \%s.'' (buffer-name))}
\end{itemize}


\section{Exercices}\etchs{3}{12}



