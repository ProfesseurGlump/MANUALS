\section{Initialisation de variable}\etchs{1}{9}

Il y a plusieurs façons d'affecter une variable. Une des façons est
d'utiliser la fonction \tm{set} ou \tm{setq}. Une autre façon est
d'utiliser la fonction \tm{let} (voir Section \cfchs{3}{6} ``let''
page \cfchsg{3}{6}). (Le jargon pour ce processus est de \textit{lier}
une variable à une valeur.)

Les sections suivantes ne décrivent pas seulement le fonctionnement de
\tm{set} et \tm{setq} mais illustrent aussi comment les arguments sont
passés. 

\subsection{Utilisation de \tm{set}}\etchss{1}{9}{1}

Pour affecter la valeur du symbole \tm{flowers} à la liste \tm{'(rose
  violet daisy buttercup)}, évaluer l'expression suivante en
positionnnant le curseur après l'expression et en tappant le raccourci
suivant : \tm{C-x  C-e}.

\tm{(set 'flowers '(rose violet daisy buttercup))}

Cette liste \tm{(rose violet daisy buttercup)} apparaîtra dans la zone
d'écho. C'est ce qui est \textit{renvoyé} par la fonction
\tm{set}. Comme un effet secondaire, le symbole \tm{flowers} est lié à
la liste; qui est, le symbole \tm{flowers} qui peut être considéré
comme une variable, est donné à la liste comme sa valeur. (Ce
processus, par la manière, illustre comment un effet secondaire à
l'interprète Lisp, réglage de la valeur, peut être l'effet principal
qui nous intéresse nous les humains. C'est parce que chaque
fonction Lisp doit renvoyer une valeur si elle ne reçoit pas une
erreur, mais elle aura seulement un effet secondaire si elle est conçue
pour avoir un.) 

Après l'évaluation de l'expression \tm{set}, vous pouvez évaluer le
symbole \tm{flowers} et cela vous renverra la valeur que vous venez de
régler. Voici le symbole. Placez le curseur après lui et tapez la
séquence de touche \tm{C-x C-e}.

\tm{flowers}

Quand vous évaluez \tm{flowers}, la liste \tm{(rose violet daisy
  buttercup)} apparaît dans la zone d'écho. 

Incidemment, si vous évaluez \tm{'flowers}, la variable avec une quote
devant elle, ce que vous verrez dans la zone d'écho c'est le symbole
lui-même. Voici le symbole quoté, donc vous pouvez essayer ça :

\tm{'flowers}

Notez aussi, que lorsque vous utilisez aussi \tm{set}, vous avez
besoin de quotes pour chaque argument de \tm{set}, à moins que vous ne
souhaitiez qu'ils soient évalués. Puisque nous ne voulons pas non plus
évalué les arguments ni de la variable \tm{flowers} ni de la liste
\tm{(rose violet daisy buttercup)}, les deux sont quotés. (Lorsque
vous utilisez \tm{set} sans quote pour son premier argument, le
premier argument est evalué avant que n'importe quoi d'autre soit
fait. Si vous aviez fait ça et que \tm{flowers} n'avait pas encore de
valeur, vous auriez reçu un message d'erreur disant que \tm{'Symbol's
  value as variable is void'}; d'autre part, si \tm{flowers} avait
renvoyé une valeur après qu'elle ait été évalué, le \tm{set} tenterait
de régler la valeur qui a été renvoyé. Il y a des situations où c'est
la bonne chose à faire pour la fonction; mais de telles situations
sont rares.)

\subsection{Utilisation de \tm{setq}}\etchss{1}{9}{2}

En pratique, vous mettez une quote presque toujours devant le premier
argument de \tm{set}. La combinaison de \tm{set} et d'une quote pour
le premier argument et si fréquente qu'elle a son propre nom: la forme
spéciale \tm{setq}. Cette forme spéciale est comme \tm{set} sauf que
le premier argument a automatiquement une quote, donc vous n'avez pas
besoin de tapez une quote vous-même. En outre, par commodité,
\tm{setq} vous permez de régler plusieurs variables différentes avec
différentes valeurs, toutes en une expression.

Pour affecter la variable \tm{carnivores} de la liste \tm{'(lion tiger
  leopard)} en utilisant \tm{setq}, l'expression suivante est utilisée
:

\tm{(setq carnivores '(lion tiger leopard))}

Avec \tm{set}, l'expression ressemblerait à ça :

\tm{(set 'carnivores '(lion tiger leopard))}

En outre, \tm{setq} peut être utlisée pour assigner différentes
valeurs à différentes variables. Le premier argument est lié à la
valeur du second argument, le troisième est lié à la valeur du
quatrième, et ainsi de suite. Par exemple, vous pouvez utiliser ce qui
suit pour affecter une liste d'arbres au symbole \tm{trees} et une
liste d'herbivores au symbole \tm{herbivores} : 

\tm{(setq trees '(pine fir oak maple)}

\tm{herbivores '(gazelle antelope zebra))}

(L'expression aurait aussi bien pu être sur une seule ligne, mais ça
n'aurait pas été bien mis en page; et les humains trouvent qu'il est
plus simple de lire les listes qui sont bien mises en page.)

Même si j'utilise le terme 'assigner', il y a une autre façon de
penser le travail de \tm{set} et \tm{setq}; et c'est de dire que
\tm{set} et \tm{setq} mettent le symbole \textit{point} dans la
liste. Cette dernière façon de penser est très communes et dans les
chapitres à venir nous verrons au moins un symbole qui a 'pointer'
comme partie de son nom. Le nome est choisi parcee le symbole a une
valeur, spécifiquement une liste, attachée à lui; ou, dit autrement,
le symbole est réglé pour <<pointer>> sur la liste.

\subsection{Compteurs}\etchss{1}{9}{3}

Voici un exemple qui montre comment utiliser \tm{setq} dans un
compteur. Vous devez utiliser cela pour compter combien de fois une
partie de votre programme se répète. D'abord initialiser une variable
à zéro; ensuite ajouter un au nombre chaque fois que le programme se
répète. Pour faire cela, vous avez besoin d'une variable qui sert à
compter, et de deux expressions: une expression \tm{setq} initiale qui
affecte le compteur à zéro; et une seconde expression \tm{setq} qui
incrémente le compteur chaque fois qu'elle est évaluée.

\begin{center}
  \begin{tabular}[m]{ll}
    \tm{(setq counter 0)} & ; appelons-le l'initialisateur  \\
    \tm{(setq counter (+ counter 1))} & ; c'est l'incrémenteur \\
    \tm{counter}                     & ; c'est le compteur
  \end{tabular}
\end{center}

(Le texte suivant le ';' sont des commentaires. Voir Section
\cfchss{3}{2}{1} ``Change a Function Definition'', page
\cfchssg{3}{2}{1}.)

Si vous évaluez la première de ces expressions, l'initialisateur,
\tm{(setq counter 0)}, et ensuite évaluez la troisième expression,
\tm{counter}, le nombre 0 apparaît dans la zone d'écho. Si vous
évaluez ensuite la seconde expression, l'incrémenteur, \tm{(setq
  counter (+ counter 1))}, le compteur aura la valeur 1. Donc si vous
évaluez \tm{counter}, le nombre 1 apparaîtra dans la zone
d'écho. Chaque fois que évaluez la seconde expression, la valeur du
compteur sera incrémentée.

Quand vous évaluez l'incrémenteur, \tm{(setq counter (+ counter 1))},
l'interprète Lisp évalue d'abord la liste la plus interne; c'est
l'addition. Afin d'évaluer cette liste, il doit évaluer la variable
\tm{counter} et le nombre 1. Quand il évalue la variable \tm{counter},
il reçoit sa valeur courante. Il passe cette valeur et le nombre 1 à
la fonction \tm{+} qui les ajoute ensemble. La somme est ensuite
renvoyée à la valeur de la liste qui la contient et passée à la
fonction \tm{setq} qui affecte la variable \tm{counter} de cette
nouvelle valeur. Ainsi la valeur de la variable, \tm{counter}, est modifiée.

