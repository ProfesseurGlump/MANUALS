\section{Noms de symboles et noms de fonctions}\etchs{1}{4}

Nous pouvons exprimer une caractéristique de Lisp basée sur ce que
nous avons discuté jusqu'à présent---une caractéristique importante :
un symbole, comme \tm{+}, n'est pas lui-même l'ensemble des
instructions de l'ordinateur nécessaire pour mener à bien sa
fonction. Au contraire, le symbole est utilisé, peut-être
temporairement, comme un moyen de localisation de la définition ou un
ensemble d'instructions. Ce que nous voyons est le nom par lequel les
instructions peuvent être trouvées. Les noms de personnes fonctionnent de
la même manière. Je peux être <<\tm{Bob}>> ; cependant, je ne suis pas
les lettres <<\tm{B}>>, <<\tm{o}>>, <<\tm{b}>>, mais je suis ou étais,
la conscience toujours associée à une forme de vie particulière. Le
nom n'est pas moi, mais il peut être utilisé pour se référer à moi. 

En Lisp, un ensemble d'instructions peut être attaché à plusieurs
noms. Par exemple, les instructions informatiques pour l'ajout de
nombres peuvent être liées au symbole, plus ainsi que pour le symbole
\tm{+} (et sont dans certains dialectes de Lisp). Chez les humains, je
peux être appelé <<\tm{Robert}>> ainsi que <<\tm{Bob}>> et par
d'autres mots ainsi. 

D'autre part, un symbole peut avoir seulement une définition de la
fonction attachée à elle à la fois. Sinon, l'ordinateur serait confus
quant à la définition à utiliser. Si c'était le cas chez les
personnes, une seule personne dans le monde pourrait être nommé
<<\tm{Bob}>>. Cependant, la définition de la fonction à laquelle le
nom se réfère peut être modifié facilement. (Voir la
section\cfchs{3}{2} <<Installer une définition de fonction>>,
page\cfchsg{3}{2}.) 

Depuis Emacs Lisp est grand, il est d'usage de symboles nom d'une
manière qui identifie la partie d'Emacs auquel la fonction
appartient. Ainsi, tous les noms de fonctions qui traitent de Texinfo
commencent par <<\tm{texinfo-}>> et ceux pour les fonctions qui
traitent de la lecture du courrier de départ avec <<\tm{rmail-}>>.


