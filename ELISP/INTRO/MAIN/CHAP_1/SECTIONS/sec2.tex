\section{Lancer un programme}\etchs{1}{2}

Une liste en Lisp---toute---liste est un programme prêt à
fonctionner. Si vous l'exécutez (en jargon Lisp on dit évaluer),
l'ordinateur fera l'une des trois choses: ne rien faire, sauf renvoyer
la liste elle-même; vous envoyer un message d'erreur; ou traiter le
premier symbole dans la liste comme une commande pour faire quelque
chose. (Habituellement, bien sûr, c'est le dernier cas que vous voulez
vraiment!) 

L'apostrophe (ou guillemet simple), ', que j'ai mise en face de quelques-uns des
exemples de listes dans les sections précédentes est appelé
quote\footnote{En anglais dans le texte car la traduction <<devis>> ne
  m'a pas paru pertinente. De plus quote est bien plus court à écrire
  que guillemet simple ou apostrophe d'autant qu'il ne s'agit pas ici
  d'un symbole de ponctuation mais d'un symbole propre au langage
  Lisp.}; quand il précède une liste, il dit à Lisp 
de ne rien faire avec la liste, autre que de la prendre comme elle est
écrite. Mais s'il n'y a pas de quote précédent une liste, le premier
élément de la liste est spécial: c'est une commande à laquelle
l'ordinateur doit obéir. (En Lisp, ces commandes sont appelées
fonctions.) La liste \tm{(+ 2 2)} ci-dessus n'a pas de quote la
précédant, de sorte que Lisp comprend que le \tm{+} est une
instruction pour faire quelque chose avec le reste de la liste:
ajouter les chiffres qui suivent. 

Si vous lisez ce manuel à depuis \gem dans Info, voici comment
vous pouvez évaluer une telle liste : placez votre curseur
immédiatement après la parenthèse de droite de la liste suivante et
puis tapez \rec{C}{x}{C}{e}:
\begin{center}
  \tm{(+ 2 2)}
\end{center}

Vous verrez que le nombre 4 appraît dans la zone écho. (Dans le jargon, ce
que vous venez de faire est ``d'évaluer la liste.'' La zone d'écho est
la ligne en bas de l'écran qui affiche (ou ``fait échos'' au) le texte.)
Maintenant, essayez la même chose avec une liste avec quote: placer le
curseur après la liste suivante et tapez \rec{C}{x}{C}{e}:
\begin{center}
  \tm{'(this is a quoted list)}
\end{center}

Vous verrez \tm{(this is a quoted list)} apparaître dans la zone
d'écho.

Dans les deux cas, ce que vous faites est de donner un ordre à
l'intérieur du programme \gem appelé l'interprète Lisp---lui donnant
une commande pour évaluer l'expression Lisp. Le nom de l'interprète
Lisp vient du mot pour la tâche accomplie par un être humain qui
traduit une langue en une autre.

Vous pouvez également évaluer un atome qui ne fait pas partie d'une
liste et un qui n'est pas entouré par des parenthèses; encore une
fois, l'interprète Lisp traduit de l'expression lisible par l'homme vers
la langue de l'ordinateur. Mais avant de discuter de cela (voir
section\cfchs{1}{7} ``Variables'', page\cfchsg{1}{7}), nous allons
discuter de ce que l'interprète Lisp fait quand vous faites une
erreur. 

