\section{Arguments}\etchs{1}{8}

Pour voir comment l'information est transmise à des fonctions,
regardons de nouveau notre vieille veille, l'ajout de deux plus
deux. En Lisp, ça s'écrit comme suit :
\begin{center}
  \tm{(+ 2 2)}
\end{center}

Si vous évaluez cette expression, le nombre 4 apparaîtra dans votre
zone écho. Ce que fait l'interprète Lisp c'est d'ajouter les nombres
qui suivent le \tm{+}.

Les nombres ajoutés par \tm{+} sont appelés les arguments de la
fonction \tm{+}. Ces nombres sont les informations que l'on donne à ou
transmis à la fonction. 

Le mot <<argument>> vient de la façon dont il est utilisé en
mathématiques et ne se réfère pas à une dispute entre deux personnes;
au contraire, il se réfère à l'information présentée à la fonction,
dans ce cas, à la fonction \tm{+}. En Lisp, les arguments d'une
fonction sont les atomes ou les listes qui suivent la fonction. Les
valeurs renvoyées par l'évaluation de ces atomes ou des listes sont
passées à la fonction. Différentes fonctions nécessitent différents
nombres d'arguments; certaines fonctions n'en exigent pas du
tout\footnote{Il est curieux de suivre le chemin par lequel le mot
  <<argument>> est venu d'avoir deux significations différentes, l'une
  en mathématiques et l'autre en anglais de tous les jours. Selon le
  Oxford English Dictionary, le mot dérive du Latin <<\tm{faire
    comprendre, prouver}>>; ainsi venu à signifier, par un fil de
  dérivation, <<la preuve présentée comme preuve>>, c'est-à-dire,
  <<l'information offerte>>, qui a conduit à sa signification en
  Lisp. Mais dans l'autre fil de dérivation, il est venu à signifier
  <<d'affirmer d'une manière contre laquelle d'autres peuvent faire
  des contre affirmations>>, qui ont conduit à la signification du mot
  comme une dispute. (Notons ici que le mot anglais a deux définitions
  différentes qui s'y rattachent en même temps. En revanche, dans
  Emacs Lisp, un symbole ne peut pas avoir deux définitions de
  fonctions différentes en même temps.)}. 

\subsection{Arguments des types de données}\etchss{1}{8}{1}

Le type de données qui doivent être transmises à une fonction dépend
de quel type d'information elle utilise. Les arguments d'une fonction
telle que \tm{+} doivent avoir des valeurs qui sont des nombres,
depuis \tm{+} ajoute des nombres. D'autres fonctions utilisent
différents types de données pour leurs arguments.

Par exemple, la fonction \tm{concat} relie ou réunit deux ou plusieurs
chaînes de caractères pour produire une chaîne. Les arguments sont des
chaînes. La concaténation des deux chaînes de caractères \tm{abc},
\tm{def} produit la chaîne \tm{abcdef} seule. Ceci peut être vu par
l'évaluation de ce qui suit : 
\begin{center}
  \tm{(concat 'abc' 'def')} % ne marche pas en TeX
\end{center}

La valeur produite en évaluant cette expression est <<\tm{abcdef}>>.

Une fonction comme sous-chaîne utilise à la fois une chaîne et des
nombres comme arguments. La fonction renvoie une partie de la chaîne,
une chaîne du premier argument. Cette fonction prend trois
arguments. Son premier argument est la chaîne de caractères, les
deuxième et troisième arguments sont des nombres qui indiquent le
début et la fin de la sous-chaîne. Les nombres sont un comptage du
nombre de caractères (ponctuation et espaces compris) depuis le début
de la chaîne.

Par exemple, si vous évaluez la suivante :
\begin{center}
  \tm{(substring ''The quick brown fox jumped.'' 16 19)}
\end{center}

vous verrez ''\tm{fox}'' apparaître dans la zone d'écho. Les arguments
sont la chaîne et les deux nombres. 

Notez que la chaîne passée à \tm{substring} est un atome, même si elle
est composée de plusieurs mots séparés par des espaces. Lisp compte
tout entre les deux guillemets dans le cadre de la chaîne, y compris
les espaces. Vous pouvez penser la fonction \tm{substring} comme une
sorte <<d'écraseur d'atome>>, car il faut un atome contraire à
l'indivisible et en extraire une partie. Cependant, \tm{substring} est
seulement capable d'extraire une sous-chaîne à partir d'un argument
qui est une chaîne, pas d'un autre type d'atome comme un nombre ou un
symbole.



\subsection{Un argument comme la valeur d'une variable ou d'une
  liste}\etchss{1}{8}{2}

Un argument peut être un symbole qui renvoie une valeur quand il est
évalué. Par exemple, lorsque le symbole de remplissage par colonne
lui-même est évalué, il renvoie un nombre. Ce nombre peut être utilisé
pour une addition. Positionner le curseur après l'expression suivante
et le type \rec{C}{x}{C}{e}:
\begin{center}
  \tm{(+ 2 fill-column)}
\end{center}

La valeur sera un nombre deux de plus que ce que vous obtenez en
évaluant \tm{fill-column} seul. Pour moi, c'est 74, parce que ma
valeur de \tm{fill-column} est 72.

Comme nous venons de le voir, un argument peut être un symbole qui
renvoie une valeur lors de l'évaluation. En outre, un argument peut
être une liste qui renvoie une valeur quand il est évalué. Par
exemple, dans l'expression suivante, les arguments de la fonction
\tm{concat} sont les chaînes <<The>> et <<red foxes.>> et la liste
\tm{(number-to-string (+ 2 fill-column))}.
\begin{center}
  \tm{(concat ''The '' (number-to-string (+ 2 fill-column)) '' red foxes.'')}
\end{center}

Si vous évaluez cette expression et si, comme avec mon Emacs,
\tm{fill-column} évalue à 72---<<The 74 red foxes.>> apparaît dans la
zone écho. (Notez que vous devez mettre des espaces après le mot
<<The>> et avant le mot <<red>> de sorte qu'ils apparaissent dans la
chaîne finale. La fonction \tm{number-to-string} convertit l'entier
que la fonction d'addition renvoie à une chaîne. \tm{number-to-string}
est également connu comme \tm{int-to-string}.)


\subsection{Nombre variable d'arguments}\etchss{1}{8}{3}

Certaines fonctions, comme \tm{concat}, \tm{+} ou \tm{*}, prendre
n'importe quel nombre d'arguments. (Le \tm{*} est le symbole de
multiplication.) Ceci peut être vu par l'évaluation de chacune des
expressions suivantes de la manière habituelle. Ce que vous verrez
dans la zone d'écho est imprimé dans ce texte après <<$\Rightarrow$>>,
que vous pouvez lire comme <<évalué à >>.

Dans la première série, les fonctions n'ont pas d'arguments:
\begin{center}
  \begin{tabular}[m]{lrl}
    \tm{(+)} &$\Rightarrow$ & 0 \\
    \tm{(*)} &$\Rightarrow$ & 1 
  \end{tabular}
\end{center}

Dans cet ensemble, les fonctions ont un argument de chaque:
\begin{center}
  \begin{tabular}[m]{lrl}
    \tm{(+ 3)} &$\Rightarrow$ & 3 \\
    \tm{(* 3)} &$\Rightarrow$ & 3 
  \end{tabular}
\end{center}

Dans cet ensemble, les fonctions ont trois arguments de chaque:
\begin{center}
  \begin{tabular}[m]{lrl}
    \tm{(+ 3 4 5)} &$\Rightarrow$ & 12 \\
    \tm{(* 3 4 5)} &$\Rightarrow$ & 60 
  \end{tabular}
\end{center}


\subsection{Utiliser le mauvais type d'objet pour un argument}\etchss{1}{8}{4}

Quand une fonction est passée un argument de type incorrect,
l'interprète Lisp produit un message d'erreur. Par exemple, la
fonction \tm{+} attend les valeurs de ses arguments pour être des
nombres. Comme une expérience, nous pouvons passer le symbole cité
bonjour au lieu d'un nombre. Positionner le curseur après l'expression
suivante et le type \rec{C}{x}{C}{e} :
\begin{center}
  \tm{(+ 2 'hello)}
\end{center}

Lorsque vous faites cela, vous allez générer un message d'erreur. Ce
qui est arrivé est que \tm{+} a essayé d'ajouter le 2 à la valeur
renvoyée par \tm{'hello}, mais la valeur renvoyée par \tm{'hello} est
le symbole \tm{hello}, pas un nombre. Seuls des chiffres peuvent être
ajoutés. Donc \tm{+} ne pouvait pas mener à bien son addition. 

Vous aller créer et entrer dans un tampon \tm{*Backtrace*} qui dit :

{\ttfamily
Debugger entered--Lisp error:
\begin{center}
  (wrong-type-argument number-or-marker-p hello)
\end{center}
+(2 hello)

eval((+ 2 (quote hello)))

eval-last-sexp-1(nil)

eval-last-sexp(nil)

call-interactively(eval-last-sexp)
}

Comme d'habitude, le message d'erreur essaie d'être utile et logique
après vous apprenez à lire\footnote{\tm{(quote hello)} est une
  expansion de l'abréviation \tm{'hello}}. 

La première partie du message d'erreur est simple; il est dit
<<\tm{wrong type argument}>>. Vient ensuite le mot de jargon
mystérieux <<\tm{number-or-marker-p}>>. Ce mot est d'essayer de vous
dire ce genre d'argument du \tm{+} prévu. 

Le symbole \tm{number-or-marker-p} dit que l'interprète Lisp essaie de
déterminer si l'information a présenté (la valeur de l'argument) est
un nombre ou un marqueur (un objet spécial représentant une position
de tampon). Ce qu'il fait est le test pour voir si le \tm{+} est étant
des nombres à ajouter. Il teste aussi de voir si l'argument est
quelque chose appelé un marqueur, qui est une caractéristique
spécifique de Emacs Lisp. (Dans Emacs, les emplacements dans une
mémoire tampon sont enregistrés en tant que marqueurs. Lorsque la
marque est réglée avec le \rep{C}{@} ou la commande \rep{C}{SPC}, sa
position est maintenue en tant que marqueur. La marque peut être
considérée comme un nombre---le nombre de caractères les emplacement
est à partir du début de la mémoire tampon). En Lisp, \tm{+} peut être
utilisé pour ajouter la valeur numérique de positions de marqueurs que
des nombres.

Le <<\tm{p}>> de <<\tm{number-or-marker-p}>> est l'incarnation d'une
pratique a commencé dans les premiers jours de la programmation
Lisp. Le <<\tm{p}>> signifie <<prédicat>>. Dans le jargon, utilisé par
les chercheurs Lisp, un prédicat renvoie à une fonction pour
déterminer si une propriété est vraie ou fausse. Ainsi, le <<\tm{p}>>
nous dit que ce \tm{number-or-marker-p} est le nom d'une fonction qui
détermine si c'est vrai ou faux que l'argument fourni est un nombre ou
un marqueur. Autres symboles Lisp qui se terminent par <<\tm{p}>>
comprennent \tm{zerop}, une fonction qui teste si son argument a la
valeur de zéro, et \tm{listp}, une fonction qui teste si son argument
est une liste. 

Enfin, la dernière partie du message d'erreur est le symbole
\tm{hello}. C'est la valeur de l'argument qui a été adopté pour
\tm{+}. Si l'addition avait été adoptée le bon type d'objet, la valeur
passée aurait été un certain nombre, comme 37, plutôt qu'un symbole
comme \tm{hello}. Mais alors vous n'auriez pas reçu le message d'erreur.

\subsection{La fonction \tm{message}}\etchss{1}{8}{5}

Comme \texttt{+}, la fonction de \texttt{message} prend un nombre
variable d'arguments. Elle est utilisée pour envoyer des messages à
l'utilisateur et est si utile que nous allons la décrire ici.

Un message est imprimé dans la zone écho. Par exemple, vous pouvez
imprimer un message dans votre zone de répercussion en évaluant la
liste suivante :   

\tm{(message ``This message appears in the echo area'')}

L'ensemble de la chaîne de caractères entre guillemets est un argument
unique et est imprimé dans sa totalité. (Notez que dans cet exemple,
le message lui-même apparaît dans la zone écho entre guillemets; cela
parce que vous voyez la valeur retournée par la fonction
\tm{message}. Dans la plupart des utilisations de \tm{message} dans
les programmes que vous écrivez, le texte sera imprimé dans la zone
d'écho comme un effet secondaire, sans les guillemets. Voir Section
\cfchss{3}{3}{1} ``multiply-by-seven in detail'' page
\cfchssg{3}{3}{1}, pour un exemple de cela).

Cependant, s'il y a un '\tm{\%s}' dans la chaîne de caractères entre
guillemets, la fonction \tm{message} n'imprimera pas le '\tm{\%s}' en
tant que tel, mais l'assimile à un argument qui suit la chaîne. Elle
évalue le second argument et imprime sa valeur à l'endroit où se situe
'\tm{\%s}' dans la chaîne.

Vous pouvez voir cela en positionnant le curseur après l'expression
suivante et en tapant \tm{C-x C-e}:

\tm{(message ``The name of this buffer is : \%s.'' (buffer-name))}

Dans Info, \tm{``The name of this buffer is :*info*.''} apparaîtra
dans la zone d'écho. La fonction \tm{buffer-name} renvoie  le nom du
tampon comme une chaîne, où la fonction \tm{message} insère à la place
de \tm{'\%s'}.

Pour imprimer une valeur comme un entier, utiliser \tm{'\%d'} de la
même manière que \tm{'\%s'}.  Par exemple, pour imprimer un message
dans la zone écho qui indique la valeur de \tm{fill-column}, évaluer les
points suivants:

\tm{(message ``The value of fill-column is: \%d.'' fill-column)}

Sur mon système, lorsque j'évalue cette list, \tm{``The value of
  fill-column is 72.''} apparaît dans ma zone d'écho.\footnote{En fait
    vous pouvez utiliser \tm{'\%s'} pour afficher un nombre. Ce n'est
    pas spécifique. \tm{\%d} n'affiche que la partie à gauche de la
    virgule d'un nombre décimal, et rien d'autre qui ne soit un
    nombre.}

S'il y a plus qu'un \tm{'\%s'} dans la chaîne entre les quotes, la
valeur du premier argument est affichée à l'endroit du premier
\tm{'\%s'}, la valeur du deuxième argument est affichée à l'endroit du
deuxième \tm{'\%s'} et ansi de suite.

Par exemple, si vous faites l'évaluation suivante : 

\tm{(message ``There are \%d \%s in the office!''}

\hspace{1.5cm}\tm{(- fill-column 14) ``pink elephants'')}

un message plutôt lunatique apparaîtra dans votre zone de
répercussion. Sur mon système, il dit, 

\tm{``The are 58 pink elephants in the office!''}.

L'expression \tm{(- fill-column 14)} est évaluée et le nombre
résultant est inséré à la place de \tm{'\%d'}; et la chaîne de
caractères \tm{``pink elephants''}, est traitée comme un argument
simple et insérée à la place de \tm{'\%s'} :

{\tt 
  \begin{center}
    \begin{tabular}[m]{*{3}l}
      (message &''He saw \%d \%s''  & \\
               & (- fill-column 32) & \\
               & (concat ``red ``   & \\
               &                    & (substring \\
               &                    & ``The quick brown foxes
                                      jumped.'' 16 21) \\
               &                    & `` leaping.''))
    \end{tabular}
  \end{center}
 }

Dans cet exemple, la fonction \tm{message} a trois arguments : la
chaîne \tm{``He saw \%d \%s''}, l'expression \tm{(- fill-column 32)}
est insérée à la place de \tm{'\%d'}; et la valeur renvoyée par l'expression
commençant par \tm{concat} est insérée à la place de \tm{'\%s'}. 

Lorsque votre colonne de remplissage est de 70 et que vous évaluer
l'expression, le message \tm{``He saw 38 red foxes leaping.''} aparaît
dans votre zone d'écho.

