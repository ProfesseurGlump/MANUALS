\section{Interpréteur Lisp}\etchs{1}{5}

Sur la base de ce que nous avons vu, nous pouvons maintenant commencer
à comprendre ce que l'interprète Lisp fait lorsque nous commandons à
évaluer une liste. Tout d'abord, il semble pour voir s'il y a une
quote avant que la liste ; s'il y a, l'interprète nous donne juste
la liste. D'un autre côté, s'il n'y a pas de quote, l'interprète
regarde le premier élément de la liste et voit si elle a une
définition de fonction. Si c'est le cas, l'interpréteur exécute les
instructions dans la définition de fonction. Sinon, l'interpréteur
affiche un message d'erreur. 

C'est ainsi que fonctionne Lisp simple. Il y a des complications
supplémentaires que nous allons obtenir dans une minute, mais ce sont
les fondamentaux. Bien sûr, pour écrire des programmes Lisp, vous
devez savoir comment écrire les définitions de fonctions et leur
donner des noms, et comment le faire sans confondre vous-même ou
l'ordinateur. 

Maintenant, pour la première complication, en plus de la liste,
l'interpréteur Lisp peut évaluer un symbole qui n'a pas de quote et
qui n'a pas de parenthèses autour de lui. L'interprète Lisp tentera de
déterminer la valeur du symbole comme une variable. Cette situation
est décrite dans la section sur les variables. (Voir la
section\cfchs{1}{7} <<Variables>>, page\cfchsg{1}{7}.)

La deuxième complication se produit parce que certaines fonctions sont
inhabituelles et ne fonctionnent pas de manière habituelle. Celles qui
ne le font pas sont appelées formes spéciales. Elles sont utilisées
pour des tâches spéciales, comme la définition d'une fonction, et il
n'y en a pas beaucoup. Dans les prochains chapitres, vous serez initié
à plusieurs de ces formes spéciales les plus importantes. 

La troisième et dernière complication est la suivante: si la fonction
que l'interprète Lisp regarde n'est pas une forme particulière, et si
elle fait partie d'une liste, l'interprète Lisp regarde pour voir si
la liste a une liste à l'intérieur de celui-ci. S'il y a une liste
interne, l'interprète Lisp décide ce qu'il doit faire avec la
liste intérieur en premier, puis il travaille sur la liste
extérieure. S'il y a encore une autre liste incorporée à l'intérieur
de la liste intérieure, cela fonctionne sur celui-là en premier, et
ainsi de suite. Il travaille toujours sur la liste la plus profonde en
premier. L'interprète travaille sur la liste la plus interne d'abord,
pour évaluer le résultat de cette liste. Le résultat peut être utilisé
par l'expression l'encapsulant.

Sinon, l'interprète travaille de gauche à droite, d'une expression à
l'autre.

\subsection{Compilation d'octet}\etchss{1}{5}{1}

Un autre aspect de l'interprétation : l'interprète Lisp est capable
d'interpréter deux types d'entités : du code lisible par des humains,
sur lequel nous nous concentrerons exclusivement, et du code
spécialement traité, appelé byte code compilé, qui n'est pas lisible
par les humains. Le byte code compilé s'exécute plus rapidement que le
code lisible par les humains. 

Vous pouvez transformer un code lisible pour les humains en un byte
code compilé en exécutant l'une des commandes de compilation telles
que \tm{byte-compile-file}. Le byte code compilé est généralement
stocké dans un fichier qui se termine par une extension \tm{.elc}
plutôt qu'une extension \tm{.el}. Vous verrez deux types de fichier
dans le répertoire \tm{emacs/lisp} ; les fichiers à lire sont ceux qui
portent l'extension \tm{.el}.

En pratique, pour la plupart des choses que vous pourriez faire pour
personnaliser ou étendre Emacs, vous n'avez pas besoin d'une
compilation de byte code ; et je ne vais pas en parler ici. Voir la
section <<Byte compilation>> dans le manuel Emacs Lisp reference, pour
une description complète de la byte compilation.
 

