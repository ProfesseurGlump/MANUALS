\section{\'Evaluation}\etchs{1}{6}

Lorsque l'interpréteur Lisp fonctionne sur une expression, le terme de
l'activité est appelée évaluation. Nous disons que l'interprète
<<évalue l'expression>>. J'ai utilisé ce terme plusieurs fois
avant. Le mot vient de son utilisation dans le langage courant, <<pour
déterminer la valeur ou la quantité de ; d'évaluer>>, selon le Webster New
Collegiate Dictionnary.

Après avoir évalué une expression, l'interprète Lisp va probablement
renvoyer la valeur que l'ordinateur produit en effectuant les
instructions trouvées dans la définition de fonction, ou peut-être
qu'il donnera pour cette fonction un message d'erreur. (L'interprète
peut également se trouver jeté, pour ainsi dire, à une fonction
différente ou il peut tenter de répéter sans cesse ce qu'il fait à
tout jamais dans ce qu'on appelle une <<boucle infinie>>. Ces actions
sont moins fréquentes ; et nous pouvons les ignorer.) Le plus souvent,
l'interprète renvoie une valeur. 

Dans le même temps l'interprète renvoie une valeur, il peut faire
autre chose ainsi, comme déplacer un curseur ou copeir un fichier ;
cet autre type d'action est appelé un effet secondaire. Les actions que
nous les humains pensont être importantes, telles que les résultats
d'impression, sont souvent des <<effets secondaires>> à l'interprète
Lisp. Le jargon peut paraître étrange, mais il s'avère qu'il est assez
facile d'apprendre à utiliser des effets secondaires.

En résumé, l'évaluation d'une expression symbolique incite le plus souvent
l'interprète Lisp à renvoyer une valeur et peut être à réaliser un
effet secondaire ; ou bien produire une erreur.

\subsection{\'Evaluation des listes internes}\etchss{1}{6}{1}

Si l'évaluation s'applique à une liste qui est à l'intérieur d'une
autre liste, la liste extérieure peut utiliser la valeur renvoyée par
la première évaluation comme information lorsque la liste extérieure
sera évaluée. Cela explique pourquoi les expressions intérieures sont
évaluées en premier : les valeurs qu'elles renvoient sont utilisées
par les expressions extérieures.

Nous pouvons enquêter sur ce processus en évaluant un autre exemple de
plus. Placez votre curseur après l'expression suivante et tapez
\rec{C}{x}{C}{e} :
\begin{center}
  \tm{(+ 2 (+ 3 3))}
\end{center}

Le nombre 8 apparaîtra dans la zone d'écho.

Ce qui se passe c'est que l'interprète Lisp évalue d'abord
l'expression intérieure, \tm{(+ 3 3)}, pour laquelle la valeur
renvoyée est 6; il évalue l'expression externe comme s'il était écrit
\tm{(+ 2 6)}, qui renvoie la valeur 8. Comme il n'y a pas
d'expressions plus englobantes pour évaluer, l'interprète affiche
cette valeur dans la zone d'écho.

Maintenant, il est facile de comprendre le nom de la commande invoquée
par le raccourci \rec{C}{x}{C}{e} : le nom est
\tm{eval-last-sexp}. Les lettres \tm{sexp} sont une abréviation pour
<<expression symbolique>>, et \tm{eval} pour <<évaluer>>. La commande
signifie <<évaluer la dernière expression symbolique>>.

\`A titre expérimental, on peut essayer d'évaluer l'expression en
plaçant le curseur au début de la ligne suivante immédiatement après
l'expression, ou l'expression à l'intérieur. 

Voici une autre copie de l'expression : 
\begin{center}
  \tm{(+ 2 (+ 3 3))}
\end{center}

Si vous placez le cursuer au début de la ligne blanche et tapez
\rec{C}{x}{C}{e}, vous obtiendrez toujours la valeur 8 imprimée dans
la zone d'écho. Maintenant, essayez de mettre le curseur dans
l'expression. Si vous le mettez juste après l'avant-dernière
parenthèse (il apparaît donc de s'asseoir sur le dessus de la dernière
parenthèse), vous obtiendrez un 6 imprimé dans la zone écho! C'est
parce que la commande évalue l'expression \tm{(+ 3 3)}.

Maintenant, mettez le curseur immédiatement après un certain
nombre. Type \rec{C}{x}{C}{e} et vous obtiendrez le nombre
lui-même. En Lisp, si vous évaluez un certain nombre, vous obtenez le
nombre lui-même, c'est en cela que les nombres diffèrent des
symboles. Si vous évaluez une liste à partir d'un symbole comme
\tm{+}, vous aurez une valeur renvoyée c'est le résultat de
l'ordinateur après exécution des instructions de la définition de la
fonction associée à ce nom. Si un symbole en lui-même est évalué,
quelque chose de différent se produit, comme nous le verrons dans la
prochaine section.

