\section{Listes Lisp}\etchs{1}{1}

En Lisp, une liste ressemble à ça : \tm{'(rose violet daisy
  buttercup)}. Cette liste est précédée d'une apostrophe. Elle
pourrait être écrit comme suit, ce qui ressemble plus au type de liste
dont vous avez l'habitude:
\begin{center}
  \begin{tabular}[m]{l}
  \tm{'(rose} \\
  \tm{violet} \\
  \tm{daisy} \\
  \tm{buttercup)}
  \end{tabular}
\end{center}
Les éléments de cette liste sont les noms de quatre fleurs différentes,
séparés par des espaces et entourés par des parenthèses, comme des
fleurs dans un champ avec un mur de pierre autour d'elles.

Les listes peuvent aussi contenir des nombres, comme dans cette liste
: \tm{(+ 2 2)}. Cette liste a un signe plus, <<\tm{+}>>, suivi
par deux <<\tm{2}>>, chacun séparé par un espace.

En Lisp, les données et les programmes sont représentés de la même
façon; autrement dit, ils sont tous les deux des listes de mots, de
chiffres ou d'autres listes, séparées par des espaces et entourées par
des parenthèses. (Puisqu'un programme ressemble à des données, un
programme peut facilement servir de donnée pour un autre, ce qui est
une fonctionnalité très puissante de Lisp.) (Soit dit en passant, ces
deux remarques entre parenthèses ne sont pas des listes Lisp, car
elles contiennent des <<\tm{;}>> et <<\tm{.}>> comme signes de
ponctuation.) 

Voici une autre liste, cette fois avec une liste à l'intérieur:

\tm{'(this list has (a list inside of it))}

\newcommand{\citcom}[1]{<<\tm{#1}>>}

Les composants de cette liste sont les mots <<\tm{this}>>,
\citcom{list}, \citcom{has}, et la liste \citcom{(a list inside of
  it)}. La liste intérieure est constituée des mots \citcom{a},
\citcom{list}, \citcom{inside}, \citcom{of}, \citcom{it}.

\subsection{Atomes Lisp}\etchss{1}{1}{1}

En Lisp, ce que nous appelons mots est appelé \textit{atomes}. Ce
terme vient du sens historique du mot atome, qui signifie
<<indivisible>>. En ce qui concerne Lisp, les mots que nous avons
utilisés dans les listes ne peuvent être divisés en parties plus
petites et de même dans le cadre d'un programme ;
même avec des chiffres et des symboles de caractères unique comme
\citcom{+}. D'autre part, contrairement à un atome antique, une liste
peut être divisée en plusieurs parties.(Voir chapitre \cfch{7}
<<\tm{car cdr \& cons} Fonctions Fondamentales>>, page\cfchg{7}.)

Dans une liste, les atomes sont séparés les uns des autres par des
espaces. Ils peuvent être juste à côté d'une parenthèse. 

Techniquement parlant, une liste en Lisp consiste en une paire de
parenthèses entourant des atomes séparés par des espaces ou entourant
d'autres listes ou entourant à la fois des atomes et d'autres
listes. Une liste peut n'avoir qu'un seul atome ou ne contenir rien du
tout. Une liste avec rien du tout ressemble à ça : (), et est appelée
la \textit{liste vide}. Contrairement à n'importe quoi d'autre, une
liste vide est considérée à la fois comme un atome et comme une liste.

La représentation imprimée des atomes et listes est appelée
\textit{symbolic expressions} ou, de façon plus concise,
\textit{s-expressions}. Le mot \textit{expression} par lui-même peut
référer à la fois à la représentation imprimée, ou à la liste comme
elle est perçue dans l'ordinateur. Souvent, les gens utilisent le
terme \textit{expression} sans distinction. (Aussi, dans beaucoup de
textes, le mot \textit{forme} est utilisé comme synonyme pour
expression.)

Incidemment, les atomes qui contituent notre univers étaient nommés
 de telle façon lorsque nous pensions qu'ils étaient indivisibles. Les parties
peuvent s'éclater en atome ou peuvent se fissionner en deux parties à
peu près égales en taille. Les atomes physique ont été nommé
prématurément, avant que leur vraie nature ne soit découverte. En
Lisp, certains types d'atome, comme un tableau, peuvent être séparés en
plusieurs parties; mais le mécanisme pour faire cela est différent de
celui pour éclater une liste. Autant qu'une liste d'opération est
concernée, les atomes d'une liste sont irréductible.

Comme en Anglais, les sens de la composition des lettres d'un atome
Lisp sont différents du sens que les lettres font pour former un mot. Par
exemple, le mot pour le sud américain sloth, le '\tm{ai}', est
complètement différent des deux mots, '\tm{a}', et '\tm{i}'.

Il y a beaucoup de types d'atome dans la nature mais seulement
quelques uns en Lisp: par exemple, \textit{numbers}, comme 37, 511, ou
1729, et \textit{symbols}, comme '\tm{+}', '\tm{foo}', ou
'\tm{forward-line}'. Les mots que nous avons listés dans les
exemples au-dessus sont tous des symboles. Dans une conversation Lisp
de tous les jours, le mot ``atome'' n'est pas souvent utilisé, parce
que les programmeurs essaient d'habitude d'être plus spécifique au
sujet du type d'atome avec lesquels ils travaillent. La programmation
Lisp est beaucoup plus à propos des symboles (et parfois des nombres)
avec des lists. (Incidemment, les trois mots précédant la remarque
entre parenthèse forment un liste en Lisp, dès qu'elle contient des
atomes, qui sont dans ce cas des symboles, séparés par des espaces et
encapsulés par des parenthèses, sans acune ponctuation non-Lips.)

Le texte entre les guillemets---même des phrases ou paragraphes---est
aussi un atome. Voici un exemple : 
\begin{center}
  \tm{'(this list includes ``text between quotation marks.'')}
\end{center}

En Lisp, tous les textes entre les guillemets incluant la marque de
ponctuation et les espaces est un atome simple. Ce type d'atome est
appelé une \textit{chaîne} (pour 'chaîne de caractères') et est le
type de chose qui est utilisée pour des messages que l'ordinateur peut
afficher pour un humain. Les chaînes sont un type d'atome différent
des nombres ou des symboles et sont utilisées différemment. 

\subsection{Espaces dans les listes}\etchss{1}{1}{2}

Le nombre d'espace dans une liste ne compte pas. Du point de vue du
langage Lisp,
\begin{center}
  \begin{tabular}[m]{>{\ttfamily}l}
    '(this list\\
    \quad looks like this)
  \end{tabular}
\end{center}

est exactement identique à ça :
\begin{center}
  \tm{'(this list looks like this)}
\end{center}

Les deux exemples montrent ce qui pour Lisp est la même liste, la
liste composée des symboles '\tm{this}', '\tm{list}', '\tm{looks}',
'\tm{like}', et '\tm{this}' dans cet ordre.

Les espaces supplémentaires ou les nouvelles lignes sont faites pour
construire une liste plus lisible pour les humains. Quand Lisp lit les
expressions, il se débarrasse de tous les espaces blancs
supplémentaires (mais il doit avoir au moins un espace entre les
atomes afin de les distinguer.)

Aussi étrange que cela semble, les exemples que nous avons vu couvrent
la quasi-totalité des listes en Lisp ! Toute autre liste en Lisp
ressemble plus ou moins à l'un de ces exemples, sauf que la liste
pourrait être plus longue et plus complexe. En bref, une liste est
entre parenthèses, une chaîne est entre guillemets, un symbole
ressemble à un mot, et un nombre ressemble à un nombre. (Pour
certaines situations, des crochets, des points et quelques autres
caractères spéciaux peuvent être utilisés ; cependant, nous allons
aller très loin sans eux.)

\subsection{\gem vous aide à taper les listes}\etchss{1}{1}{3}

Lorsque vous tapez une expression Lisp dans \gem en utilisant
soit le mode d'interaction Lisp ou le mode Emacs Lisp, vous avez à
votre disposition plusieurs commandes pour formater l'expression Lisp
de sorte qu'il soit facile à lire. Par exemple, appuyer sur la
touche \TAB indente automatiquement la ligne où se trouve le curseur
avec le bon décalage. Une commande pour indenter proprement le code
dans une zone est habituellement liée à \rec{M}{C}{\textbackslash}{.}
L'indentation est conçue de sorte que vous pouvez voir quels sont les
éléments d'une liste qui appartiennent à la liste---les éléments d'une
sous-liste sont plus indentés que les éléments de la liste qui
l'encapsule.

En outre, lorsque vous tapez une parenthèse fermante, Emacs saute
momentanément le curseur sur la parenthèse ouvrante correspondante, de
sorte que vous pouvez voir laquelle est-ce. C'est très utile, car
chaque liste que vous tapez dans Lisp doit avoir sa parenthèse
fermante correspondant à celle ouvrante. (Voir la section <<mode
majeurs>> dans le Manuel \gem, pour plus d'information sur les
modes Emacs.)

