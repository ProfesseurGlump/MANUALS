\section{Générer un message d'erreur}\etchs{1}{3}

Partant de sorte que vous ne vous inquiétez pas si vous le faites
accidentellement, nous allons maintenant donner un ordre à
l'interprète Lisp qui génère un message d'erreur. C'est une activité
inoffensive; et en effet, nous allons souvent essayer de générer des
messages d'erreur intentionnellement. Une fois que vous comprenez le
jargon, les messages d'erreur peuvent être informatif. Au lieu d'être
appelé ``messages d'erreur'', ils devraient être appelés messages
``d'aide''. Ils sont comme des panneaux pour un voyageur dans un pays
étranger ; les déchiffrer peut être difficile, mais une fois compris,
ils peuvent indiquer la voie. 

Le message d'erreur est généré par un débogueur \gem intégré. Nous
allons <<entrer dans le débogueur>>. Pour sortir du débogueur taper
\tm{q}.

Ce que nous allons faire c'est d'évaluer une liste qui n'a pas de
quote la précédant et qui n'a pas de commande significative comme
premier élément. Voici une liste presque exactement la même que celle
que nous avons utilisé, mais sans la quote la précédant. Placez le
curseur juste après et taper \rec{C}{x}{C}{e}:
\begin{center}
  \tm{(this is an unquoted list)}
\end{center}

(this is an unquoted list)
Une fenêtre \tm{*Backtrace*} s'ouvrira et vous devriez voir ce qui suit:
{\ttfamily
\begin{flushleft}
  Debugger entered--Lisp error: (void-function this)
  
  (this is an unquoted list)

  eval((this is an unquoted list))

  eval-last-sexp-1(nil)

  eval-last-sexp(nil)

  call-interactively(eval-last-sexp)
\end{flushleft}}

Votre curseur sera dans cette fenêtre (vous pouvez avoir à attendre
quelques secondes avant qu'il ne devienne visible). Pour quitter le
débogueur et faire la fenêtre du débogueur s'en aller, tapez: \tm{q}.

S'il vous plaît tapez \tm{q} maintenant, alors vous devenez plus
confiant et vous pouvez sortir du débogueur. Ensuite, tapez
\rec{C}{x}{C}{e} de nouveau pour y rentrer.

Sur la base de ce que nous savons déjà, nous pouvons presque lire ce
message d'erreur.

Vous avez lu le tampon \tm{*Backtrace*} de bas en haut; il vous dit ce
que Emacs fait. Lorsque vous avez tapé \rec{C}{x}{C}{e}, vous avez
fait un appel à la commande \tm{eval-last-sexp}. \tm{eval} est une
abréviation pour <<évaluer>> et \tm{sexp} est une abréviation pour
<<expression symbolique>>. La signification de cette commande est
<<évaluer la dernière expression symbolique>>, qui est l'expression
juste avant le curseur.

Chaque ligne ci-dessus vous indique ce que l'interprète Lisp évalue
après. L'action la plus récente est en haut. Le tampon est appelé
\tm{*Backtrace*} car il vous permet de suivre Emacs en arrière.

En haut du tampon \tm{*Backtrace*}, vous voyez la ligne :
\begin{center}
  \tm{Debugger entered--Lisp error: (void-function this)}
\end{center}

L'interprète Lisp a tenté d'évaluer le premier atome de la liste, le
mot <<this>>. C'est cette action qui a généré le message d'erreur
<<void-finction this>>. 

Le message contient les mots <<void-function>> et <<this>>. Le mot
<<function>> a été mentionné une seule fois auparavant. C'est un mot
très important.

Pour nos fins, nous pouvons la définir en disant qu'une fonction est
un ensemble d'instructions données à l'ordinateur qui signale à
l'oridnateur de faire quelque chose. 

Maintenant, nous pouvons commencer à comprendre le message d'erreur:
<<\tm{void-function this}>>. La fonction (ce mot est le mot
<<\tm{this}>>) n'a pas de définition d'un ensemble d'instructions que
l'ordinateur peut mener à bien. 

Le mot un peu bizarre, <<\tm{void-function}>>, est conçu pour couvrir
la façon dont Emacs Lisp est mis en \oe{}uvre, qui est que lorsqu'un
symbole n'a pas de définition de fonction attaché à lui, la place qui
doit contenir les instructions est <<\tm{void}>>.

D'autre part, puisque nous avons pu ajouter 2 plus 2 avec succès, en
évaluant \tm{(+ 2 2)} nous pouvons en déduire que le symbole \tm{+}
doit avoir un ensemble d'instructions pour l'ordinateur qui doit obéir
et ces instructions doivent être d'ajouter les nombres qui suivent le
\tm{+}.

Il est possible de prévenir Emacs d'entrer dans le débogueur dans de
tels cas. Nous n'expliquons pas comment faire ici, mais nous allons
parler de ce à quoi le résultat ressemble, parce que vous pouvez
rencontrer une situation similaire s'il y a un bogue dans un code
Emacs que vous utilisez. Dans de tels cas, vous verrez seulement une
ligne de message d'erreur ; il apparaîtra dans la zone écho et
ressemblera à ceci:
\begin{center}
  \tm{Symbol's function definition is void: this}
\end{center}

Le message disparaît dès que vous tapez une touche, même juste pour
déplacer le curseur. Nous connaissons le sens du mot
<<\tm{Symbol}>>. Il se réfère au premier atome de la liste, le mot
<<\tm{this}>>. Le mot <<\tm{function}>> se réfère aux instructions qui
indiquent à l'ordinateur ce qu'il faut faire. (Techniquement, le
symbole indique à l'ordinateur où trouver les instructions, mais c'est
une complication que nous pouvons ignorer pour le moment.) Le message
d'erreur peut être compris: <<\tm{Symbol's function definition is
  void: this}>>. Le symbole (c'est le mot <<\tm{this}>>) manque
d'instructions pour que l'ordinateur les mènent à bien. 


