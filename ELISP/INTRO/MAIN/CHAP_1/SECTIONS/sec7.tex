\section{Variables}\etchs{1}{7}

Dans Emacs Lisp, un symbole peut avoir une valeur associée à lui tout
comme il peut avoir une définition de fonction associée. Les deux sont
différents. La définition de la fonction est un ensemble
d'instructions auquelles un ordinateur obéira. Une valeur, d'autre
part, est quelque chose, comme un nombre ou un nom, qui peut varier
(c'est pourquoi un tel symbole est appelé variable). La valeur d'un
symbole peut être n'importe quelle expression Lisp, comme un symbole,
un nombre, une liste, ou une chaîne. Un symbole qui a une valeur est
souvent appelé une variable. 

Un symbole peut avoir à la fois une définition de fonction et une
valeur fixée à lui en même temps. Ou il peut avoir l'un ou
l'autre. Les deux sont séparés. C'est un peu similaire à la façon dont
le nom Cambridge peut référer à la ville du Massachussets et avoir
quelques informations attachées au nom ainsi, comme <<grand centre de
programmation>>. 

Une autre façon de penser à ce sujet est d'imaginer un symbole comme
étant une commode. La définition de la fonction est mise dans un
tiroir, la valeur dans un autre et ainsi de suite. Ce qui est mis dans
le tiroir contenant la valeur peut être modifié sans affecter le
contenu du tiroir maintenant la définition de la fonciton, et
vice-versa. 

La variable \tm{fill-column} illustre un symbole avec une valeur
attachée à elle : dans chaque tampon \gem , ce symbole est réglé à une
valeur, généralement 72 ou 70, mais parfois à une autre valeur. Pour
trouver la valeur de ce symbole, d'évaluer par lui-même. Si vous lisez
ceci dans Info à l'intérieur de \gem , vous pouvez le faire en le
curseur après le symbole et en tapant \rec{C}{x}{C}{e} :
\begin{center}
  \tm{fill-column}
\end{center}

Après que j'ai tapé \rec{C}{x}{C}{e}, Emacs a imprimé le nombre 70
dans ma zone d'écho. c'est la valeur pour laquelle \tm{fill-column}
est réglée pour moi qui écris ceci. Il peut être différent pour vous
dans votre tampon Info. Notez que la valeur renvoyée comme une
variable est imprimée exactement de la même manière que la valeur
renvoyée par la fonction d'exécution des instructions. Du point de vue
de l'interprète Lisp, une valeur renvoyée est une valeur
renvoyée. Peu importe le genre d'expression une fois que la valeur est
connue.

Un symbole peut avoir n'importe quelle valeur attachée à lui ou, pour
utiliser le jargon, on peut lier la variable à une valeur : à un
certain nombre, comme \tm{72} ; à une chaîne, <<\tm{telle que ça}>>, à une
liste, comme \tm{(spruce pine oak)}; nous pouvons même lier une
variable à une définition de fonction. 

Un symbole peut être lié à une valeur de plusieurs façons. Voir la
section\cfchs{1}{9} <<Réglage de la valeur d'une variable>>,
page\cfchsg{1}{9}, pour des informations sur une façon de le faire.  

\subsection{Message d'erreur pour un symbole sans
  fonction}\etchss{1}{7}{1}

Lorsque nous avons évalué \tm{fill-column} pour trouver sa valeur en
tant que variable, nous n'avons pas placé des parenthèses autour du
mot. C'est parce que nous n'avons pas l'intention de l'utiliser comme
un nom de fonction.

Si \tm{fill-column} était le premier ou le seul élément d'une liste,
l'interprète Lisp tenterait de trouver la définition de fonction
attachée à elle. Mais \tm{fill-column} n'a pas de définition de
fonction. Essayez d'évaluer ceci :
\begin{center}
  \tm{(fill-column)}
\end{center}

Vous allez créé un tampon \tm{*Backtrace*} disant ceci:
{\ttfamily
\begin{flushleft}
  Debugger entered--Lisp error: (void-function fill-column)

  (fill-column)

  eval((fill-column))

  eval-last-sexp-1(nil)

  eval-last-sexp(nil)

  call-interactively(eval-last-sexp)
\end{flushleft}
}

(Rappelez-vous, pour quitter le débogueur et faire la fenêtre du
débogueur s'en aller, tapez \tm{q} dans le tampon \tm{*Backtrace*}.)


\subsection{Message d'erreur pour un symbole sans
  valeur}\etchss{1}{7}{2}

Si vous essayez d'évaluer un symbole qui ne possède pas une valeur qui
lui est liée, vous recevrez un message d'erreur. Vous pouvez voir cela
en expérimentant avec notre addition 2 plus 2. Dans l'expression
suivante, placez votre curseur à droite après le \tm{+}, avant le
premier nombre 2, tapez \rec{C}{x}{C}{e} :
\begin{center}
  \tm{(+ 2 2)}
\end{center}
Vous allez créé un tampon \tm{*Backtrace*} disant ceci:
{\ttfamily
\begin{flushleft}
  Debugger entered--Lisp error: (void-variable +)

  eval(+)

  eval-last-sexp-1(nil)

  eval-last-sexp(nil)

  call-interactively(eval-last-sexp)
\end{flushleft}
}

(Encore une fois, vous pouvez quitter le débogueur en tapant \tm{q}
dans le tampon \tm{*Backtrace*}.)

Ce backtrace est différent du premier message d'erreur que nous avons
vu, qui dit <<\tm{Debugger entered--Lisp error: (void-function
  this)}>>. Dans ce cas, la fonction n'a pas de valeur en tant que
variable; tandis que dans l'autre message d'erreur, la fonction (le
mot <<\tm{this}>>) n'avait pas de définition.

Dans cette expérience avec le \tm{+}, ce que nous faisions était
provoquer l'interprète Lisp évaluer le \tm{+} et rechercher la valeur
de la variable au lieu de la définition de la fonction. Nous l'avons
fait en plaçant le curseur juste après le symbole plutôt qu'après la
parenthèse de la liste englobante que nous avons fait avant. En
conséquence, l'interprète Lisp a évalué la s-expression précédente,
qui dans ce cas était \tm{+} lui-même.

Depuis \tm{+} n'a pas de valeur liée à elle, juste la définition de
fonction, le message d'erreur rapportait que la valeur du symbole
comme variable était nulle. 

